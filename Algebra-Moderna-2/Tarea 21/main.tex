\documentclass[a4paper,12pt]{article}
\usepackage[top = 2.5cm, bottom = 2.5cm, left = 2.5cm, right = 2.5cm]{geometry}
\usepackage[T1]{fontenc}
\usepackage[utf8]{inputenc}
\usepackage{multirow} 
\usepackage{booktabs} 
\usepackage{graphicx}
\usepackage[spanish]{babel}
\usepackage{setspace}
\setlength{\parindent}{0in}
\usepackage{float}
\usepackage{fancyhdr}
\usepackage{amsmath}
\usepackage{amssymb}
\usepackage{amsthm}
\usepackage[numbers]{natbib}
\newcommand\Mycite[1]{%
	\citeauthor{#1}~[\citeyear{#1}]}
\usepackage{graphicx}
\usepackage{subcaption}
\usepackage{booktabs}
\usepackage{etoolbox}
\usepackage{minibox}
\usepackage{hyperref}
\usepackage{xcolor}
\usepackage[skins]{tcolorbox}
%---------------------------

\newtcolorbox{cajita}[1][]{
	 #1
}

\newenvironment{sol}
{\renewcommand\qedsymbol{$\square$}\begin{proof}[\textbf{Solución.}]}
	{\end{proof}}

\newenvironment{dem}
{\renewcommand\qedsymbol{$\blacksquare$}\begin{proof}[\textbf{Demostración.}]}
	{\end{proof}}

\newtheorem{problema}{Problema}
\newtheorem{definicion}{Definición}
\newtheorem{ejemplo}{Ejemplo}
\newtheorem{teorema}{Teorema}
\newtheorem{corolario}{Corolario}[teorema]
\newtheorem{lema}[teorema]{Lema}
\newtheorem{prop}{Proposición}
\newtheorem*{nota}{\textbf{NOTA}}
\renewcommand\qedsymbol{$\blacksquare$}
\usepackage{svg}
\usepackage{tikz}
\usepackage[framemethod=default]{mdframed}
\global\mdfdefinestyle{exampledefault}{%
linecolor=lightgray,linewidth=1pt,%
leftmargin=1cm,rightmargin=1cm,
}




\newenvironment{noter}[1]{%
\mdfsetup{%
frametitle={\tikz\node[fill=white,rectangle,inner sep=0pt,outer sep=0pt]{#1};},
frametitleaboveskip=-0.5\ht\strutbox,
frametitlealignment=\raggedright
}%
\begin{mdframed}[style=exampledefault]
}{\end{mdframed}}
\newcommand{\linea}{\noindent\rule{\textwidth}{3pt}}
\newcommand{\linita}{\noindent\rule{\textwidth}{1pt}}

\AtBeginEnvironment{align}{\setcounter{equation}{0}}
\pagestyle{fancy}

\fancyhf{}









%----------------------------------------------------------
\lhead{\footnotesize Álgebra Moderna}
\rhead{\footnotesize  Rudik Roberto Rompich}
\cfoot{\footnotesize \thepage}


%--------------------------

\begin{document}
 \thispagestyle{empty} 
    \begin{tabular}{p{15.5cm}}
    \begin{tabbing}
    \textbf{Universidad del Valle de Guatemala} \\
    Departamento de Matemática\\
    Licenciatura en Matemática Aplicada\\\\
   \textbf{Estudiante:} Rudik Roberto Rompich\\
   \textbf{Correo:}  \href{mailto:rom19857@uvg.edu.gt}{rom19857@uvg.edu.gt}\\
   \textbf{Carné:} 19857
    \end{tabbing}
    \begin{center}
        MM2035 - Álgebra Moderna - Catedrático: Ricardo Barrientos\\
        \today
    \end{center}\\
    \hline
    \\
    \end{tabular} 
    \vspace*{0.3cm} 
    \begin{center} 
    {\Large \bf  Tarea 13
} 
        \vspace{2mm}
    \end{center}
    \vspace{0.4cm}
%--------------------------
Problemas 2, 4, 5 y 12, sección 5.1.


\begin{problema}[Problema 2]
    Let $F$ be a field and let $F[x]$ be the ring of polynomials in $x$ over $F$. Let $g(x)$, of degree $n$, be in $F[x]$ and let $V = (g(x)) $be the ideal generated by $g(x)$ in $F[x]$. Prove that $F[x]/V$ is an $n$-dimensional vector space over $F$.
    \begin{dem}
        Usando la demostración del \textbf{teorema 5G demostrado en clase}: 
        Por el lema $3.22, V=((g(x)))$ es un ideal maximal de $F$ en $F[x] \Longrightarrow$ por el teorema 3B, $F[x] /((p))$ es un campo. Si $f(x)+[g(x)] \in F[x] /(g(x))$ con $f(x) \in F[x]$, aplicando el algoritmo de la división en $F[x]$ (lema 3.17), a $f(x)$ y $g(x), \exists q(x), r(x) \in F[x], r(x)=0$ o $r(x)=\sum_{i=0}^{g r(g)-1} \alpha_i x^i$, i.e. $gr(r)<gr(p) \Longrightarrow$ $f(x)+[g(x)]=(q(x) g(x)+r(x))+[g(x)]=[q(x) g(x)+[g(x)]]+[r(x)+[g(x)]]=$ $[0+[g(x)]]+[r(x)+[g(x)]]=[g(x)]+[r(x)+[g(x)]]=r(x)+[g(x)]=$ $\sum_{i=0}^{g r(g)-1} \alpha_i x^i+[g(x)]=\sum_{i=0}^{g r(g)-1}\left(\alpha_i+[g(x)]\right)=\sum_{i=0}^{g r(g)-1} \alpha_i\left(x^i+[g(x)]\right)=$ $\sum_{i=0}^{g r(g)-1} \alpha_i(x+(g(x)))^i \Longrightarrow F[x] /(g(x))=\left\langle\left\{1, \cdots,(x+[g(x)])^{g r(g)-1}\right\}\right\rangle_F$ Sea $\phi: F[x] \rightarrow F[x] /(g(x))$. Si $\beta_0, \cdots, \beta_{g r(g)-1} \in F \ni((g(x)))=\sum_{i=0}^{g r(g)-1} \beta_i(x+$ $[g(x)])^i=\left(\sum_{i=0}^{g r(g)-1} \beta_i x^i\right)+[g(x)]$. Sea $z(x)=\sum_{i=0}^{g r(g)-1} \beta_i x^i \in F[x] \Longrightarrow[g(x)]=$ $z(x)+[g(x)] \Longrightarrow z(x) \in[g(x)] \Longrightarrow p(x) \mid z(x) \Longrightarrow g r(g)-1 \geq g r(g) \geq$ $g r(g) \Longrightarrow g(x)=0 \Longrightarrow \beta_0=\cdots=\beta_{g r(g)-1} \Longrightarrow\left\{1, \cdots,(x+[g(x)]]^{g r(g)-1}\right\}$ es linealmente independiente es $F[x] /(g(x))$ sobre $F \implies \left\{1, \cdots,(x+[g(x)])^{g r(g)-1}\right\}$ es una base de $F[x] /(g(x))$ sobre $F$. Por lo tanto, $F[x]/V$ es un espacio vectorial $n-$dimensional sobre $F$.  
    \end{dem}

\end{problema}

\begin{problema}[Problema 4]
    Let 
    \begin{enumerate}
        \item Let $R$ be the field of real numbers and $Q$ the field of rational numbers. In $R, \sqrt{2}$ and $\sqrt{3}$ are both algebraic over $Q$. Exhibit a polynomial of degree 4 over $Q$ satisfied by $\sqrt{2}+\sqrt{3}$.
        \begin{sol}
            Sea 
            \begin{align*}
                \implies& &x &= \sqrt{2}+\sqrt{3}\\
                \implies& &x-\sqrt{2} &= \sqrt{3}\\
                \implies& &(x-\sqrt{2})^2 &= (\sqrt{3})^2\\
                \implies& &(x-\sqrt{2})^2 &= (\sqrt{3})^2\\
                \implies& &x^2-2x\sqrt{2}+2 &= 3\\
                \implies& &x^2-1 &= 2\sqrt{2}x\\
                \implies& &(x^2-1)^2 &= (2\sqrt{2}x)^2\\
                \implies& &x^4-2x^2+1 &= 8x^2\\
                \implies& &x^4-10x^2+1 &= 0
            \end{align*}
        \end{sol}
        \item What is the degree of $\sqrt{2}+\sqrt{3}$ over $Q$ ? Prove your answer.
        \begin{sol}
            Nótese que la factorización del polinomio sobre Q es única, entonces:
            $$x^4-10x^2+1 = \left(x-\sqrt{5+2\sqrt{6}}\right) \left(x-\sqrt{5-2\sqrt{6}}\right) \left(x+\sqrt{5-2\sqrt{6}}\right) \left(x+\sqrt{5+2\sqrt{6}}\right),$$
            la cual es grado 4 y por lo tanto el grado de $\sqrt{2}+\sqrt{3}$ sobre $Q$ es 4. 
        \end{sol}
        \item What is the degree of $\sqrt{2} \sqrt{3}$ over $Q$?
        \begin{sol}
            Nótese que $\sqrt{2}\sqrt{3}=\sqrt{6}$ y además el polinomio mínimo sobre $Q$ que satisface $\sqrt{6}$ es $f(x)=x^2-6$. Por lo tanto, el grado $\sqrt{2} \sqrt{3}$ sobre $Q$ es 2. 
        \end{sol}
    \end{enumerate}
\end{problema}

\begin{problema}[Problema 5]
    With the same notation as in Problem 4 , show that $\sqrt{2}+\sqrt[3]{5}$ is algebraic over $Q$ of degree 6.
    \begin{dem}
        Sea 
        \begin{align*}
            \implies& &x &= \sqrt{2}+\sqrt[3]{5}\\
            \implies& &x-\sqrt{2} &= \sqrt[3]{5}\\
            \implies& &(x-\sqrt{2})^3 &= 5\\
            \implies& &x^3-3\sqrt{2}x^2+6x-2\sqrt{2} &= 5\\
            \implies& &  (x^3+6x-5) &= 5\sqrt{2}\\
            \implies& &  (x^3+6x-5)^2 &= 50\\
            \implies& &  x^6+12x^4-10x^3+36x^2-60x-25&= 0
        \end{align*}
        Considerando la factorización de este polinomio, sabemos que pertenece a $\mathbb{C}$, el cual es de grado 6 y por lo tanto el grado  del algebraico $\sqrt{2}+\sqrt[3]{5}$ sobre $Q$ es 6. 
    \end{dem}
\end{problema}

\begin{problema}[Problema 12]
    If $a$ is an algebraic integer and $m$ is an ordinary integer, prove
    \begin{enumerate}
        \item $a+m$ is an algebraic integer.
        \begin{dem}
            Sea 
            \begin{align*}
                f(x) &= \sum_{i=0}^n\alpha_ix^i\in \mathbb{Z}[x],
            \end{align*}
            el polinomio mónico que satisface $a$. Ahora bien, sea $k=a+m$ tal que $a=k-m$, entonces:
            \begin{align*}
                g(k) = f(k-m) &= \sum_{i=0}^n\alpha_i(k-m)^i,
            \end{align*}
            el cual también es mónico y 
            $$g(a+m)=f(a)=0,$$
            cumpliendo la definición para que $a+m$ sea un entero algebraico. 
        \end{dem}   
        \item $m a$ is an algebraic integer.
        \begin{dem}
            Sea 
            \begin{align*}
                f(x) &= \sum_{i=0}^n\alpha_ix^i\in \mathbb{Z}[x],
            \end{align*}
            el polinomio mónico que satisface $a$. Ahora bien, sea $g(x)=m^nf(x)\in\mathbb{Z}[x]$ entonces: 
            $$0=g(a)=m^n\sum_{i=0}^n\alpha_ia^i=\sum_{i=0}^n m^{n-i}\alpha_i(ma)^i,$$
            cumpliendo la definición para que $am$ sea un entero algebraico. 
        \end{dem}  
    \end{enumerate}
\end{problema}


%---------------------------
%\bibliographystyle{apa}
%\bibliography{referencias.bib}

\end{document}