Clase: 18/08/2022

\begin{cajita}
    $(x^2-2)(x^2+1)$\\
    $\left(x^3-2\right)=\{f(x)(x^3-2):f(x)\in \mathbb{Q}[x]\}$
\end{cajita}

\begin{ejemplo}
    $x^2-2\in\mathbb{Q}[x]$, irreducible sobre $\mathbb{Q}\implies$ por el lema 3.22 $(x^3-2)$ es un ideal maximal de $\mathbb{Q}[x]\implies$ por el teorema 3B, $\mathbb{Q}[x]/(x^3-1)$. Se verificará detalladamente este hecho, nótese que $\mathbb{Q}[x]/(x^3-2)=\{f(x)+[x^3-2]: f(x)\in \mathbb{Q}[x]\}$. Por el algoritmo de la división en $\mathbb{Q}[x],\exists q(x),r(x)\in \mathbb{Q}[x]\ni f(x)=q(x)(x^2-2)+r(x)$, con $r(x)=0$ o $gr(r)<gr(x^3-2)=3\implies \exists a_0,a_1,a_2\in\mathbb{Q}\ni r(x)=a_0+a_1x+a_2x^2\implies f(x)+[x^3-2]=(q(x)(x-2)+r(x))+[x^3-2]= q(x)(x^3-2)+[x^3-2]+r(x)+[x^3-2]=(x^3-2)+r(x)+[x^3-2]=r(x)+[x^3-2] = (a_0 +a_1x+a_2x^2)+(x^3-2)=[a_0+[x^3-2]]+[a_1 x+ [x^3-2]]+[a_2x^2+[x^3-2]] = a_0[x+[x^3-2]]^0 + a_1[x+[x^3-2]]^1+a_2[x+[x^3-2]]^2$
    \begin{cajita}
        $$q(x)(x^3-2)+[x^3-2]=0+[x^3-2]=[x^3-2]$$
    \end{cajita}

    Sea $\alpha = x+[x^3-2]\in \mathbb{Q}[x]/(x^3-2)$, con lo cual $f(x)+[x^3-2]=a_0+a_1\alpha +a_2\alpha^2$

    \begin{cajita}
        $$\langle\{1,\alpha^1,\alpha^2\} \rangle_{\mathbb{Q}}$$
    \end{cajita}

    $\implies \langle\{1,\alpha^1,\alpha^2\} \rangle_{\mathbb{Q}}=\mathbb{Q}[x]/(x^3-2)$. Por otro lado, nótese que $\alpha^3 -2 \approx [x+(x^3-2)]^3-[2+(x^3-2)]=[x^3+(x^3-2)]-[2+(x^3-2)]= (x^3-2)+[x^3-2]=0+[x^3-2]=[x^3-2]\approx 0\implies \alpha$ es una raíz de $x^3-2\implies \alpha \in\mathbb{Q}[x]/(x^3-2)-\mathbb{Q}$ (Nótese que si $a\in\mathbb{Q}\implies a\approx a +(x^3-2)\in \mathbb{Q}[x]/(x^3-2)$, con lo cual, $\mathbb{Q}\subseteq \mathbb{Q}[x]/(x^3-2)$) contiene una raíz de $x^2-2$. Si $\exists a_0,a_1,a_2\in\mathbb{Q}$ no todos cero $\ni a_0+a_1\alpha +a_2\alpha^2=0\implies -a_0-a_2\alpha^2 =a_1\alpha \in\mathbb{Q}(\to\gets)\implies \{1,\alpha,\alpha^2\}$ es linealmente independiente sobre $\mathbb{Q}\implies \{1,\alpha,\alpha^2\}$ es una base para el espacio vectorial $(\mathbb{Q}[X]/(x^3-2),+,\cdot, \mathbb{Q})\implies \operatorname{dim}\left(\mathbb{Q}[x]/(x^3-2),+,\cdot, \mathbb{Q}\right)=3=\operatorname{gr}(x^3-2)$. Por otro lado, si $a_0,a_1,a_2,b_0,b_1,b_2\in \mathbb{Q}\ni a_0+a_1\alpha +a_2\alpha^2 =f(x)+(x^3-2)=a_0(x+(x^3-2))+a_1(x+(x^3-2))+a_2(x+(x^3-2))^2=f(x)+(x^2-2)=b_0(x^0+(x^3-2))+b_1(x+(x^3-2))+b^2(x^2+(x^2-2))\implies (a_0-b_0)(x^0+(x^3-2))+(a_1-b_1)(x+(x^2-2))+(a_2-b_2)(x^2+(x^3-2))=(a_0-b_0)+(a_1-b_1)x+(a_2-b_2)x^2+[x^2-2]=[x^3-2]=0+[x^2-2]\implies (a_0-b_0)+(a_1-b_1)x+(a_2-b_2)x^2\equiv 0 \mod (x^2-2)\implies x^3-2|(a_0-b_0)+(a_1-b_1)x+(a_2-b_2)x^2\implies a_0-b_0=a_1-b_1 = a_2-b_2=0\implies a_0=b_0,a_1=b_1$ y $a_2=b_2\implies$ todo elemento $f(x)+(x^3-2)$ de $\mathbb{Q}[x]/(x^3-2)$ de $\mathbb{Q}[x]/(x^3-2)$ tiene representación única como polinomio cuadrático en $\alpha$ sobre $\mathbb{Q}$. \break 
    Sea $a_0+a_1\alpha +a_2\alpha^2\in \mathbb{Q}[x]/(x^2-2)-\{(x^3-2)\}\implies a_0,a_1$ y $a_2$ no son todos cero. El lema 3.22 asegura que $\mathbb{Q}[x]/(x^3-2)$ es un campo $\implies \exists b_0,b_1,b_2\in\mathbb{Q}\ni 1=(a_0+a_1\alpha +a_2\alpha^3)(b_0+b_1\alpha +b_2\alpha^2)= a_0b_0+a_0b_1\alpha +a_0b_2\alpha^2 +a_1b_0\alpha +a_1b_1\alpha^2 +a_1b_2\alpha^3 +a_2b_0\alpha^2 +a_2b_1\alpha^3+a_2b_2\alpha^4 = a_0b_0 +a_0b_1\alpha + a_0b_2\alpha^2 + a_1b_0\alpha +a_0b_1\alpha^2+2a_1b_2+ a_2b_0\alpha^2 +2a_2b_1 + 2a_2b_2\alpha = (a_0b_0 + 2a_1+b_2+2a_2b_1)+(a_0b_1 +a_1b_0+2a_2b_2)+(a_0b_2+a_1b_1+a_2b_0)\alpha^2\implies$ resolviendo el sistema de ecuaciones... por medio de la regla de Cramer, encontramos que el determinantes es $a_0^3+2a_1^3+4a_2^3- 6a_0a_1a_2\neq$. Nótese que $a_0=p_0/q_0,a_1=p_1/q_1,a_2=p_2/q_2\in \mathbb{Q}\ni \cdots \cdots \cdots $
\end{ejemplo}