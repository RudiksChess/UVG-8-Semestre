\documentclass[a4paper,12pt]{article}
\usepackage[top = 2.5cm, bottom = 2.5cm, left = 2.5cm, right = 2.5cm]{geometry}
\usepackage[T1]{fontenc}
\usepackage[utf8]{inputenc}
\usepackage{multirow} 
\usepackage{booktabs} 
\usepackage{graphicx}
\usepackage[spanish]{babel}
\usepackage{setspace}
\setlength{\parindent}{0in}
\usepackage{float}
\usepackage{fancyhdr}
\usepackage{amsmath}
\usepackage{amssymb}
\usepackage{amsthm}
\usepackage[numbers]{natbib}
\newcommand\Mycite[1]{%
	\citeauthor{#1}~[\citeyear{#1}]}
\usepackage{graphicx}
\usepackage{subcaption}
\usepackage{booktabs}
\usepackage{etoolbox}
\usepackage{minibox}
\usepackage{hyperref}
\usepackage{xcolor}
\usepackage[skins]{tcolorbox}
%---------------------------

\newtcolorbox{cajita}[1][]{
	 #1
}

\newenvironment{sol}
{\renewcommand\qedsymbol{$\square$}\begin{proof}[\textbf{Solución.}]}
	{\end{proof}}

\newenvironment{dem}
{\renewcommand\qedsymbol{$\blacksquare$}\begin{proof}[\textbf{Demostración.}]}
	{\end{proof}}

\newtheorem{problema}{Problema}
\newtheorem{definicion}{Definición}
\newtheorem{ejemplo}{Ejemplo}
\newtheorem{teorema}{Teorema}
\newtheorem{corolario}{Corolario}[teorema]
\newtheorem{lema}[teorema]{Lema}
\newtheorem{prop}{Proposición}
\newtheorem*{nota}{\textbf{NOTA}}
\renewcommand\qedsymbol{$\blacksquare$}
\usepackage{svg}
\usepackage{tikz}
\usepackage[framemethod=default]{mdframed}
\global\mdfdefinestyle{exampledefault}{%
linecolor=lightgray,linewidth=1pt,%
leftmargin=1cm,rightmargin=1cm,
}




\newenvironment{noter}[1]{%
\mdfsetup{%
frametitle={\tikz\node[fill=white,rectangle,inner sep=0pt,outer sep=0pt]{#1};},
frametitleaboveskip=-0.5\ht\strutbox,
frametitlealignment=\raggedright
}%
\begin{mdframed}[style=exampledefault]
}{\end{mdframed}}
\newcommand{\linea}{\noindent\rule{\textwidth}{3pt}}
\newcommand{\linita}{\noindent\rule{\textwidth}{1pt}}

\AtBeginEnvironment{align}{\setcounter{equation}{0}}
\pagestyle{fancy}

\fancyhf{}









%----------------------------------------------------------
\lhead{\footnotesize Álgebra Moderna}
\rhead{\footnotesize  Rudik Roberto Rompich}
\cfoot{\footnotesize \thepage}


%--------------------------

\begin{document}
 \thispagestyle{empty} 
    \begin{tabular}{p{15.5cm}}
    \begin{tabbing}
    \textbf{Universidad del Valle de Guatemala} \\
    Departamento de Matemática\\
    Licenciatura en Matemática Aplicada\\\\
   \textbf{Estudiante:} Rudik Roberto Rompich\\
   \textbf{Correo:}  \href{mailto:rom19857@uvg.edu.gt}{rom19857@uvg.edu.gt}\\
   \textbf{Carné:} 19857
    \end{tabbing}
    \begin{center}
        MM2035 - Álgebra Moderna - Catedrático: Ricardo Barrientos\\
        \today
    \end{center}\\
    \hline
    \\
    \end{tabular} 
    \vspace*{0.3cm} 
    \begin{center} 
    {\Large \bf  Tarea 13
} 
        \vspace{2mm}
    \end{center}
    \vspace{0.4cm}
%--------------------------

Problemas 1, 3 y 4, sección 3.5.
\section*{Sección 3.5}
\begin{problema}[Problema 1]
    Let $R$ be a ring with unit element, $R$ not necessarily commutative, such that the only right-ideals of $R$ are $(0)$ and $R$. Prove that $R$ is a division ring.
   \begin{dem}
        Debemos probar que $R$ es un anillo de división, es decir que para cada elemento $a\in R-\{0\}, aa^{-1}=1$. Tenemos dos casos:
        \begin{itemize}
            \item Si $R=(0)$, el resultado es trivial
            \item Si $R\neq(0)$, sea $x\in R-\{0\}$, el cual es un ideal derecho $xR$ y $x=x\cdot 1\in xR,xR\neq 0$. Entonces tenemos $xR=R\implies \exists y \in R \ni x\cdot y=1.$
        \end{itemize}
        
        Por lo tanto, $R$ es un anillo de división. 
   \end{dem} 
\end{problema}

\begin{problema}[Problema 3]

    Let $J$ be the ring of integers, $p$ a prime number, and ($p)$ the ideal of $J$ consisting of all multiples of $p$. Prove
    \begin{itemize}
        \item $J /(p)$ is isomorphic to $J_{p}$, the ring of integers $\bmod p$.
        \begin{dem}
            Debemos probar que $J /(p)$ es isomorfo a $J_{p}$. Proponemos una función $\phi: J/(p)\to J_p$ de la forma 
            $$\phi((p)+n)=n\bmod p$$
            Entonces, comprobaremos las siguientes propiedades: 
            \begin{itemize}
                \item Función bien definida. 
                    \begin{itemize}
                        \item Sea $\phi((p)+n)\in J_p,\forall (p)+n\in J/(p)$.
                        \item Supóngase que $(p)+n=(p)+n_0$ y por la hipótesis tenemos que $(p)$ es el ideal de $J$ que consiste en todos los múltiplos de $p$, es decir que $n=n_0+kp,\forall k\in\mathbb{Z}$. Ahora bien, tenemos:
                        
                        \begin{align*}
                            \phi((p)+n)&= n\bmod p\\
                                       &= n\bmod p\\
                                       &= (n_0+kp) \bmod p\\
                                       &= n_0 \bmod p\\
                                       &= \phi((p)+n_0)
                        \end{align*}
                        Por lo tanto, hemos probado las dos propiedades para ser una función bien definida. 
                        
                    \end{itemize}
                    \item Ahora bien, intentaremos probar que $\phi$ es isomorfismo, primero demostrando que es homomorfismo y luego que es inyectivo. 
                        \begin{itemize}
                            \item Sea $\phi$ un homomorfismo, tal que: 
                            \begin{enumerate}
                                \item Para la suma. Sea 
                                \begin{align*}
                                    \phi([(p)+n_1]+[(p+n_2)]) &= \phi((p)+[n_1+n_2])\\
                                    &= (n_1+n_2) \bmod p\\
                                    &= n_1 \bmod p+n_2\bmod p\\
                                    &= \phi((p)+n_1) +\phi((p)+n_2)
                                \end{align*}
                                \item Para la multiplicación. Sea 
                                \begin{align*}
                                    \phi([(p)+n_1]\cdot[(p+n_2)]) &= \phi((p)+[n_1\cdot n_2])\\
                                    &= (n_1\cdot n_2) \bmod p\\
                                    &= (n_1 \bmod p)\cdot (n_2\bmod p)\\
                                    &= \phi((p)+n_1)\cdot \phi((p)+n_2)
                                \end{align*} 
                                
                            \end{enumerate}
                            Por lo tanto, $\phi$ es un homomorfismo para la suma y la multiplicación.
                        \end{itemize}
                        \item Ahora bien, demostraremos que también se cumple la inyectividad. Se a
                        \begin{align*}
                            \phi((p)+n_1) &= \phi((p)+n_2)\\
                            n_1&\equiv n_2 \mod p\\
                            n_1-n_2&\equiv 0 \mod p\\
                            n_1-n_2&= kp \quad k\in \mathbb{Z}
                            \intertext{De esto, tenemos que $n_1-n_2\in (p)$, lo que nos permite concluir que:}
                            \phi((p)+n_1) &= \phi((p)+n_2)
                        \end{align*}
            \end{itemize}
            Por lo tanto, se cumple una de las definición de isomorfismo para $J/(p)$ y $J_p$
        \end{dem}
        \item Using Theorem 3.5.1 and part (a) of this problem, that $J_{p}$ is a field.
        \begin{dem}
            En clase demostramos que en el anillo de los enteros $(J,+,\cdot)$, $(p)$ es un ideal maximal de $J$ si y solo si $p$ es primo $\implies$ por el teorema 3.5.1, $J/(p)$ es un campo, pero por el inciso (a) de este problema $J/(p)\approx J_p$. Por lo tanto, $J_p$ es un campo. 
        \end{dem}
    \end{itemize}

\end{problema}

%---------------------------
%\bibliographystyle{apa}
%\bibliography{referencias.bib}

\end{document}