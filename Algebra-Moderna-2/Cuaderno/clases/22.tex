Clase: 29/09/2022
\begin{teorema}[5G]
    Si $F$ es un campo, $p(x)\in F[x],gr(p)\geq 1$, irreducible sobre $F$, entonces existe $E$, extensión de $F$ tal que $[E:F]=gr(p)$ y $E$ contiene por lo menos una raíz de $p(x)$.
    \begin{dem}
        Por el lema 3.22, $((p(x)))$ es un ideal maximal de $F$ en $F[x]\implies$ por el teorema 3B, $F[x]/((p))$ es un campo. Si $f(x)+[p(x)]\in F[x]/(p(x))$ con $f(x)\in F[x]$, aplicando el algoritmo de la división en $F[x]$ (lema 3.17), a $f(x)$ y $p(x), \exists q(x),r(x)\in F[x],r(x)=0$ o $r(x)=\sum_{i=0}^{gr(p)-1}\alpha_i x^i$, i.e. $g(r)<gr(p)\implies f(x)+[p(x)]=(q(x)p(x)+r(x))+[p(x)]=[q(x)p(x)+[p(x)]]+[r(x)+[p(x)]]=[0+[p(x)]]+[r(x)+[p(x)]]=[p(x)]+[r(x)+[p(x)]]=r(x)+[p(x)]=\sum_{i=0}^{gr(p)-1}\alpha_i x^i+[p(x)]=\sum_{i=0}^{gr(p)-1}(\alpha_i + [p(x)])=\sum_{i=0}^{gr(p)-1}\alpha_i (x^i +[p(x)])=\sum_{i=0}^{gr(p)-1}\alpha_i (x+(p(x)))^i\implies F[x]/(p(x))=\langle\{1,\cdots, (x+[p(x)])^{gr(p)-1}\}\rangle_F$ Sea $\phi:F[x]\to F[x]/(p(x))$. Si $\beta_0,\cdots, \beta_{gr(p)-1}\in F\ni ((p(x)))=\sum_{i=0}^{gr(p)-1}\beta_i(x+[p(x)])^i = \left(\sum_{i=0}^{gr(p)-1}\beta_i x^i\right)+[p(x)]$. Sea $g(x)=\sum_{i=0}^{gr(p)-1}\beta_i x^i\in F[x]\implies [p(x)]=g(x)+[p(x)]\implies g(x)\in [p(x)]\implies p(x)|g(x)\implies gr(p)-1\geq gr(q)\geq gr(p)\implies p(x)=0\implies \beta_0=\cdots =\beta_{gr(p)-1}\implies \{1,\cdots,(x+[p(x)])^{gr(p)-1}\}$ es linealmente independiente es $F[x]/(p(x))$ sobre $F$ $\implies\{1,\cdots, (x+[p(x)])^{gr(p)-1}\}$ es una base de $F[x]/(p(x))$ sobre $F$. Nótese que además, $p(x+[p(x)])=p(x)+[p(x)]=0+[p(x)]=[p(x)]\implies x+[p(x)]\in F[x]/(p(x))$ es una raíz de $p(x)$. Si $\phi:F\to F[x]/(p(x))\ni \phi(\alpha)=\alpha+[p(x)]$ y nótese que $\phi(\alpha_1+\alpha_2)=(\alpha_1+\alpha_2)+[p(x)]=(\alpha +[p(x)])+(\alpha_2+[p(x)])=\phi(\alpha_1)+\phi(\alpha_2)$ y $\phi(\alpha_1\alpha_2)=\alpha_1\alpha_2 +[p(x)]=(\alpha_1+[p(x)])(\alpha_2+[p(x)])=\phi(\alpha_1)\phi(\alpha_2)\implies\phi$ es un homomorfismo. Sea $\alpha\in K_\phi\implies \phi(\alpha)=\alpha + [p(x)]=[p(x)]\implies \alpha\in [p(x)]\implies p(x)|\alpha \implies \alpha=0\implies K_\phi =(0)\implies \phi$ es inyectivo $F$ está inmerso en $F[x]/(p(x))\implies$ salvo isomorfismo, $F\subseteq F[x]/(p(x))$ y $F[x]/(p(x))$ es la extensión de $F$ requerida. Si $E=F[x]/(p(x))\implies [E:F]=gr(p)$ y $E$ contiene una raíz de $p(x)$. 
    \end{dem}
\end{teorema}