\documentclass[a4paper,12pt]{article}
\usepackage[top = 2.5cm, bottom = 2.5cm, left = 2.5cm, right = 2.5cm]{geometry}
\usepackage[T1]{fontenc}
\usepackage[utf8]{inputenc}
\usepackage{multirow} 
\usepackage{booktabs} 
\usepackage{graphicx}
\usepackage[spanish]{babel}
\usepackage{setspace}
\setlength{\parindent}{0in}
\usepackage{float}
\usepackage{fancyhdr}
\usepackage{amsmath}
\usepackage{amssymb}
\usepackage{amsthm}
\usepackage[numbers]{natbib}
\newcommand\Mycite[1]{%
	\citeauthor{#1}~[\citeyear{#1}]}
\usepackage{graphicx}
\usepackage{subcaption}
\usepackage{booktabs}
\usepackage{etoolbox}
\usepackage{minibox}
\usepackage{hyperref}
\usepackage{xcolor}
\usepackage[skins]{tcolorbox}
%---------------------------

\newtcolorbox{cajita}[1][]{
	 #1
}

\newenvironment{sol}
{\renewcommand\qedsymbol{$\square$}\begin{proof}[\textbf{Solución.}]}
	{\end{proof}}

\newenvironment{dem}
{\renewcommand\qedsymbol{$\blacksquare$}\begin{proof}[\textbf{Demostración.}]}
	{\end{proof}}

\newtheorem{problema}{Problema}
\newtheorem{definicion}{Definición}
\newtheorem{ejemplo}{Ejemplo}
\newtheorem{teorema}{Teorema}
\newtheorem{corolario}{Corolario}[teorema]
\newtheorem{lema}[teorema]{Lema}
\newtheorem{prop}{Proposición}
\newtheorem*{nota}{\textbf{NOTA}}
\renewcommand\qedsymbol{$\blacksquare$}
\usepackage{svg}
\usepackage{tikz}
\usepackage[framemethod=default]{mdframed}
\global\mdfdefinestyle{exampledefault}{%
linecolor=lightgray,linewidth=1pt,%
leftmargin=1cm,rightmargin=1cm,
}




\newenvironment{noter}[1]{%
\mdfsetup{%
frametitle={\tikz\node[fill=white,rectangle,inner sep=0pt,outer sep=0pt]{#1};},
frametitleaboveskip=-0.5\ht\strutbox,
frametitlealignment=\raggedright
}%
\begin{mdframed}[style=exampledefault]
}{\end{mdframed}}
\newcommand{\linea}{\noindent\rule{\textwidth}{3pt}}
\newcommand{\linita}{\noindent\rule{\textwidth}{1pt}}

\AtBeginEnvironment{align}{\setcounter{equation}{0}}
\pagestyle{fancy}

\fancyhf{}









%----------------------------------------------------------
\lhead{\footnotesize Álgebra Moderna}
\rhead{\footnotesize  Rudik Roberto Rompich}
\cfoot{\footnotesize \thepage}


%--------------------------

\begin{document}
 \thispagestyle{empty} 
    \begin{tabular}{p{15.5cm}}
    \begin{tabbing}
    \textbf{Universidad del Valle de Guatemala} \\
    Departamento de Matemática\\
    Licenciatura en Matemática Aplicada\\\\
   \textbf{Estudiante:} Rudik Roberto Rompich\\
   \textbf{Correo:}  \href{mailto:rom19857@uvg.edu.gt}{rom19857@uvg.edu.gt}\\
   \textbf{Carné:} 19857
    \end{tabbing}
    \begin{center}
        MM2035 - Álgebra Moderna - Catedrático: Ricardo Barrientos\\
        \today
    \end{center}\\
    \hline
    \\
    \end{tabular} 
    \vspace*{0.3cm} 
    \begin{center} 
    {\Large \bf  Tarea 13
} 
        \vspace{2mm}
    \end{center}
    \vspace{0.4cm}
%--------------------------
Problemas 1 y 4, sección 3.8.


\begin{problema}[Problema 1]
    Find all the units in $J[i]$.
    \begin{dem}
        Sabemos que $J$ es un anillo euclideano y sea ahora $r=x+yi\in R-\{0\}$, entonces por \textbf{proposición} demostrada en clase $r$ es una unidad de $R\iff d(r)=d(1)$. Es decir entonces que para $r$ tenemos:

        $$d(r)=x^2+y^2 = d(1)=1^2+0^2=1,$$
        entonces las posibles soluciones son: $x=0,y=\pm 1$ o $x=\pm 1+,y=0$. Es decir, $i,-i,1,-1$ son las posibles soluciones de $J[i]$.
    \end{dem}
\end{problema}

\begin{problema}[Problema 4]
    Prove that if $p$ is a prime number of the form $4 n+3$, then there is no $x$ such that $x^2 \equiv-1 \bmod p$.
    \begin{dem}
        Por reducción al absurdo, supóngase que existe un $x$ tal que $x^2\equiv -1\mod p$. Ahora bien, considérese el pequeño teorema de Fermat, tal que 
        \begin{align*}
            &\implies x^p \equiv x\\
            &\iff x^{4n+3}\equiv x \bmod p\\
            &\iff x^{4n}x^3 \equiv 1\cdot x \bmod p\\
            &\iff x^{4n}x^3 \equiv ((1\bmod p)(x \bmod p)) \bmod p
        \end{align*}
        Entonces, se tiene $x^{4n} \equiv 1\bmod p$ y $x^3\equiv x \bmod p$. Nótese que si consideramos a $n=1$, tenemos 
        \begin{align*}
            &\implies x^4 \equiv 1\bmod p\\
            &\iff x^{2}x^2 \equiv ((1\bmod p)(1 \bmod p)) \bmod p
        \end{align*}
        Es decir, $x^2 \equiv 1\mod p$, lo cual es una contradicción a nuestra suposición. Por lo tanto, se cumple que no existe un $x$ tal que $x^2\equiv -1\bmod p$.
    \end{dem}
\end{problema}

%---------------------------
%\bibliographystyle{apa}
%\bibliography{referencias.bib}

\end{document}