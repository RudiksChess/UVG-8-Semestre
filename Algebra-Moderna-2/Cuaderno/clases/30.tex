Clase: 15/11/2022

\begin{teorema}
    Si $\alpha\in\mathbb{B}$ es constructible, si y solo si, existen $\lambda, \cdots,\lambda_n\in \mathbb{R}^+$, tales que 
    \begin{enumerate}
        \item $\lambda_i^2\in \mathbb{Q}$
        \item $\lambda_i^2\in \mathbb{Q}(\lambda_1,\cdots, \lambda_{i-1})$ y $\alpha\in \mathbb{Q}(\lambda_1,\cdots,\lambda_n)$
    \end{enumerate}
    \begin{dem}
        
    \end{dem}
\end{teorema}

\begin{corolario}
    Sea $\alpha\in\mathbb{R}$ es constructible si y solo si $\alpha$ pertenece a una extensión de $\mathbb{Q}$ de grado potencia de 2 sobre $\mathbb{Q}$. 
\end{corolario}

\begin{corolario}
    Si $\alpha\in \mathbb{R}$ satisface un polinomio irreducible en $\mathbb{Q}[x]$ de grado $K$ y $K$ no es un polinomio de 2, entonces $\alpha$ no es constructible. 
\end{corolario}

\begin{ejemplo}
    Múltiples aplicaciones.
\end{ejemplo}

\begin{teorema}[De la raíz racional]
    Si $f(x)=\sum_{i=0}^n a_ix^i \in \mathbb{X}[x],a_n,a_0\in \mathbb{Z}-\{0\}$. Sea $p/q\in \mathbb{Q}\ni f(p/q)=0$ y $(p,q)=1$, entonces $p|a_0$ y $q|a_n$.
    \begin{dem}
        
    \end{dem}
\end{teorema}

\begin{definicion}
    Si $F$ es un campo,
    $$f(x)=\sum_{i=0}^n a_i x^i \in F[x],$$
    entonces la derivada de $f(x)$ es 
    $$f'(x)=\sum_{i=1}^n ia_i x^{i-1}\in F[x]$$
\end{definicion}

\begin{definicion}
    Un dominio entero $D$ es de característica 0 si la relación $ma=0,a\in D-\{0\},m\in \mathbb{Z}$, es válido solo si $m=0$
\end{definicion}


\begin{definicion}
    Un dominio entero $D$ es de característica finita, si existe $m\in \mathbb{Z}^+$ tal que $ma=0$ para todo $a\in D$. Si $D$ es de característica finita, entonces la característica de $D$ es el entero positivo más pequeño $p$ tal que $pa=0$, para todo $a\in D$.
\end{definicion}

\begin{prop}
    Si $D$ es un dominio entero de característica finita, entonces su característica es un número primo. 
\end{prop}

\begin{ejemplo}
    Múltiples ejemplos.
\end{ejemplo}

\begin{lema}[5.5]
    Si $F$ es un campo, $f(x),g(x)\in F[x]$ y $\alpha\in F$ , entonces: 
    \begin{enumerate}
        \item $(f(x)+g(x))'=f'(x)+g'(x)$
        \item $(\alpha f(x))'=\alpha f'(x)$
        \item $(f(x)g(x))'=f'(x)g(x)+f(x)g'(x)$
    \end{enumerate}
\end{lema}
\begin{lema}[5.5]
    Si $F$ es un campo, $K$ es una extensión de $F$ y $f(x),g(x)\in F[x]$ tiene un factor común trivial (i.e de grado positivo ) en $K[x]$, entonces tienen un factor común no trivial en $F[x]$. 
\end{lema}
\begin{lema}[5.6]
    Si $F$ es un campo, entonces $f(x)\in F[X]$ tiene una raíz múltiple, si y solo si, $f(x)$ y $f'(x)$ tienen un factor común no trivial. 
\end{lema}

\begin{corolario}
    Si $F$ es un campo, y $f(x)\in F[x]$ es irreducible sobre $F$, entonces
    \begin{enumerate}
        \item Si la característica de $F$ es 0, entonces $f(x)$ no tiene raíces comunes. 
        \item Si la característica de $F$ es $p\neq 0$, entonces $f(x)$ tiene una raíz múltiple si y solo si, $\exists g(x)\in F[x]\ni f(x)=g(x^p)$
    \end{enumerate}
\end{corolario}

\begin{corolario}
    Si $F$ es un campo de característica de $p\neq 0$ entonces el polinomio $x^{p^n}-x\in F[x],n\in\mathbb{Z}^+$ no tiene raíces múltiples 
\end{corolario}

\begin{definicion}
    Si $F$ es un campo, entonces una extensión $K$ de $F$ es simple si existe $\alpha\in K$ tal que 
\end{definicion}

\begin{definicion}
    Si $F$ es un campo y $K$ es una extensión de $F$, $a\in K$ separable sobre $F$ si satisface un polinomio en $F[x]$
\end{definicion}

\begin{definicion}
    Si $F$ es un campo, entonces una extensión $K$ de $F$ simple si existe $\alpha\in K$ tal que $K=F(\alpha)$
\end{definicion}

\begin{definicion}
    Si $F$ es un campo, y $K$ es una extensión de $F$, $\alpha \in K$ es separable sobre $F$ si satisface un polinomio en $F[x]$ sin raíces múltiples. $K$ es separable sobre $F$ si todos los elementos $K$ son separables sobre $F$. Un campo es perfecto, si todas sus extensiones finitas son separables. 
\end{definicion}


