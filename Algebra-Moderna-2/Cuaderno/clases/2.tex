

Clase: 12/07/2022

\begin{lema}[3.1]
    Si $R$ es un anillo, entonces para $r_1,r_2\in R$.
    \begin{enumerate}
        \item $r_1\cdot 0=0\cdot r_1=0$
        \item $r_1\cdot(-r_2)=(-r_1)\cdot (r_2)= -(r_1\cdot r_2)$
        \item $(-r_1)\cdot (-r_2)=r_1r_2$
        Si además $R$ tiene neutro multiplicativo 1, entonces: 
        \item $(-1)\cdot r_1 =r_1$
        \item $(-1)(-1)=1$
    \end{enumerate}
    \begin{dem}
        \begin{enumerate}
            \item Usando la ley distributiva derecha, $r_1\cdot 0=r_1\cdot(0+0)=r_1\cdot0 +r_1\cdot0\implies$ Por la ley de cancelación en $(R,+)$, $r_1\cdot 0 = 0$. Ahora usando la ley de distributividad izquierda tenemos 
            $0\cdot r_1=(0+0)\cdot r_1 = 0\cdot r_1+0\cdot r_1$, y de nuevo, por la ley de cancelación en el grupo $(R,+)$, $0\cdot r_1=0$. 
            \item $r_1\cdot r_2 +r_1\cdot (-r_2)=r_1\cdot (r_2-r_2)=r_1\cdot 0 =0\implies$ por el (2) del lema 2.1, unicidad de los inversos en los grupos, $r_1\cdot (-r_2)=-r_1\cdot r_2$. Un argumento similar verifica que $(-r_1)\cdot r_2 =-(r_1\cdot r_2)$
            \item $(-r_1)\cdot (-r_2)=-(r_1\cdot (-r_2))=-(-(r_1\cdot r_2))=r_1\cdot r_2 $
            \item Si $\exists 1\in R$, neutro multiplicativo $\implies r_1+(-1)\cdot r_1= (1)r_1+(-1)r_1=(1-1)r_1=0\cdot r_1=0\implies$ Lema 2.1, unicidad de inverso $(-1)r_1=-r_1$.
            \item Caso especial de (iv), haciendo $r_1=-1\implies (-1)(-1)=-(-1)=1$.
        \end{enumerate}
    \end{dem}

\end{lema}

\begin{nota}[El principio de las casillas]
    Para $n,m\in\mathbb{Z}^+,n>m$, si $n$ objetos se distribuyen en $m$ casillas, entonces alguna casilla recibe 2 o más objetos. De manera equivalente, si $n$ objetos se distribuyen en $n$ casillas, de forma que ninguna casilla recibe más de un objeto, entonces todas las casillas reciben exactamente un objeto. 
\end{nota}

\begin{lema}[3.2]
    Un dominio entero finito es un campo. 
    \begin{dem}
        Sea $D$ un dominio entero finito y $D=\{x_1,\cdots, x_n\},n\in \mathbb{Z}^+$. Debemos encontrar: neutro multiplicativo e inversos multiplicativos. Sea $a\in D-\{0\}$ y considérese $ax_a,\cdots, ax_n$. Si $ax_i=ax_j$ con $i\neq j\implies 0 = ax_i -ax_j =a(x_i-x_j)\implies$ Como $a\neq 0$ y $D$ es un dominio entero, y por lo tanto, carece de divisores de 0. $\implies x_i=x_j$ con $i\neq j (\to\gets)\implies ax_1,\cdots,ax_n$ son todos distintos y para el principio de las casillas $D=\{ax_a,\cdots, ax_n\}\implies $ Como $a\in D\implies \exists i,1\leq i\leq n\ni a=ax_{i_0}=x_{i_0}a$. Si $d\in D\implies \exists i_d,1\leq i_d\leq n\ni d=ax_{i_d}\implies dx_{i_d}=(ax_{i_d})x_{i_d}=(x_{i_d}a)x_{i_d}=x_{i_d}(ax_{i_d})=x_{i_d}a=ax_{i_d}=d\implies x_{i_d}=1$ es neutro multiplicativo de $D$. Pero $1\in D\implies \exists i_1,1\leq $
    \end{dem}
\end{lema}

\begin{corolario}
    Si $p$ es un número primo, entonces $(\mathbb{Z}_p,+,\cdot)$ es un campo. 
    \begin{dem}
        Se sabe que $(\mathbb{Z}_n,+,\cdot)$ es un anillo conmutativo $\forall n\in \mathbb{Z}^+$. Si $p$ es un número primo y $\bar{a},\bar{b}\in\mathbb{Z}_p\ni \bar{a}\bar{b}=\bar{0}\implies ab\equiv 0\quad \mod p\implies p|ab\implies p|a$ o $p|b\implies a\equiv 0\mod p$ o $b\equiv 0\mod p\implies \bar{a}=\bar{0}$ o $\bar{b}=\bar{0}\implies \mathbb{Z}_p$ carece de divisores de 0 $\implies \mathbb{Z}_p$ es un dominio entero $\implies$ por el lema 3.2, $\mathbb{Z}_p$ es un campo.
    \end{dem}
\end{corolario}

\begin{definicion}
    Si $(R,+,\cdot)$ y $(R,\oplus, \odot)$ son anillos y $\phi:R\to R'$ es una función, entonces $\phi$ es un homomorfismo.
    \begin{enumerate}
        \item $\phi(r_1+r_2)=\phi(r_1)\oplus \phi(r_2)$
        \item $\phi(r_1\cdot r_2)=\phi(r_1)\odot \phi(r_2)$
    \end{enumerate}
\end{definicion}

\begin{lema}[3.3]
    Si $R$ y $R'$ son anillos y $\phi: R\to R'$ es un homomorfismo entonces:
    \begin{enumerate}
        \item $\phi(0)=0'$
        \item $\phi(-r)=-\phi(r),\forall r\in R$.
    \end{enumerate}
    \begin{dem}
        Se deduce directamente del hecho que $(R,+)$ y $(R',+)$ son grupos y del lema 2.14. 
    \end{dem}
\end{lema}

\begin{ejemplo}
    Si $\phi:\mathbb{Z}_6\to \mathbb{Z}_6\ni \phi(\bar{a})=\bar{0}\implies \phi(\bar{a}_1+\bar{a}_2)=\bar{0}=\bar{0}+\bar{0} =\phi(\bar{a}_1)+\phi(\bar{a}_2)$ y $\phi(\bar{a}_1\bar{a}_2)=\phi(\bar{a}_1)\phi(\bar{a}_2)$ 
    
    que la imagen homomórfica de un neutro multiplicativo no necesariamente es neutro multiplicativo. 
\end{ejemplo}
\begin{prop}
    Si $R$ es un anillo con elemento neutro multiplicativo 1, $R'$ un dominio entero y $\phi:R\to R'$ es un homomorfismo tal que $k_\phi\neq R$, entonces $\phi(1)$ es neutro multiplicativo de $R'$. 
    \begin{dem}
        Tarea.
    \end{dem}
\end{prop}

\begin{prop}
    Si $R$ es un anillo con elemento neutro 1, $R'$ es un anillo y $\phi:R\to R'$ es un homomorfismo sobreyectivo, entonces $\phi(1)$ es neutro multiplicativo de $R'$
    \begin{dem}
        Tarea.
    \end{dem}
\end{prop}

\begin{definicion}
    Si $R$ y $R'$ son anillos y $\phi:R\to R'$ es un homomorfismo, entonces el kernel de $\phi$ es $k_\phi:\{r\in R:\phi(r)=0\}$
\end{definicion}
\begin{lema}[3.4]
    Si $R$ y $R'$ son anillos y $\phi:R\to R'$ es un homomorfismo, entonces:
    \begin{enumerate}
        \item $(K_\theta, +)$ es un subgrupo de $(R,+)$ 
        \item Si $k\in\phi_\theta$ y $r\in R\implies kr,rk\in k_\theta$, es decir el núcleo de $\theta$ atrapa productos.  
    \end{enumerate}
    \begin{dem}
        \begin{enumerate}
            \item Lema 2.15
            \item Si $k\in k_\theta$ y $r\in R\implies \theta(kr)=\theta(k)\theta(r)=0'\cdot \theta(r)=0'=\theta(r)\cdot 0'=\theta(r)\theta(k)=\theta(rk)\implies kr,rk\in K_\theta$
        \end{enumerate}
    \end{dem}
\end{lema}

\begin{ejemplo}
\begin{enumerate}
    \item Si $R$ es un anillo y $\phi:R\to R\ni \phi(r)=r\implies \phi$ es el homomorfismo identidad. 
    \item Si $\mathbb{Z}(\sqrt{2})=\{m+n\sqrt{2}:m,n\in\mathbb{Z}\}\implies (\mathbb{Z}(\sqrt{2}),+,\cdot)$ con $+$ y $\cdot$ la suma y producto usuales de números reales, es un anilo (¡ejercicio!). Si $\phi:\mathbb{Z}(\sqrt{2})\to \mathbb{Z}(\sqrt{2})\ni \phi(m\cdot n \sqrt{2})= m\cdot n\sqrt{2}$. Si $m_1+n_1\sqrt{2}, m_2+n_2\sqrt{2}\in\mathbb{Z}(\sqrt{2})\implies \phi((m_1+n_1\sqrt{2})+(m_2+n_2\sqrt{2}))=\cdots = \phi(m_1+n_1\sqrt{2})\phi(m_2+n_2\sqrt{2})\implies \phi $ es homomorfismo y $k_\theta=\{m+n\sqrt{2}:\phi(m+n\sqrt{2})=m-n\sqrt{2}=0=0-0\sqrt{2}\} =\{0\}\implies \phi$ es un homomorfismo inyectivo.
    \item Si $\theta:\mathbb{Z}\to\mathbb{Z}_n\ni \phi(a)=\bar{a}$. Sean $a,b\in \mathbb{Z}\implies \exists q_1,q_2\in\mathbb{Z}, a=nq_1+\bar{a}$ y $b=nq_2+\bar{b}$ con $0\leq \bar{a}< n$ y $0\leq \bar{b}<n$. Además, $\exists q_3\in\mathbb{Z}\ni a+b =q_3+n +a+b$, con $a\leq \overline{a+b}<n$ y $\exists q_4\in\mathbb{Z}\ni ab=q_4n +\overline{ab}$ con $0\leq \overline{ab}<n$. Ahora bien, nótese lo siguiente: $(nq_1+nq_2)+\bar{a}+\bar{b}=(nq_1+\bar{a})+(nq_2+\bar{b})=a+b=q_3n+qb$. Eso quiere decir: $\overline{a+b}-(\bar{a}-\bar{b})=nq_3-(nq_1+nq_2)=n(q_3-q_1-q_2)\implies n|\overline{a+b} - (\bar{a}+\bar{b})\implies \overline{a+b}=\bar{a}+\bar{b} \mod n$. Además, $(n^2q_1q_2 +nq_1\bar{b}+nq_2 \bar{a})+\bar{a}\bar{b}= \cdots $. Por lo tanto, $\phi$ es homomorfismo, y $k_\phi = n\mathbb{Z}$.
\end{enumerate}    
\end{ejemplo}