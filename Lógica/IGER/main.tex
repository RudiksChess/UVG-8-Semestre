\documentclass[a4paper, 12pt]{article}
\input{settings/packages}
\input{settings/page}
\input{ambientes}

\begin{document}
\newgeometry{margin=2.5cm}
\begin{titlepage}
\thispagestyle{empty}
\newcommand{\HRule}{\rule{\linewidth}{0.5mm}}
\hspace{1cm}
\center

\textsc{\huge IGER}\\[2.0cm]

\HRule\\[1.4cm]
\MSdoublespacing
{ \huge \bfseries INFORME FINAL}\\[0.2cm]
{ \large Curso de apoyo: Matemática Financiera - Cuarto bachillerato}\\[0.3cm] % Falls nicht benötigt, einfach auskommentieren
\HRule \\[2.4cm]
\MSonehalfspacing

\begin{minipage}[t]{0.8\textwidth}
	\begin{itemize}
	\item[\emph{Estudiante:}] Rudik Roberto Rompich Cotzojay
	\item[\emph{Carrera:}] Licenciatura en Matemática Aplicada
	\item[\emph{Carné:}] 19857
	\item[\emph{Correo:}] rom19857@uvg.edu.gt
	\end{itemize}
\end{minipage}

\vspace{2.9cm}

\flushright \today
\end{titlepage}
\restoregeometry

\setstretch{3}
\microtypesetup{protrusion=false}
\tableofcontents
\microtypesetup{protrusion=true}
\thispagestyle{empty}

\MSonehalfspacing
\newpage
\setcounter{page}{1}
\pagestyle{fancy}+
\setcounter{page}{1}
%----------------------------------------

\section{Consolidación de materiales elaborados}

\begin{table}[H]
    \centering
    \resizebox{\linewidth}{!}{%
    \begin{tabular}{lllll}
    No & MATERIAL   & TEMA                                              & SEMANA & FECHA DE ENTREGA \\
    1  & Infografía & Descuento simple                                  & 3      & 15/02/2022       \\
    2  & Video      & Descuento simple                                  & 3      & 15/02/2022       \\
    3  & Infografía & Descuentos sucesivos                              & 4      & 15/02/2022       \\
    4  & Video      & Descuentos sucesivos                              & 4      & 15/02/2022       \\
    5  & Infografía & Comisiones                                        & 5      & 16/05/2022       \\
    6  & Video      & Comisiones                                        & 5      & 16/05/2022       \\
    7  & Infografía & Resumen de las primeras 7 semanas                 & 8      & 16/05/2022       \\
    8  & Video      & Resumen de las primeras 7 semanas                 & 8      & 16/05/2022       \\
    9  & Infografía & Reparto proporcional simple                       & 9      & 16/05/2022       \\
    10 & Video      & Reparto proporcional simple                       & 9      & 16/05/2022       \\
    11 & Infografía & Reparto proporcional compuesto                    & 10     & 16/05/2022       \\
    12 & Video      & Reparto proporcional compuesto                    & 10     & 16/05/2022       \\
    13 & Infografía & Prorrateo de costos                               & 11     & 16/05/2022       \\
    14 & Video      & Prorrateo de costos                               & 11     & 16/05/2022       \\
    15 & Video      & Interés simple                                    & 12     & 16/05/2022       \\
    16 & Infografía & Interés compuesto                                 & 13     & 16/05/2022       \\
    17 & Video      & Interés compuesto                                 & 13     & 16/05/2022       \\
    18 & Infografía & Problemas interesantes interés compuesto y simple & 14     & 16/05/2022       \\
    19 & Video      & Problemas interesantes interés compuesto y simple & 14     & 16/05/2022       \\
    20 & Infografía & Anualidades                                       & 15     & 16/05/2022       \\
    21 & Video      & Anualidades                                       & 15     & 16/05/2022       \\
    22 & Infografía & Interés simple/compuesto                          & 16     & 16/05/2022       \\
    23 & Video      & Interés simple/compuesto                          & 16     & 16/05/2022       \\
    24 & Infografía & Resumen de semanas 9 a 16                         & 17     & 16/05/2022       \\
    25 & Video      & Resumen de semanas 9 a 16                         & 17     & 16/05/2022      
    \end{tabular}%
    }
    \end{table}

\section{Reflexión de aprendizajes}
\begin{itemize}
    \item \textbf{¿Cómo se sintió con su labor como tutor o tutora a distancia?, ¿qué
    experiencias recoge del voluntariado?}

    Fue una experiencia bastante cómoda, la verdad. Ya que este año empecé a trabajar de tiempo completo y al ser virtual, pude mantener a regla todas mis responsabilidad. Las experiencias que recojo es la forma de diseñar materiales didácticas para el área de matemática financiera. 

    \item \textbf{¿Considera que su trabajo fue satisfactorio? ¿Por qué?}
    \item 
    Sí, considero que sí. Hice todo lo que se me requirió. 
    \item \textbf{¿Qué aspectos personales cree que debería mejorar o continuar teniendo
    para una próxima oportunidad?}

    Pienso que quizás debería haberme organizado un poco mejor, en lugar de haber mandado todos los materiales en dos fechas. 
\end{itemize}


\section{Retroalimentación al programa de tutores a distancia}

\begin{itemize}
    \item \textbf{¿Con qué tipo de dificultades metodológicas, técnicas o de contenido de las
    materias se encontró al realizar su voluntariado a distancia? ¿Cómo las
    resolvió?}

    Al haberme basado en un libro electrónico, algunas deducciones matemáticas se me hacían bastante engorrosas y poco intuitivas. Por lo cual, a veces me tocaba hacer mi propia versión de las deducciones para tener un mejor entendimiento y reflejarlo en los materiales que hice. 
    \item \textbf{¿Cómo considera que ha sido el proceso de envío y revisión de sus materiales?}
    Creo que se podría mejorar. Hace falta automatización de los procesos. Quizás un Google forms enlazado a un Google sheets, ayudaría un montón. Mandar las cosas por correo es muy impráctico. 
    \item \textbf{¿Qué percepción tiene del apoyo y seguimiento de la Unidad de Formación
    hacia el programa de TAD?}
    Todo muy bien. Buena experiencia. 
    \item \textbf{¿Qué propuestas de cambio tiene para mejorar la función del tutor a
    distancia?}
    Ninguna. 
    \item \textbf{¿Qué comentarios puede realizar sobre el voluntariado a distancia del IGER
    para que pueda mejorar?}
    Definitivamente la automatización de la recolección y retroalimentación de los materiales, ayudaría a agilizar las revisiones y las aprobaciones.
\end{itemize}




%----------------------------------------
\newpage

\end{document}