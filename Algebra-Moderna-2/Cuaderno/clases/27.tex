Clase: 18/11/2022


\begin{ejemplo}
    \begin{enumerate}
        \item Si $F$ es un campo, sea $p(x)=x^2+\alpha x+\beta \in F[x]$. Sean $K$ una extensión de $F$ y $a_1\in K\ni p(a_1)=0\implies a_2=-\alpha -a_1\in K$ es también raíz de $p(x)$. En efecto,
        \begin{align*}
            p(a_2) &= a_2^2+\alpha a_2+\beta \\
            &= (-\alpha -a_1)^2+\alpha(-\alpha-a_1)+\beta\\
            &= \alpha^2+2\alpha a_1 + a_1^2-\alpha^2 -\alpha a_1+\beta\\
            &= a_1^2-\alpha a_1+\beta =0
        \end{align*}
        Entonces $K$ contiene el campo de descomposición de $p(x)$. Si $a_2=a\implies p(x)=(x-a_1)(x-a_1)=(x-a_1)^2=x^2-2a^1x+a_1^2\in F[x]\implies -2a_1\in F\implies F$ es el campo de descomposición de $p(x)$. Nótese que estos hechos confirman el teorema 5H.
        \item $x^3-2\in \mathbb{Q}[x]$. Se sabe que si $w=-1/2 -\sqrt{3}/2 i\implies $ las raíces de $x^3-2$ son $\sqrt[3]{2}, \sqrt[3]{2}w,\sqrt[3]{2}w^2\in \mathbb{C}-\mathbb{Q}\implies x^3-2$ es irreducible sobre $\mathbb{Q}\implies$ por el teorema 5C, $[\mathbb{Q}(\sqrt[3]{2}):\mathbb{Q}]=3$. Si $E$ es el campo de descomposición de $x^3-2$ sobre $\mathbb{Q}\implies$ por el teorema 5H, $6=3!=gr(x^3-2)\geq [E:\mathbb{Q}]=[E:\mathbb{Q}(\sqrt[3]{2})][\mathbb{Q}(\sqrt[3]{2}):\mathbb{Q}]=[E:\mathbb{Q}(\sqrt[3]{2})]\cdot 3\implies [E:\mathbb{Q}(\sqrt[3]{2})]\leq 6/3=2\implies [E:\mathbb{Q}(\sqrt[3]{2})]=1$ o $[E:\mathbb{Q}(\sqrt[3]{2})]=2$. Ahora bien, $\mathbb{Q}(\sqrt[3]{2})=\langle \{1,\sqrt[3]{2},(\sqrt[3]{2})^2\}\rangle_\mathbb{Q}\implies \sqrt[3]{2}\not\in \mathbb{Q}(\sqrt[3]{2})\implies E\neq \mathbb{Q}(\sqrt[3]{2})\implies [E:\mathbb{Q}(\sqrt[3]{2})]\neq 1\implies [E:\mathbb{Q}(\sqrt[3]{2})]=2\implies [E:\mathbb{Q}]=6$. Nótese que, en este ejemplo, se alcanza de cota superior del teorema 5H. Además $\mathbb{Q}(\sqrt[3]{2})\subseteq E$ y $\sqrt[3]{2},\sqrt[3]{2}w, \sqrt[3]{2}w^2\in E.$
        \begin{cajita}
            \begin{align*}
                \sqrt[3]{2} &=1\cdot \sqrt[3]{2}\\
                -\frac{\sqrt[3]{2}}{2}+\frac{\sqrt[3]{2}\sqrt{3}}{2}i &= -\frac{1}{2}(\sqrt[3]{2})+\frac{1}{2}(\sqrt[3]{2}\sqrt{3}i)\\
                -\frac{3\sqrt{2}}{2}-\frac{\sqrt[3]{2}\sqrt{3}}{2}i &= -\frac{1}{2}(\sqrt[3]{2})-\frac{1}{2}(\sqrt[3]{2}\sqrt{3}i)
            \end{align*}
            $$\{1,\sqrt[3]{2},(\sqrt[3]{2})^2,\sqrt{3}i, \sqrt[3]{2}\sqrt{3}i, (\sqrt[3]{2})^2,(\sqrt[3]{2})^2\sqrt{3}i\}$$
        \end{cajita}
        Además, nótese que: 
        \begin{align*}
            \sqrt[3]{2} &=0\cdot 1 +1(\sqrt[3]{2})+0\cdot (\sqrt[3]{2})^2+0\cdot(\sqrt{3}i)+0\cdot(\sqrt{3}i\sqrt[3]{2})+0\cdot (\sqrt{3}i(\sqrt[3]{2})^2)\\
            \sqrt[3]{2}w &= -\frac{1}{2}+\frac{\sqrt[3]{2}\sqrt{3}}{2} i = -\frac{1}{2}(1)+0(\sqrt[3]{2})+0(\sqrt[3]{2})^2+0(\sqrt[3]{2})^2+0(\sqrt{3}i)+\\
            &+\frac{1}{2}(\sqrt{3}i\sqrt[3]{2})+0((\sqrt[3]{2})^2\sqrt{3}i)\\
            \sqrt[3]{2}w^2&=-\frac{1}{2}-\frac{\sqrt[3]{2}\sqrt{3}}{2}i = -\frac{1}{2}(1)+0(\sqrt[3]{2})+0(\sqrt[3]{2})^2+0(\sqrt{3}i)-\frac{1}{2}(\sqrt[2]{3}\sqrt{3}i)+\\
            +0((\sqrt[3]{2})^2\sqrt{3}i)
        \end{align*}
        Además, $\{1,\sqrt[3]{2},(\sqrt[3]{2})^2,\sqrt{3}i, \sqrt[3]{2}\sqrt{3}i, (\sqrt[3]{2})^2,(\sqrt[3]{2})^2\sqrt{3}i\}$ es linealmente independiente sobre $\mathbb{Q}\implies \mathbb{Q}(\sqrt[3]{2},\sqrt{3}i)=\langle \{1,\sqrt[3]{2},(\sqrt[3]{2})^2,\sqrt{3}i, \sqrt[3]{2}\sqrt{3}i, (\sqrt[3]{2})^2,(\sqrt[3]{2})^2\sqrt{3}i\}\rangle_\mathbb{Q}\implies$ por el teorema 5J, $\mathbb{Q}(\sqrt[3]{2},\sqrt{3}i)$ es el campo de descomposición de $x^3-2$ sobre $\mathbb{Q}$, salvo isomorfismo. 
        \item Sea $f(x)=x^4+x^2+1\in \mathbb{Q}[x]$. Si $a\in \mathbb{Q}\implies a^2\geq 0$ y $a^4\geq 0\implies a^4+a^2\geq 0\implies a^4+a^2+1>0\implies f(x)$ es irreducible sobre $\mathbb{Q}$. Si $E$ es el campo de descomposición de $f(x)$ sobre $\mathbb{Q}\implies$ por el teorema 3H, $[E:\mathbb{Q}]\leq 4!=24$. Tenemos $x^4+x^2+1\in \mathbb{Q}[x]\implies x^4+x^2+1 = x^4+2x^2+1-x^2=(x^2+1)^2-x^2 = (x^2+x+1)(x^2-x+1)\implies$ las raíces de $x^4+x^2+1$ son $\pm \frac{1}{2}\pm\frac{\sqrt{3}}{2}i$. Si $w=-\frac{1}{2}+\frac{\sqrt{3}}{2}i$, entonces $\mathbb{Q}(w)$ es el campo de descomposición de $x^4+x^2+1$ sobre $\mathbb{Q}$ y $[\mathbb{Q}(w):\mathbb{Q}]=2$. Nótese que $\mathbb{Q}(\sqrt{3}i)$ también es campo de descomposición de $x^4+x^2+1$ sobre $\mathbb{Q}$, y por el teorema 5J, $\mathbb{Q}(w)\approx \mathbb{Q}(\sqrt{3}i)$
        \item Un número $\alpha\in \mathbb{R}$ es constructible si dado un segmento rectilíneo de longitud unitaria, es posible únicamente empleando la regla y el compás, construir un segmento rectilíneo de longitud $\alpha$.\bigbreak 
        Sea $\mathbb{B}=\{\alpha\in\mathbb{R}\quad \text{es constructible}\}$. Claramente $\mathbb{Z}\subseteq \mathbb{B}$.
        \end{enumerate}
\end{ejemplo}

