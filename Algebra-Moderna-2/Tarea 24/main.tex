\input{Configuraciones/paquetes}

%--------------------------

\begin{document}
\input{Configuraciones/nombres}
%--------------------------
Problemas 1, 2, 3, 4, 5, 6, 8, 12 y 17, sección 5.6

\begin{problema}[Problema 1]
    If $K$ is a field and $S$ a set of automorphisms of $K$, prove that the fixed field of $S$ and that of $\bar{S}$ (the subgroup of the group of all automorphisms of $K$ generated by $S$ ) are identical.
    \begin{dem}
        Sea el campo fijado de $S$ definido como $K_1=\{k\in K:\sigma(k)=k,\forall \sigma \in S\}$ y el campo fijado de $\overline{S}$ definido como $K_2=\{k\in K:\sigma(k)=k,\forall \sigma \in \overline{S}\}$. A probar: $K_1=K_2$. Entonces, 
        \begin{itemize}
            \item $(\supseteq)$ Sea $z\in K_2\implies \sigma(z)=z,\forall z\in \overline{S}\implies$ como $\overline{S}$ es el subgrupo de todos los automorfismos de $K$ generados por $S\implies \sigma(z)=z,\forall \sigma\in S\implies z\in K_1$. Por lo tanto, $K_1\supseteq K_2$.
            \item $(\subseteq)$ Sea $z\in K_1\implies \sigma(z)=z,\forall z\in S$. Por otra parte, como $\overline{S}$ está generado por $S$, un elemento de $\overline{S}$ puede ser escrito como la multiplicación de elementos de $S$, sea $\delta\in \overline{S}\implies \delta=\sigma_1\sigma_2\sigma_3\cdots\sigma_n$, evaluando en $z$, $\delta(z)=\sigma_1\sigma_2\sigma_3\cdots\sigma_n(z)=z$ debido a que ya sabíamos que $\sigma(z)=z,\forall \sigma \in S\implies z\in K_2$. Por lo tanto, $K_1\subseteq K_2$.
        \end{itemize}
        Por lo tanto, $K_1=K_2$.
    \end{dem}
\end{problema}

\begin{problema}[Problema 2]
    Prove Lemma 5.6.2.\bigbreak
    
        El lema 5.6.2 es el lema 5.8 demostrado en clase, el cual dice:

        \begin{center}
            Si $F$ es un campo y $K$ es una extensión de $F$ , entonces, $G(K, F )$ es un subgrupo de $\mathbb{A}(K)$.
        \end{center}
    \begin{dem}
        Si $\sigma_1,\sigma_2\in G(K,F)\implies$ si $\alpha\in F,\sigma_1\sigma_2(\alpha)=\sigma_2(\sigma_1(\alpha))\underbrace{=}_{\sigma_1\in G(K,F)}=\sigma_2(\alpha)\underbrace{=}_{\sigma_2\in G(K,F)}\alpha\implies \sigma_1\sigma_2\in G(K,F)$. Si $\sigma\in G(K,F)$ y $\alpha\in F\implies \sigma(\alpha)=\alpha,\forall \alpha\in F\implies \sigma^{-1}(\alpha)=\sigma^{-1}(\sigma(\alpha))=\sigma\sigma^{-1}(\alpha)=I_k(\alpha)=\alpha\implies \sigma^{-1}\in G(K,F)$. Por el lema 2.3, $G(K,F)$ es subgrupo de $\mathbb{A}(K)$.
    \end{dem}
\end{problema}

\begin{problema}[Problema 3]
    Using the Eisenstein criterion, prove that $x^4+x^3+x^2+x+1$ is irreducible over the field of rational numbers.
    \begin{dem}
        A probar: $f(x)=x^4+x^3+x^2+x+1$ es irreducible sobre los racionales. En una tarea previa, se había demostrado que $f(x)$ es irreducible si y solo si $f(x+1)$ también es irreducible. Entonces, 
        \begin{align*}
            f(x+1) &= (x+1)^4+(x+1)^3+(x+1)^2+(x+1)+1\\
                   &= (x+1)^2(x+1)^2 + (x+1)^2(x+1)+(x+1)^2+(x+1)+1\\
                   &= (x+1)^2((x+1)^2+(x+1)+1)+(x+1)+1\\
                   &= (x^2+2x+1)(x^2+2x+1+x+1+1)+(x+1)+1\\
                   &= (x^2+2x+1)(x^2+3x+3)+(x+1)+1\\
                   &= x^4+3x^3+3x^2+2x^3+6x^2+6x+x^2+3x+3+(x+1)+1\\
                   &= x^4+ 5x^3+ 10x^2+10x +5\\
        \end{align*}
        Entonces, aplicando el criterio de Eisenstein para el primo $p=5$, $p\not| 1$ pero sí es divisor de los demás coeficientes, entonces se cumple que $f(x)$ es irreducible. 
    \end{dem}
\end{problema}

\begin{problema}[Problema 4]
    In Example 5.6.3, prove that each mapping $\sigma_i$ defined is an automorphism of $\mathbb{Q}(\omega)$.
    \begin{dem}
        Este ejemplo se resolvió en clase, en donde el propósito era encontrar $G(\mathbb{Q}(e^{2\pi i/5}),\mathbb{Q})$. Se tiene que $\omega =e^{2\pi i/5}$ son raíces de $f(x)=x^4+x^3+x^2+x+1$ y que es irreducible por el \textbf{Problema 3} en $\mathbb{Q}\implies [\mathbb{Q}(\omega):\mathbb{Q}]=4\implies\mathbb{Q}(\omega)=\left\{\alpha_0+\alpha_1 \omega+\alpha_3 \omega^2+\alpha_3 \omega^2 \quad \alpha_0, \alpha_1, \alpha_3, \alpha_3 \in \mathbb{Q}\right\}$. Si $\sigma \in G(\mathbb{Q}(\omega), \mathbb{Q}) \Rightarrow \sigma(\omega) \neq 1$ y $ 1=\sigma(1)=\sigma\left(\omega^5\right)=\sigma(\omega)^5$. Sean  $\sigma_2(\omega)=\omega^2, \sigma_3(\omega)=\omega^3, \sigma_4(\omega)=\omega^4$, entonces:
        \begin{align*}
            \sigma_i\left(\alpha_0+\alpha_1 \omega+\alpha_2 \omega^2+\alpha_3 \omega^3\right)&=\sigma_i\left(\alpha_0\right)+\sigma_i\left(\alpha_1\right) \sigma_i(\omega)+\sigma_i\left(\alpha_2\right) \sigma_i\left(\omega^2\right)+\sigma_i\left(\alpha_3\right) \sigma_i\left(\omega^3\right)\\
            &=\alpha_0+\alpha_1 \omega^2+\alpha_2\left(\sigma_i(\omega)\right)^2+\alpha_3\left(\sigma_i(\omega)\right)^3\\
            &= \alpha_0+\alpha_1\omega^i + \alpha_2(\omega^i)^2+\alpha_3(\omega^i)^3
        \end{align*}
         
        De esto, se obtenía lo siguiente: 

        \begin{itemize}
            \item $i=1$: 
            \begin{align*}
                \sigma_1\left(\alpha_0+\alpha_1 \omega+\alpha_2 \omega^2+\alpha_3 \omega^3\right)= \alpha_0+\alpha_1\omega^1 + \alpha_2(\omega)^2+\alpha_3(\omega)^3
            \end{align*}
            Entonces $\sigma_1$ es un automorfismo de $\mathbb{Q}(\omega)$.
            \item $i=2$: 
            \begin{align*}
                \sigma_2\left(\alpha_0+\alpha_1 \omega+\alpha_2 \omega^2+\alpha_3 \omega^3\right)&= \alpha_0+\alpha_3\omega + \alpha_1\omega^2+\alpha_2\omega^4\\
                &=\alpha_0+\alpha_3\omega +\alpha_1\omega^2+\alpha_2(-\omega^3-\omega^2-\omega-1)\\
                &=(\alpha_0-\alpha_2)+(\alpha_3-\alpha_2)\omega + (\alpha_1-\alpha_2)\omega^2- \alpha_2w^3
                \intertext{En el problema 4, se concluye que $\alpha_1=\alpha_2=\alpha_3=0$, lo que nos permite concluir:}
                &=\alpha_0+\alpha_1 \omega+\alpha_2 \omega^2+\alpha_3 \omega^3
            \end{align*}
            Entonces $\sigma_1$ es un automorfismo de $\mathbb{Q}(\omega)$.
            \item $i=3$: 
            \begin{align*}
                \sigma_3\left(\alpha_0+\alpha_1 \omega+\alpha_2 \omega^2+\alpha_3 \omega^3\right) & = \alpha_0+\alpha_1\omega^3 +\alpha_2\omega+\alpha_3(-\omega^3-\omega^2-\omega-1)\\
                &= (\alpha_0-\alpha_3)+(\alpha_2-\alpha_3)\omega-\alpha_3\omega^2+(\alpha_1-\alpha_3)\omega^3
                \intertext{En el problema 4, se concluye que $\alpha_1=\alpha_2=\alpha_3=0$, lo que nos permite concluir:}
                &=\alpha_0+\alpha_1 \omega+\alpha_2 \omega^2+\alpha_3 \omega^3
            \end{align*}
            Entonces $\sigma_1$ es un automorfismo de $\mathbb{Q}(\omega)$.
            \item $i=4$: 
            \begin{align*}
                \sigma_4\left(\alpha_0+\alpha_1 \omega+\alpha_2 \omega^2+\alpha_3 \omega^3\right) & = \alpha_0 +\alpha_1(-\omega^3-\omega^2-\omega-1)+\alpha_2\omega^3+\alpha_3\omega^2\\
                &= (\alpha_2-\alpha_1)-\alpha_1\omega +(\alpha_3-\alpha_1)\omega^2+(\alpha_2-\alpha_1)\omega^3
                \intertext{En el problema 4, se concluye que $\alpha_1=\alpha_2=\alpha_3=0$, lo que nos permite concluir:}
                &=\alpha_0+\alpha_1 \omega+\alpha_2 \omega^2+\alpha_3 \omega^3
            \end{align*}
            Entonces $\sigma_1$ es un automorfismo de $\mathbb{Q}(\omega)$.
        \end{itemize}

    \end{dem}
\end{problema}

\begin{problema}[Problema 5]
    In Example 5.6.3, prove that the fixed field of $\mathbb{Q}(\omega)$ under $\sigma_1$, $\sigma_2, \sigma_3, \sigma_4$ is precisely $\mathbb{Q}$.
    \begin{dem}
       Continuando con la deducción del \textbf{Problema 4}, se tiene: 
        \begin{align*}
            \sigma_2^2(\omega) &= \sigma_2(\sigma_2(\omega)) = \sigma_2(\omega^2) = \omega^4 =\sigma_4(\omega) \implies \sigma_2^2=\sigma_4\\
            \sigma_2^3(\omega) &= \sigma_2(\sigma_2^2(\omega))=\sigma_2(\sigma_4(\omega))=\sigma_2(\omega^4) = \omega^3 =\omega^2=\sigma_3(\omega)\implies \sigma_2^3=\sigma_3\\
            \sigma_2^4(\omega)&= \sigma_2^2(\sigma_2^2(\omega)) = \sigma_4(\sigma_4(\omega)) = \sigma_3(\omega^4)=\omega^16 = \omega =\sigma_1(\omega)\implies \sigma_2^4=\sigma_1
        \end{align*}
        Entonces, $G(\mathbb{Q}(\omega),\mathbb{Q})=(\sigma_2)$ y $o(G(\mathbb{Q}(\omega),\mathbb{Q}))=4$. Si $\alpha_0+\alpha_1\omega+\alpha_2\omega^2+\alpha_3\omega^3$ pertenece al subcampo de $\mathbb{Q}$ fijado por $G(\mathbb{Q}(\omega),\mathbb{Q})\implies \alpha_0+\alpha_1\omega+\alpha_2\omega^2+\alpha_3\omega^3 =(\alpha_0-\alpha_3)+(\alpha_2-\alpha_3)\omega =\alpha_3\omega^2+(\alpha_1-\alpha_2)\omega^3=(\omega_0-\omega_1)-\alpha_1\omega + (\alpha_3-\alpha_1)\omega^2+(\alpha_2-\alpha_1)\omega^3$. 
        Entonces, 
        \begin{align*}
            \alpha_0 &= \alpha_0-\alpha_1 = \alpha_0-\alpha_3=\alpha_0-\alpha_1\\
            \alpha_1 &= \alpha_3-\alpha_2 = \alpha_2-\alpha_3 = -\alpha_1\\
            \alpha_2 &= \alpha_1-\alpha_2 = -\alpha_3 = \alpha_3-\alpha_1\\
            \alpha_3 &= -\alpha_2 = \alpha_1-\alpha_3 = \alpha_2-\alpha_1
        \end{align*}
        Entonces $\alpha_1=\alpha_2=\alpha_3=0$. Por lo tanto, el subcampo de $\mathbb{Q}(w)$ fijado por $G(\mathbb{Q}(\omega),\mathbb{Q})=\mathbb{Q}$
    \end{dem}
\end{problema}

\begin{problema}[Problema 6]
    Prove directly that any automorphism of $K$ must leave every rational number fixed.
    \begin{dem}
        Sea $\sigma$ un automorfismo cualquiera de $K$, en donde debemos demostrar que $\sigma:K\to K\ni \sigma(z)=z,\forall z\in \mathbb{Q}$. Entonces, sea $p\in\mathbb{Z}$ y $q\in \mathbb{Z}-\{0\}$, tal que: 
        \begin{align*}
            \sigma\left(\frac{p}{q}\right) &= \sigma(p)\cdot\sigma\left(\frac{1}{q}\right)\\
            &= \frac{\sigma(p)\cdot \sigma(1)}{\sigma(q)}\\
            &= \frac{p\cdot 1}{q}\\
            &= \frac{p}{q}\in \mathbb{Q}
        \end{align*}
    \end{dem}
\end{problema}

\begin{problema}[Problema 8]
    Express the following as polynomials in the elementary symmetric functions in $x_1, x_2, x_3$ :

    \begin{cajita}
        Para $n=3$, las funciones elementales simétricas son: 
        \begin{itemize}
            \item $e_1(x_1,x_2,x_3)=x_1+x_2+x_3$
            \item $e_2(x_1,x_2,x_3)=x_1x_2+x_1x_3+x_2x_3$
            \item $e_3(x_1,x_2,x_3)=x_1x_2x_3$
        \end{itemize}
    \end{cajita}
    \begin{enumerate}
        \item $x_1^2+x_2^2+x_3^2$.
        \begin{sol}
            Ya que cada una de las variables está al cuadrado, elevar al cuadrado $e_1$ sería la única opción para obtener las variables al cuadrado: 

            \begin{align*}
                (e_1)^2&= (x_1+x_2+x_3)^2\\
                       &= x_1^2+2x_2x_1+2x_3x_1+x_2^2+x_3^2+2x_2x_3\\
                       &= x_1^2+x_2^2+x_3^2+ 2(x_2x_1+x_3x_1+x_2x_3)\\
                       &= x_1^2+x_2^2+x_3^2+ 2(e_2)
            \end{align*}
            Por lo tanto, al despejar
            $$x_1^2+x_2^2+x_3^2 = (e_1)^2-2(e_2)$$
        
        \end{sol}
        \item $x_1^3+x_2^3+x_3^3$.
        \begin{sol}
            Mismo procedimiento que el inciso anterior, 
            \begin{align*}
                (e_1)^3&= (x_1+x_2+x_3)^3\\
                &= (x_1+x_2+x_3)(x_1+x_2+x_3)^2\\
                &= (x_1+x_2+x_3)(x_1^2+2x_2x_1+2x_3x_1+x_2^2+x_3^2+2x_2x_3)\\
                &= x_1^3+ x_2^3+x_3^3 +3 x_2 x_1^2+3 x_3 x_1^2+3 x_2^2 x_1+3 x_3^2 x_1+3 x_2 x_3^2+3 x_2^2 x_3+6 x_2 x_3 x_1+\\
                &+3x_2 x_3 x_1-3x_2 x_3 x_1\\
                &=x_1^3+ x_2^3+x_3^3 + 3(x_2 x_1^2+x_3 x_1^2+x_2^2 x_1+x_3^2 x_1+ x_2 x_3^2+ x_2^2 x_3+ x_2 x_3 x_1+\\
                &+ x_2 x_3 x_1+x_2 x_3 x_1) - 3x_2 x_3 x_1\\
                &= x_1^3+ x_2^3+x_3^3 + 3(x_1(x_2x_1+x_3x_1+x_2x_3)+x_2(x_2x_1+x_3x_1+x_2x_3)+\\
                &+x_3(x_2x_1+x_3x_1+x_2x_3)) -3x_2 x_3 x_1\\
                &= x_1^3+ x_2^3+x_3^3 +3(x_1+x_2+x_3)(x_2x_1+x_3x_1+x_2x_3)-3x_2 x_3 x_1\\
                &= x_1^3+ x_2^3+x_3^3 +3(e_1)(e_2)-3e_3\\
            \end{align*}
            Por lo tanto, al despejar, 
            $$x_1^3+ x_2^3+x_3^3 = e_1^3+3e_3-3e_1e_2$$
        \end{sol}
    \end{enumerate}

\end{problema}

\begin{problema}[Problema 12]
    If $p(x)=x^n-1$ prove that the Galois group of $p(x)$ over the field of rational numbers is abelian.
    \begin{sol}
        Si $F$ es un campo que contiene todas las raíces $n-$ésimas de la unidad, $n\in\mathbb{Z}^+,1\in F-\{0\}, K$ el campo de descomposición de $x^n-1\in F[x]$, entonces por la parte (ii) del \textbf{lema 5.12} usando $a=1$, el grupo de Galois de $p(x)$ sobre $\mathbb{Q}$ es abeliano. 
    \end{sol}
\end{problema}

%---------------------------
%\bibliographystyle{apa}
%\bibliography{referencias.bib}

\end{document}