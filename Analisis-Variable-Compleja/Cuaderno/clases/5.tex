Clase: 19/07/2022

\begin{cajita}
    \begin{definicion}[Región]
        Conexo (por trayectorias) y abierto. 
    \end{definicion}
    \begin{definicion}[Conexo]
        No disconexo. 
    \end{definicion}
    \begin{definicion}[Disconexo]
        $A$ es disconexo si existen abierto $G$ y $H$ en la topología $\mathcal{X}\ni G\cap A\neq0, H\cap A\neq\varnothing,G\cap H = \varnothing, (G\cap A)\cup (H\cap A)=A$. 
    \end{definicion}
    \begin{definicion}[Conexo por trayectorias]
        $A$ es conexo por trayectorias si $\exists$ una curva $\gamma\ni \gamma: [a,b]\to A\ni \gamma(a)=x,\gamma(b)=y,\gamma([a,b])\subset A, \gamma$ es diferenciable. 
    \end{definicion}
\end{cajita}

\begin{prop}
    Sea $A\subset \mathbb{C}$ una región, y sea $f:A\to\mathbb{C}$ una función analítica. Si $f'(z)=0$ sobre $A$, entonces $f(z)$ es constante sobre $A$. 
    \begin{dem}
        Sean $z_1,z_2\in A\implies f(z_1)=f(z_2)$. Como $A$ es conexo por trayectorias $\implies$ existe una curva $\gamma(t)$ que une a $z_1$ con $z_2$. Entonces, 
        $$\frac{d}{dz}f\left(\gamma(t)\right)= f'(\gamma(t))\cdot \gamma'(t)=0$$
        Si $f=u+iv\implies \frac{d}{dt}u(\gamma(t))=0$ y $\frac{d}{dt}v(\gamma(t))=0$. $\implies u(\gamma(t))=\text{const}_1$ y $c(\gamma(t))=\text{const}_2$. Entonces, $f$ es una función constante.
    \end{dem}
\end{prop}

\subsection{Ecuaciones de Cauchy-Riemann}

\begin{nota}
    \begin{enumerate}
        \item Suponga que $f(z)$ es analítica en $z_0\in \mathbb{C}$. Entonces la medida de error de aproximación de $f'(z_0)$ se define: 
        $$\frac{f(z)-f(z_0)}{z-z_0}-f'(z_0)$$
        \item Como $f$ es analítica en $z_0$, entonces $z\to z_0\implies \varepsilon(z)\to 0$.
        \item Despeje: 
            $$f(z)=f(z_0)+f'(z_0)\cdot (z-z_0)+\varepsilon(z)\cdot (z-z_0),$$
            i.e., cerca de $z_0$, $f(z)$ puede aproximarse por una función lineal.
    \end{enumerate}
    
\end{nota}

\begin{teorema}[Ecuaciones de Cauchy-Riemann (CR)]
    Sea $U$ un abierto de $\mathbb{R}^2$ y sea $u$ y $v$ funciones de valores reales definidas de $U$. Entonces, la función 
    $$f(z)=f(x+iy)=u(x,y)+iv(x,y)$$
    es analítica sobre $U$ ssi se safisfacen:
    
    $$\frac{\partial u }{\partial x}(x,y)=\frac{\partial v}{\partial y}(x,y)\wedge\frac{\partial u }{\partial y}(x,y)=-\frac{\partial v}{\partial x}(x,y) $$
    para cada $(x,y)\in U$. En este caso, se tiene que: $$f'(z)=\frac{\partial u}{\partial x}+i\frac{\partial v}{\partial x}= \frac{\partial v}{\partial y}+i\frac{\partial u}{\partial y}$$
\end{teorema}