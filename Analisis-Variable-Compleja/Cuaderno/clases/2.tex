Clase: 12/07/2022

\begin{prop}
    $\forall z,w\in\mathbb{C},|z\cdot w|=|z|\cdot|w|, \arg(z*w)=\arg z+\arg w   (\mod 2\pi) $
    \begin{dem}
        Sean $x=r_1[\cos\theta_1 + i\sen\theta_1], w=r_2[\cos \theta_2+i\sen \theta_2]\implies zw = r_1r_2[\cos t_1+i\sin\theta_1]*[\cos \theta_2 +i\sin\theta_2]= r_1r_2[\cos(\theta_1+\theta_2)+i\sin(\theta_1+\theta_2)]$
        $\implies |x*w|=r_1r_2=|z||w|\implies \arg(z*w)=\theta_1+\theta_2=\arg z+\arg w   (\mod 2\pi)$
    \end{dem}
\end{prop}

\begin{prop}[De Moivre]
    Sea $z=r[\cos\theta i\sin \theta]$ y $n\in\mathbb{Z}^+$. Entonces:
    $$z^n=r^n\left[\cos n\theta +i\sin n\theta \right]$$
    \begin{dem}
        Por inducción sobre $n$
        \begin{enumerate}
            \item $n=1$: $z=r\left[\cos \theta + i \sin \theta \right]$
            \item Suponemos que $z^k = r^k[\cos k \theta+i\sin k\theta]$
            \item $z^{k+1}=z\cdot z^k = r[\cos \theta +i\sen \theta]\cdot r^k[\cos k\theta +i\sen k\theta] = r^{k+1}[\cos(k+1)\theta +i\sin(k+1)\theta]$ 
        \end{enumerate}
    \end{dem}

\end{prop}

\begin{problema}
    Dado $w\in\mathbb{C}$, encuentre $z\in\mathbb{C}\ni z^n=w, n\in\mathbb{Z}^+,n>1$
    \begin{sol}
        Sean: $w=r[\cos\theta +i\sin \theta],z=\rho[\cos\phi +i\sin \phi]\implies z^n =w\iff \rho^n[\cos n\phi +i\sin n\phi]=r[\cos\theta+i\sin\theta]\implies \rho^n =r\implies \rho =r^{1/n}$ y $n\phi =\theta+2\pi k\implies \phi =\theta/n +2\pi k/n, \quad k=0,1,\cdot,n-1$. $$\implies z_k= r^{1/n}\left[\cos\left(\theta/n + 2\pi k/n\right)+i\sin \left(\theta/n +2\pi k/n\right)\right], k=0,1,\cdots, n-1$$
    \end{sol}
\end{problema}

\begin{ejemplo}
    Encuentre las raíces cúbicas de 1+$i$: 
    \begin{sol}
        A resolver: $z^3=1+i$. $\implies 1+i = \sqrt{2}\left[\cos\pi/4 +i \sin \pi/4\right]$
        $$\implies z_k= (\sqrt{2})^{1/3}\left[\cos\left(\frac{\pi/4 +2\pi k}{3}\right)+i\sin \left(\frac{\pi/4 +2\pi k}{3}\right)\right], k=0,1,2$$
    \end{sol}
\end{ejemplo}

\begin{prop}
    Si $z,w\in\mathbb{C}$, entonces: 
    \begin{enumerate}
        \item $\overline{z+w}=\bar{z}+\bar{w}$
        \item $\overline{z\cdot w}=\bar{z}\cdot\bar{w}$
        \item $\overline{(z/w)}=\bar{z}/\bar{w},w\neq 0$
        \item $z\cdot\bar{z}=|z|^2$. Además, si $z\neq 0\implies z^{-1}=\bar{z}/|z|^2$
        \item $z=\bar{z}\iff z\in\mathbb{R}$
        \item Re $Z= (z+\bar{z})/2$; Im $z=(z-\bar{z})/zi$
        \item $\bar{\bar{z}}=z$
    \end{enumerate}
    \begin{dem}
        \begin{enumerate}
            \item Sean $z=a+bi,w=c+di\implies \overline{z+w}=\overline{(a+c)+(b+d)}i=(a+c)-(b+d)i=(a-bi)+(c-di)=\bar{z}+\bar{w}$.
            \item Sea $\overline{(a+bi)\cdot (c+di)}=\overline{(ac-bd)+i(ad+bc)}=(ac-bd)-i(ad+bc)$ y se desarrolla el otro lado.
            \item Sea $\bar{z}=\overline{\frac{z\cdot w}{w}}=\overline{(z/w)\cdot w}=\overline{(z/w)}\cdot\bar{w} = \bar{z}/\bar{w}=\overline{(z/w)}$
            \item Sea $z=a+bi\implies z\cdot\bar{z}=(a+bi)(a-bi)=a^2+b^2=|z|^2$. Entonces 
            $z\cdot z^{-1}=1\implies \bar{z}(z\cdot z^{-1})=\bar{z}\implies (\bar{z}z)z^{-1}=\bar{z}\implies z^{-1}=\bar{z}/|z|^2$

        \end{enumerate}
    \end{dem}
\end{prop}

\begin{prop}
    Si $z,w\in\mathbb{C}$, entonces: 
    \begin{enumerate}
        \item $|z\cdot w|=|z|\cdot |w|$
        \item Si $w\neq 0\implies |z/w|=|z|/|w|$
        \item $|\operatorname{Re}z|\leq |z|;|\operatorname{Im}z|\leq |z|$
        \item $|z|=|\bar{z}|$
        \item $|z+w|\leq |z|+|w|$
        \item $|z-w|\geq ||z|-|w||$
    \end{enumerate}
    \begin{dem}
        \begin{enumerate}
            \item OK
            \item $|z|=|(z/w)\cdot w|=|(z/w)||w|\implies |z|/|w|=|z/w|$
            \item Sea $z=a+bi\implies b^2\geq 0\implies a^2+b^2\geq a^2\implies \sqrt{a^2+b^2}\geq \sqrt{a^2}\implies |z|\geq |\operatorname{Re} z|$
            \item Si $z=a+bi\implies |\bar{z}|=|a-bi|=|a+(-b)i|=\sqrt{a^2+(-b)^2}=|z|$
            \item $|z+w|^2=(z+w)\overline{(z+w)}=(z+w)(\bar{z}+\bar{w})=z\bar{z}+z\bar{w}+w\bar{z}+w\bar{w}$
            \item $|z|=|(z-w)+w|\leq |z-w|+|w|\implies |z|-|w|\leq |z-w|$ y $|w|=|(w-z)+z|\leq |w-z|+|z|=|z-w|+|z|$. Tenemos $-|z|+|w|\leq |z-w|$ y $-(|z|-|w|)\leq |z-w|$. Por lo tanto, 
            $||z|-|w||\leq |z-w|$.
        \end{enumerate}
    \end{dem}
\end{prop}

\begin{ejemplo}
    Si $w$ es $n$-ésima raíz de la unidad, $w\neq 1$, entonces 
    $$1+w+w^2+\cdots + w^{n-1}=0$$
    \begin{sol}
        $w^n=1\implies w^n-1=0\implies (w-1)(1+w+w^2+\cdots + w^{n-1})=0$. Como $w\neq 1\implies 1+w+w^2+\cdot + w^{n-1}=0$.
    \end{sol}
\end{ejemplo}

\begin{ejemplo}
    \begin{enumerate}
        \item $\arg \bar{z}=-\arg z (\mod 2\pi)$
        \item $\arg(z/w)=\arg z -\arg w, w\neq 0 (\mod 2\pi)$
    \end{enumerate}
    \begin{sol}
        \begin{enumerate}
            \item $z\cdot\bar{z}=|z|^2\implies \arg(z\cdot \bar{z})=\arg|z|^2=0\implies \arg(z)+\arg(\bar{z})=0 \quad (\mod 2\pi)$
            \item $\arg z = \arg(\frac{z}{w}\cdot w)=\arg(z/w)+\arg(w),\quad \mod 2\pi$
        \end{enumerate}
    \end{sol}
\end{ejemplo}