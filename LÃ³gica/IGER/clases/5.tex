Clase: 19/07/2022

\subsection{Enfoque axiomático de álgebra booleana}

\begin{definicion}
    Sea $B$ no vacío que contiene $Z$ elementos especiales 0 (el cero como elemento neutro) y 1 (el uno o elemento unidad), sobre el cual se define, las operaciones binarias $+,\cdot$ y una operación unaria $-$. Entonces, $(B,+,\cdot, -,0,1)$ es un álgebra booleana. 
    \begin{itemize}
        \item Leyes conmutativas: 
            \begin{itemize}
                \item $\forall x,y\in, x+y=y+v$
                \item $\forall x,y\in B, x\cdot y = y\cdot x$
            \end{itemize}
        \item Leyes distributivas 
            \begin{itemize}
                \item $\forall x,y\in B, x(y+z)=(x\cdot y)+ (x\cdot z)$
                \item $\forall x,y\in B, x+ (y\cdot z) = (x+y)(x+z)$
            \end{itemize}
        \item Leyes de identidad.
            \begin{itemize}
                \item $\forall x\in B, x+0=x$
                \item $\forall x\in B, x\cdot 1 = 0$
            \end{itemize}
        \item Leyes de inversos.
    \end{itemize}
\end{definicion}


\begin{teorema}
    Teorema 17.
    \begin{dem}
        Sea $x\in B$, 
        \begin{align*}
            x &= x+0 & \text{ ley de identidad}\\
            x &= x+(x\cdot \overline{x}) & \text{ ley de inversos}\\
            x &= (x+x)(x+\overline{x}) & \text{ ley distributiva respecto a $\cdot$}\\
            x &= (x+x)\cdot 1 & \text{ ley de inversos}\\
            x &= (x+x) & \text{ ley de identidad}\\
        \end{align*}
    \end{dem}
\end{teorema}


\begin{teorema}[Principio de dualidad]
    Si $s$ es un teorema relativo a un álgebra booleana y $s$ puede demostrarse a partir de los axiomas de la definición del álgebra booleana y de otras propiedades obtenidas a partir de estos mismos axiomas, entonces también es teorema.
\end{teorema}

\begin{teorema}
    Dado $B$ un álgebra booleana, si $x,y\in B$, entonces: 
    \begin{enumerate}
        \item \textbf{Ley de dominancia}.
        \begin{enumerate}
            \item $x\cdot 0 = 0$
            \item $x+1 =1$
        \end{enumerate}
        \item \textbf{Ley de absorción}. 
        \begin{enumerate}
            \item $x\cdot (x+y)=x$
            \item $x+(x\cdot y)=x$
        \end{enumerate}
        \item \textbf{Ley de cancelación}. 
        \begin{enumerate}
            \item $(x\cdot y = x\cdot z)$ y $(\overline{x}\cdot y) =  \overline{x}\cdot z$ entonces $y=z$.
        \end{enumerate} 
        \item \textbf{Ley de asociatividad}.
        \item  \textbf{Unicidad de inversos}.
    \end{enumerate}
\end{teorema}

\begin{definicion}[24]
    asd
\end{definicion}

\begin{teorema}
    La relación $\leq$ es una relación de orden parcial. 
    \begin{dem}
        \begin{itemize}
            \item Sea $x\in B$ como $x\cdot x=x$ por teorema 17, entonces $x\leq x$ por definición 24. 
            \item Sea $x,y\in B$ supóngase $x\leq y$ y $y\leq x$, entonces: $x\cdot y =x$ y $y\cdot x =y$ por def. 24. Luego, $x=x\cdot  y = y\cdot x = y$ por ley de conmutatividad. 
            \item Sea $x,y,z\in B$. Supóngase $x\leq y$ y $y\leq z$. Entonces, $x\cdot y=\underline{x}$ y $y\cdot z = y$. Luego, $x\cdot z = (x\cdot y)\cdot z = x\cdot (y\cdot z) = x\cdot y = x$ por asociatividad del $\cdot$ y  igualdades anteriores con lo cual $x\leq z$ por definición 24. 
        \end{itemize}
    \end{dem}
\end{teorema}