\input{Configuraciones/paquetes}

%--------------------------

\begin{document}
\input{Configuraciones/nombres}
%--------------------------

\section{¿Bajo qué principios de contabilidad se rige Guatemala?}

En Guatemala, los principios de contabilidad están basados en los principios de contabilidad generalmente aceptados (PCGA). Ahora bien, de acuerdo de IGCPA (Instituto Guatemalteco de Contadores Públicos y Auditores, 1998)\cite{conta1} se tiene: 


\begin{enumerate}
    \item Registrar en forma adecuada los activos invertidos en la empresa por los miembros, socios, accionistas y por los acreedores; registrar todos los pasivos conocidos para que, conjuntamente con el patrimonio, presenten razonablemente la situación financiera de la empresa. 
    \item Presentar la inversión de los propietarios sobre bases acumulativas. 
    \item Presentar razonablemente el resultado de las operaciones. 
    \item Presentar informe y estados financieros según el concepto de la entidad.
\end{enumerate}





\section{¿Qué es el US GAAP?}

Considerando a \cite{conta2} se tiene que el US GAAP son los principios de contabilidad generalmente aceptados (PCGA) usado por las compañías en Estados Unidos. En resumen, son varias nombres autorizadas por entidades reguladoras. Los principios son: 

\begin{enumerate}
    \item Principio de costo histórico, es la contabilidad de costos de adquirir activos y pasivos. 
    \item Principio de reconocimiento de ingresos. Registro de ingresos que se obtienen.
    \item Principio de congruencia. Gastos coinciden con ingresos. 
    \item Principio de divulgación total. Cantidad y tipo de información divulgada con base en el análisis de compensaciones. 
\end{enumerate}

\section{Principales diferencias con el US GAAP}

Esencialmente son los mismos, sin embargo quizás la diferencia más notoria es que en Guatemala no hay un principio que diga algo acerca de la divulgación.



%---------------------------

\bibliographystyle{plain}
\bibliography{referencias.bib}

\end{document}