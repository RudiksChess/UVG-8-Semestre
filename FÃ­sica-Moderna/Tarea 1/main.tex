\documentclass[a4paper,12pt]{article}
\usepackage[top = 2.5cm, bottom = 2.5cm, left = 2.5cm, right = 2.5cm]{geometry}
\usepackage[T1]{fontenc}
\usepackage[utf8]{inputenc}
\usepackage{multirow} 
\usepackage{booktabs} 
\usepackage{graphicx}
\usepackage[spanish]{babel}
\usepackage{setspace}
\setlength{\parindent}{0in}
\usepackage{float}
\usepackage{fancyhdr}
\usepackage{amsmath}
\usepackage{amssymb}
\usepackage{amsthm}
\usepackage[numbers]{natbib}
\newcommand\Mycite[1]{%
	\citeauthor{#1}~[\citeyear{#1}]}
\usepackage{graphicx}
\usepackage{subcaption}
\usepackage{booktabs}
\usepackage{etoolbox}
\usepackage{minibox}
\usepackage{hyperref}
\usepackage{xcolor}
\usepackage[skins]{tcolorbox}
%---------------------------

\newtcolorbox{cajita}[1][]{
	 #1
}

\newenvironment{sol}
{\renewcommand\qedsymbol{$\square$}\begin{proof}[\textbf{Solución.}]}
	{\end{proof}}

\newenvironment{dem}
{\renewcommand\qedsymbol{$\blacksquare$}\begin{proof}[\textbf{Demostración.}]}
	{\end{proof}}

\newtheorem{problema}{Problema}
\newtheorem{definicion}{Definición}
\newtheorem{ejemplo}{Ejemplo}
\newtheorem{teorema}{Teorema}
\newtheorem{corolario}{Corolario}[teorema]
\newtheorem{lema}[teorema]{Lema}
\newtheorem{prop}{Proposición}
\newtheorem*{nota}{\textbf{NOTA}}
\renewcommand\qedsymbol{$\blacksquare$}
\usepackage{svg}
\usepackage{tikz}
\usepackage[framemethod=default]{mdframed}
\global\mdfdefinestyle{exampledefault}{%
linecolor=lightgray,linewidth=1pt,%
leftmargin=1cm,rightmargin=1cm,
}




\newenvironment{noter}[1]{%
\mdfsetup{%
frametitle={\tikz\node[fill=white,rectangle,inner sep=0pt,outer sep=0pt]{#1};},
frametitleaboveskip=-0.5\ht\strutbox,
frametitlealignment=\raggedright
}%
\begin{mdframed}[style=exampledefault]
}{\end{mdframed}}
\newcommand{\linea}{\noindent\rule{\textwidth}{3pt}}
\newcommand{\linita}{\noindent\rule{\textwidth}{1pt}}

\AtBeginEnvironment{align}{\setcounter{equation}{0}}
\pagestyle{fancy}

\fancyhf{}









%----------------------------------------------------------
\lhead{\footnotesize Álgebra Moderna}
\rhead{\footnotesize  Rudik Roberto Rompich}
\cfoot{\footnotesize \thepage}


%--------------------------

\begin{document}
 \thispagestyle{empty} 
    \begin{tabular}{p{15.5cm}}
    \begin{tabbing}
    \textbf{Universidad del Valle de Guatemala} \\
    Departamento de Matemática\\
    Licenciatura en Matemática Aplicada\\\\
   \textbf{Estudiante:} Rudik Roberto Rompich\\
   \textbf{Correo:}  \href{mailto:rom19857@uvg.edu.gt}{rom19857@uvg.edu.gt}\\
   \textbf{Carné:} 19857
    \end{tabbing}
    \begin{center}
        MM2035 - Álgebra Moderna - Catedrático: Ricardo Barrientos\\
        \today
    \end{center}\\
    \hline
    \\
    \end{tabular} 
    \vspace*{0.3cm} 
    \begin{center} 
    {\Large \bf  Tarea 13
} 
        \vspace{2mm}
    \end{center}
    \vspace{0.4cm}
%--------------------------

\begin{problema}
    Está  desliándose  sobre  la  superficie  de  la  Tierra  a  gran  velocidad,  llevando  tu  reloj  de  alta precisión.  En  los  puntos  X  e  Y  en  el  suelo  hay  relojes  similares, sincronizados  en  el  marco  de referencia terrestre. Al pasar por encima del reloj X, éste y su reloj leen ambos 0.  
\begin{enumerate}
    \item Según usted, ¿los relojes X e Y avanzan más lento o más rápido que el suyo?   
    \begin{sol}
        Más lentos, por el principio de la dilatación del tiempo. Un observador ve al otro yendo más lento. 
    \end{sol}
    \item Cuando pasa por encima del reloj Y, ¿leen la misma hora, una hora anterior o una hora posterior a la suya? Asegúrese que su respuesta coincide con lo que deberían ver los observadores en tierra. 
    \begin{sol}
        Por principio de dilatación de tiempo, una hora posterior a la mía. 
    \end{sol}
    \item Compagine cualquier contradicción aparente entre tus respuestas a las partes (a) y (b). 
    \begin{sol}
        Primero, estoy viendo que los relojes de la tierra están avanzando más lentos que el mío; pero ellos también están viendo que mi reloj está avanzando más lento respecto a mí. Sin embargo, el reloj de la tierra tiene que tener una hora mayor a la mía, por definición del marco de referencia inercial. 
    \end{sol}
\end{enumerate}

 
\end{problema}

\begin{problema}
    Usted está flotando en el espacio cuando nota que un platillo volador está rodeándolo. Cada vez que pasa frente a usted, usted anota la lectura del reloj en el platillo volador. ¿Ve que el reloj avanza más rápido o más lento que su reloj de pulsera? ¿El extraterrestre ve que su reloj de pulsera avanza más rápido o más lento que el reloj en el platillo volador? Explique. 
    \begin{sol}
        Notaré que su reloj avanza más lento con respecto a mi reloj, mientras que el extraterrestre verá que mi reloj avanza más rápido. 
    \end{sol}
\end{problema}


\begin{problema}
    Dos objetos aislados del resto del universo chocan y se pegan. ¿La energía cinética final del sistema depende del marco de referencia en el que se mire? ¿El cambio de energía cinética del sistema depende del marco de referencia en el que se mire? Explique. 
    \begin{sol}
        Tenemos: 
        \begin{enumerate}
            \item Sí, la energía cinética final del sistema depende del marco de referencia en el que se le mire. 
            \item No, el cambio de energía cinética no depende del marco de referencia. 
        \end{enumerate}
        
    \end{sol}
\end{problema}

\begin{problema}
    A través de una ventana de la nave espacial de Carl, que pasa a 0.5$c$, usted ve a Carl haciendo un importante cálculo de física. Según tu reloj, tarda 1 minuto. ¿Cuánto tiempo empleó Carl en su cálculo?
    \begin{sol}
        Tenemos que: 
        \begin{align*}
            \Delta t &= \gamma \Delta t_0\\
                     &= \left(\frac{1}{\sqrt{1-\frac{v^2}{c^2}}}\right)\Delta t_0\\
                    60 &= \frac{1}{\sqrt{1-\frac{(0.5c)^2}{c^2}}}\Delta t_0\\
            60\sqrt{0.75} &= \Delta t_0\\
            52 &= \Delta t_0\\
        \end{align*}
    \end{sol}
\end{problema}

\begin{problema}
    Usted está en un autobús que viaja por una carretera recta a 20 m/s. Al pasar por una gasolinera, su reloj y el de la estación de servicio marcan exactamente 0. Pasas por otra gasolinera 900 m más adelante. (En el marco de referencia de las gasolineras, todos los relojes de las gasolineras están sincronizados). 
\begin{enumerate}
    \item Al pasar por la segunda gasolinera, ¿encuentra que el reloj de la gasolinera está adelantado o retrasado con respecto al suyo, y
    \begin{sol}
        Por dilatación de tiempo, debe estar adelantado. 
    \end{sol} 
    \item ¿en cuánto?
    \begin{sol}
        Tenemos que: 
        \begin{align*}
            \Delta t &= \gamma \Delta t_0\\
            \frac{\Delta x}{v}&= \left(\frac{1}{\sqrt{1-\frac{v^2}{c^2}}}\right)\Delta t_0\\
            \frac{\Delta x}{v} \left(\sqrt{1-\frac{v^2}{c^2}}\right) &= \Delta t_0\\
            \frac{900}{20} \left(\sqrt{1-\left(\frac{20}{300000000}\right)^2}\right) &=\Delta t_0\\
            \frac{90}{2} \left(\sqrt{1-\left(\frac{2}{30000000}\right)^2}\right) &=\Delta t_0\\
            45 \left(\sqrt{1-\left(\frac{1}{15000000}\right)^2}\right) &=\Delta t_0
        \end{align*}
        (Es decir, algo totalmente insignificante que podría resolverse por Newton :v)
    \end{sol} 
\end{enumerate}

\end{problema}

\begin{problema}
    Es el año 2150 y la Federación Espacial de las Naciones Unidas ha perfeccionado finalmente el almacenamiento de antiprotones para su uso como combustible en una nave espacial. (Los antiprotones son las antipartículas de los protones). Los preparativos están en marcha para una visita de una nave espacial tripulada a posibles planetas orbitando una de las tres estrellas del sistema estelar Alfa Centauri, a unos 4.30 años luz. Se han colocado provisiones a bordo para permitir un viaje de 16 años de duración total. a
    \begin{enumerate}
        \item ¿A qué velocidad debe viajar la nave para que las provisiones duren? Olvídese de el período de aceleración, el giro y los tiempos de visita, ya que son insignificantes en comparación con el tiempo de tiempo de viaje. 
        \begin{sol}
            Consideramos que el viaje es de ida y de vuelta, entonces tenemos:
            $$T= \frac{2L}{v} = \frac{T_0}{\sqrt{1-v^2/c^2}}$$
            Nótese que un 1 año luz es igual a $9.461*10^{15} m$, tal que:
            \begin{align*}
                 \frac{2\left[ (4.30 \text{ años luz}) * (9.461*10^{15} m/\text{ año luz})\right]}{v} &= \frac{16}{\sqrt{1-v^2/c^2}}\\
                 \frac{8.13646*10^{16}}{v} &= \frac{16}{\sqrt{1-v^2/c^2}}\\
                 \frac{8.13646*10^{16}}{16} &= \frac{v}{\sqrt{1-v^2/c^2}}\\
                 \left(\frac{8.13646*10^{16}}{16}\right)^2 &= \frac{v^2}{1-v^2/c^2}\\
                 \left(\frac{8.13646*10^{16}}{16}\right)^2\left[ 1- v^2/c^2\right] -v^2&= 0\\
                 \left(\frac{8.13646*10^{16}}{16}\right)^2 - \frac{\left(\frac{8.13646*10^{16}}{16}\right)^2}{c^2}v^2-v^2&= 0\\
                 \left(\frac{8.13646*10^{16}}{16}\right)^2 + v^2\left[-1-\frac{\left(\frac{8.13646*10^{16}}{16}\right)^2}{c^2}\right]&= 0\\
                 v&=\sqrt{\frac{\left(\frac{8.13646*10^{16}}{16}\right)^2 }{\left[-1-\frac{\left(\frac{8.13646*10^{16}}{16}\right)^2}{c^2}\right]}}\\
                 v&= 1.42*10^8 m/s.
            \end{align*}
        \end{sol}
        \item Consideremos la solución del inciso (a) desde el punto de vista de la contracción de la longitud. 
        \begin{sol}
            Es decir, la solución hace sentido, ya que se intentan aprovechar los 16 años de recursos que se tienen. 
        \end{sol}
    \end{enumerate}
\end{problema}

\begin{problema}
    Anna y Bob nacen justo cuando la nave espacial de Anna pasa por la Tierra a 0.9c. Según Bob en la Tierra, el planeta Z está a una distancia fija de 30 años luz. Cuando Anna pasa por el Planeta Z en su viaje hacia el exterior, ¿cuál será (a) la edad de Bob según Bob, (b) la edad de Bob según Anna, (c) la edad de Anna según Anna, y (d) la edad de Anna según Bob?
    \begin{sol}
        Tenemos:
        \begin{itemize}
            \item La edad de Bob según Bob.
                $$\frac{30}{0.9c} = 33,3$$
            \item La edad de Anna según Anna.
                $$\frac{\left(\frac{30}{\frac{1}{\sqrt{1-(0.9)^2}}}\right)}{0.9c} = 14.5 $$
    
                
            \item La edad de Bob según Anna.
                $$\frac{14.5}{\frac{1}{\sqrt{1-(0.9)^2}}}= 6.33 $$
            
            \item La edad de Anna según Bob.
            $$\left(\frac{33.3}{\frac{1}{\sqrt{1-(0.9)^2}}}\right) = 14.5 $$

        \end{itemize}

    \end{sol}
\end{problema}

\begin{problema}
    La luz de la galaxia NGC 221 consta de un espectro de longitudes de onda. Sin embargo, todas están desplazadas hacia el extremo de menor longitud de onda del espectro. En particular, la "línea" de calcio que normalmente se observa a 396.85 nm se observa a 396.58 nm. ¿Esta galaxia se está acercando o alejando de la Tierra? ¿A qué velocidad? 
    \begin{sol}
        Tenemos 
        \begin{align*}
            f_{\text{observado}} & =\sqrt{\frac{1+v/c}{1-v/c}}\\
            \frac{f_{\text{observado}}}{ f_{\text{fuente}}}&= \sqrt{\frac{1+v/c}{1-v/c}}\\
            \frac{c/\lambda_{\text{observado}}}{c/\lambda_{\text{fuente}}}&= \sqrt{\frac{1+v/c}{1-v/c}}\\
            \frac{\lambda_{\text{fuente}}}{\lambda_{\text{observado}}}&= \sqrt{\frac{1+v/c}{1-v/c}}\\
            \left(\frac{\lambda_{\text{fuente}}}{\lambda_{\text{observado}}}\right)^2&= 
            \frac{1+v/c}{1-v/c}\\
            \left(\frac{\lambda_{\text{fuente}}}{\lambda_{\text{observado}}}\right)^2 -\left(\frac{\lambda_{\text{fuente}}}{\lambda_{\text{observado}}}\right)^2\frac{v}{c}-1-\frac{v}{c} &=0\\
            -\frac{v}{c}\left[1-\left(\frac{\lambda_{\text{fuente}}}{\lambda_{\text{observado}}}\right)^2 \right] &= 1 - \left(\frac{\lambda_{\text{fuente}}}{\lambda_{\text{observado}}}\right)^2\\
            \frac{v}{c}&=\frac{1 - \left(\frac{\lambda_{\text{fuente}}}{\lambda_{\text{observado}}}\right)^2}{\left[\left(\frac{\lambda_{\text{fuente}}}{\lambda_{\text{observado}}}\right)^2 -1\right]}\\
            \frac{v}{c} &= \frac{1 - \left(\frac{396.58}{396.85}\right)^2}{\left[\left(\frac{396.58}{396.85}\right)^2 -1\right]}\\
            &= 0.000681
        \end{align*}
        Por lo tanto, ya que la frecuencia observada es mayor que la de la fuente, la galaxia se está moviendo hacia la tierra y la velocidad observada es de: 
$$v = 0.000681c$$

    \end{sol}
\end{problema}


\begin{problema}
    En un experimento del colisionador de partículas, la partícula 1 se mueve hacia la derecha a 0.99$c$ y la partícula 2 a la izquierda a 0.99$c$, ambas con respecto al laboratorio. ¿Cuál es la velocidad relativa de las dos partículas según (un observador que se mueve con) la partícula 2? 
    \begin{sol}
        Usando la fórmula para la velocidad relativa: 
        \begin{align*}
            u' &=\frac{u-v}{1-\frac{uv}{c^2}}\\
            &= \frac{-0.99c-0.99c}{1-\frac{(-0.99c)(0.99c)}{c^2}}\\
            &= -0.9995c
        \end{align*}
    \end{sol}
\end{problema}

\begin{problema}
    Un resorte tiene una constante de fuerza de 18 N/m. Si se comprime 50 cm de su longitud de 
    equilibrio, ¿cuánta masa habrá ganado?
    \begin{sol}
        Tenemos: 
        \begin{align*}
            \Delta E & = \Delta m c^2\\
            \frac{1}{2}kx^2 & = \Delta m c^2\\
            \frac{\frac{1}{2}kx^2}{c^2} & = \Delta m\\
            \frac{\frac{1}{2}(18)(0.5)^2}{(300000000)^2} & = \Delta m\\
            2.5*10^{-17}\text{ kg}&= \Delta m
        \end{align*}
    \end{sol} 
\end{problema}
 



%---------------------------
%\bibliographystyle{apa}
%\bibliography{referencias.bib}

\end{document}