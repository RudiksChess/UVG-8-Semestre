\documentclass[a4paper,12pt]{article}
\usepackage[top = 2.5cm, bottom = 2.5cm, left = 2.5cm, right = 2.5cm]{geometry}
\usepackage[T1]{fontenc}
\usepackage[utf8]{inputenc}
\usepackage{multirow} 
\usepackage{booktabs} 
\usepackage{graphicx}
\usepackage[spanish]{babel}
\usepackage{setspace}
\setlength{\parindent}{0in}
\usepackage{float}
\usepackage{fancyhdr}
\usepackage{amsmath}
\usepackage{amssymb}
\usepackage{amsthm}
\usepackage[numbers]{natbib}
\newcommand\Mycite[1]{%
	\citeauthor{#1}~[\citeyear{#1}]}
\usepackage{graphicx}
\usepackage{subcaption}
\usepackage{booktabs}
\usepackage{etoolbox}
\usepackage{minibox}
\usepackage{hyperref}
\usepackage{xcolor}
\usepackage[skins]{tcolorbox}
%---------------------------

\newtcolorbox{cajita}[1][]{
	 #1
}

\newenvironment{sol}
{\renewcommand\qedsymbol{$\square$}\begin{proof}[\textbf{Solución.}]}
	{\end{proof}}

\newenvironment{dem}
{\renewcommand\qedsymbol{$\blacksquare$}\begin{proof}[\textbf{Demostración.}]}
	{\end{proof}}

\newtheorem{problema}{Problema}
\newtheorem{definicion}{Definición}
\newtheorem{ejemplo}{Ejemplo}
\newtheorem{teorema}{Teorema}
\newtheorem{corolario}{Corolario}[teorema]
\newtheorem{lema}[teorema]{Lema}
\newtheorem{prop}{Proposición}
\newtheorem*{nota}{\textbf{NOTA}}
\renewcommand\qedsymbol{$\blacksquare$}
\usepackage{svg}
\usepackage{tikz}
\usepackage[framemethod=default]{mdframed}
\global\mdfdefinestyle{exampledefault}{%
linecolor=lightgray,linewidth=1pt,%
leftmargin=1cm,rightmargin=1cm,
}




\newenvironment{noter}[1]{%
\mdfsetup{%
frametitle={\tikz\node[fill=white,rectangle,inner sep=0pt,outer sep=0pt]{#1};},
frametitleaboveskip=-0.5\ht\strutbox,
frametitlealignment=\raggedright
}%
\begin{mdframed}[style=exampledefault]
}{\end{mdframed}}
\newcommand{\linea}{\noindent\rule{\textwidth}{3pt}}
\newcommand{\linita}{\noindent\rule{\textwidth}{1pt}}

\AtBeginEnvironment{align}{\setcounter{equation}{0}}
\pagestyle{fancy}

\fancyhf{}









%----------------------------------------------------------
\lhead{\footnotesize Álgebra Moderna}
\rhead{\footnotesize  Rudik Roberto Rompich}
\cfoot{\footnotesize \thepage}


%--------------------------

\begin{document}
 \thispagestyle{empty} 
    \begin{tabular}{p{15.5cm}}
    \begin{tabbing}
    \textbf{Universidad del Valle de Guatemala} \\
    Departamento de Matemática\\
    Licenciatura en Matemática Aplicada\\\\
   \textbf{Estudiante:} Rudik Roberto Rompich\\
   \textbf{Correo:}  \href{mailto:rom19857@uvg.edu.gt}{rom19857@uvg.edu.gt}\\
   \textbf{Carné:} 19857
    \end{tabbing}
    \begin{center}
        MM2035 - Álgebra Moderna - Catedrático: Ricardo Barrientos\\
        \today
    \end{center}\\
    \hline
    \\
    \end{tabular} 
    \vspace*{0.3cm} 
    \begin{center} 
    {\Large \bf  Tarea 13
} 
        \vspace{2mm}
    \end{center}
    \vspace{0.4cm}
%--------------------------
Problemas 2, 3 y 5, sección 3.10.


\begin{problema}[Problema 2]
    If $p$ is a prime number, prove that the polynomial $x^n-p$ is irreducible over the rationals.
    \begin{dem}
        Nótese que $x^n-p$ se puede reescribir como: 
        $$f(x)=x^n -p = a_0+a_1x+a_2x^2 +\cdots + a_nx^n,$$
        en donde $a_0=-p, a_n=1$ y los demás coeficientes en 0. $\implies$ Aplicamos el criterio de Eisenstein  y entonces suponemos que para $p$ se tiene $p\not|a_n, p\not| a_n, p|a_1,p|a_2,\cdots, p|a_0, p^2\not| a_0$. Por lo tanto, $f(x)$ es irreducible sobre los racionales.
    \end{dem}
\end{problema}

\begin{problema}[Problema 3]
    Prove that the polynomial $1+x+\cdots+x^{p-1}$, where $p$ is a prime number, is irreducible over the field of rational numbers. (Hint: Consider the polynomial $1+(x+1)+(x+1)^2+\cdots+(x+1)^{p-1}$, and use the Eisenstein criterion.)
    \begin{dem}
        Sea $f(x)=1+x+\cdots +x^{p-1}$, usando la sugerencia, nótese que la expresión anterior es equivalente a $f(x+1)=1+(x+1)+(x+1)^2+\cdots+(x+1)^{p-1}$ la cual también es irreducible sobre el campo de los números racionales. Ahora bien, nótese que 
        \begin{align*}
            f(x+1) &=1+(x+1)+(x+1)^2+\cdots+(x+1)^{p-1}\\
                 &= \frac{((x+1)-1)\left[1+(x+1)+(x+1)^2+\cdots+(x+1)^{p-1}\right]}{((x+1)-1)}\\
                 &= \frac{(x+1)+(x+1)^2+(x+1)^3+\cdots+(x+1)^{p}-1-(x+1)-(x+1)^2-\cdots -(x+1)^{p-1}}{((x+1)-1)}\\
                 &= \frac{(x+1)^{p}-1}{x}
                 \intertext{Por teorema binomial,}
                 &= \frac{\sum_{k=0}^p\begin{pmatrix}
                    p\\
                    k
                 \end{pmatrix}x^{p-k}1^k -1}{x} =\frac{\sum_{k=0}^p\begin{pmatrix}
                    p\\
                    k
                 \end{pmatrix}x^{p-k} -1}{x}\\
                 &= \frac{\begin{pmatrix}
                    p\\
                    0
                 \end{pmatrix}x^{p}+\begin{pmatrix}
                    p\\
                    1
                 \end{pmatrix}x^{p-1}+\begin{pmatrix}
                    p\\
                    2
                 \end{pmatrix}x^{p-2}+\cdots+ \begin{pmatrix}
                    p\\
                    p-1
                 \end{pmatrix}x^{p-(p-1)}+\begin{pmatrix}
                    p\\
                    p
                 \end{pmatrix}x^{p-p}   -1}{x}\\
                 &= \frac{x^{p}+px^{p-1}+\begin{pmatrix}
                    p\\
                    2
                 \end{pmatrix}x^{p-2}+\cdots +px+1 -1}{x}\\
                 &= x^{p-1}+px^{p-2}+\begin{pmatrix}
                    p\\
                    2
                 \end{pmatrix}x^{p-3}+\cdots +p
        \end{align*}
        $\implies$ Por el criterio de Eisenstein, $f(x+1)$ es irreducible sobre los racionales. Por lo tanto, como $f(x+1)$ era equivalente a $f(x)$, $f(x)$ también es irreducible sobre los irracionales. 
    \end{dem}
\end{problema}

\begin{problema}[Problema 5]
    If $a$ is rational and $x-a$ divides an integer monic polynomial, prove that $a$ must be an integer.

   
    \begin{dem}
        Sea $a=m/n\in \mathbb{Q}$ y sea $f(x)= \left(a_0+a_1x+\cdots + (1)\cdot x^n\right)$ tenemos que 
        $$\left(x-\frac{m}{n}\right)\Big|f(x),$$

        de esto, tenemos que $\exists g(x)\in \mathbb{Q}[x] \ni f(x)=\left(x-\frac{m}{n}\right)g(x)$. Sea entonces, 
        \begin{align*}
            f(x)&=\left(x-\frac{m}{n}\right)g(x)\\
            \left(a_0+a_1x+\cdots + x^n\right) &= \left(x-\frac{m}{n}\right)\frac{p}{q}\cdot g'(x), \quad (p,q)=1 \\
            &= \left(nx-m\right)\frac{p}{qn}\cdot g'(x)
            \intertext{Por \textbf{proposición} $f(x)$ es primitivo en los enteros, por ser mónico. Entonces $p/qn=1$. }
             &= \left(nx-m\right)\left(b_0+b_1x+\cdots +b_{n-1}x^{n-1} \right)\\
            &= mb_0+\left(nb_0-mb_1\right)x+\cdots +nb_{n-1}x^{n}
        \end{align*}
        $\implies$ De esto, tenemos que
        $$a_0=mb_0, a_1= nb_0-mb_1,\cdots, 1= nb_{n-1}$$
        Pero nótese que entonces, que para que se cumpla la igual de arriba, $n=1,-1$, por lo que $m/n$ debe ser un entero. Por lo tanto, $a$ es un entero. 
    \end{dem}
\end{problema}

%---------------------------
%\bibliographystyle{apa}
%\bibliography{referencias.bib}

\end{document}