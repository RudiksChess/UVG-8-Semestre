\input{Configuraciones/paquetes}

%--------------------------

\begin{document}
\input{Configuraciones/nombres}
%--------------------------
Problemas 4, 5, 8 y 11, sección 3.11.


\begin{problema}[Problema 4]
    If $R$ is an integral domain with unit element, prove that any unit in $R[x]$ must already be a unit in $R$.
    \begin{dem}
        Supóngase que $f(x)=a_0+a_1x+a_2+\cdots + a_nx^n$ es un elemento unitario en $R[x]\implies$ por definición $\exists g(x)=b_0+b_1x+b_2+\cdots + b_nx^n\ni f(x)g(x)=1$. De esto, nótese que el grado de los polinomios,
        \begin{align*}
            \operatorname{gr}(f(x)g(x)) &= \operatorname{gr}(1)\\
            &= 0
        \end{align*}
        Como $\operatorname{gr}(f(x)),\operatorname{gr}(g(x))\geq 0$, entonces $\operatorname{gr}(f(x))=\operatorname{gr}(g(x))=0\implies f(x)=a_0$ y $g(x)=b_0$ para $a_0,b_0\in R$, es decir $a_0b_0=1$. Por lo tanto, cualquier unidad en $R[x]$ debe ser una unidad en $R$. 
    \end{dem}  
\end{problema}

\begin{problema}[Problema 5]
    Let $R$ be a commutative ring with no nonzero nilpotent elements (that is, $a^n=0$ implies $\left.a=0\right)$. If $f(x)=a_0+a_1 x+\cdots+a_m x^m$ in $R[x]$ is a zero-divisor, prove that there is an element $b \neq 0$ in $R$ such that $b a_0=b a_1=\cdots=b a_m=0$. 
    \begin{dem}
        Sea $f(x)=a_0+a_1 x+\cdots+a_m x^m$ en $R[x]$ el cual es un divisor de cero, entonces debe existir para $i\in \mathbb{N}_0$, $g(x)=b_ix^i+b_{i+1}x^{i+1} +\cdots+b_n x^n\neq 0$ en $R[x]$ tal que $f(x)g(x)=0$. Entonces, 
        \begin{align*}
            f(x)g(x) &= (a_0+a_1 x+\cdots+a_m x^m)(b_ix^i+b_{i+1}x^{i+1} +\cdots+b_n x^n)\\
            \begin{split}
                &= a_0\left[b_ix^i+b_{i+1}x^{i+1} +\cdots+b_n x^n\right] +a_1 x\left[b_ix^i+b_{i+1}x^{i+1} +\cdots+b_n x^n\right]+\\
                &+\cdots + a_m x^m\left[b_ix^i+b_{i+1}x^{i+1} +\cdots+b_n x^n\right]\\
                &= (a_0b_ix^i+a_0b_{i+1}x^{i+1} +\cdots+a_0b_n x^n)+(a_1 b_ix^{i+1}+a_1b_{i+1}x^{i+2} +\cdots+a_1 b_n x^{n+1})+\\
                &+\cdots + (a_mb_ix^{i+m}+a_mb_{i+1}x^{i+1+m} +\cdots+a_mb_n x^{n+m})\\
                &= a_0b_ix^i+(a_0b_{i+1}+a_1b_1)x^{i+1}+(a_0b_{i+2}+a_1b_{i+1}+a_2b_i)x^{i+2}+\cdots+ a_mb_nx^{m+n}
            \end{split}\\
            &= 0
        \end{align*}
        Nótese que los coeficientes deben ser 0, en donde $k\in \{0,1,\cdots, m\}$ es decir: 
        \begin{align*}
            a_0b_i &= 0\\
            a_0b_{i+1}+a_1b_1 &= 0\\
            a_0b_{i+2}+a_1b_{i+1}+a_2b_i &= 0\\
            \vdots \\
            a_0b_{i+k}+a_1b_{i+(k-1)}+\cdots 
            a_2k_i &= 0
        \end{align*}
        Entonces, por hipótesis, tenemos $a^i=0\implies a=0$, es decir que $b_n^i\neq 0\implies b_n\neq 0 $ y si $b:= b_i^{m+1}$ tenemos que:  
        $$b a_0=b a_1=\cdots=b a_m=0$$
    \end{dem}
\end{problema}

\begin{problema}[Problema 8]
   Prove that when $F$ is a field, $F\left[x_1, x_2\right]$ is not a principal ideal ring.
\begin{dem}
    Por reducción al absurdo, supóngase que $F[x_1,x_2]$ es un anillo de anillos principales $\implies (x_1,x_2)=(f(x_1,x_2))$ para un $f(x_1,x_2)\in F[x_1,x_2]\implies x_1,x_2$ son irreducibles tal que $x_1=k_1f(x_1,x_2)$ y $x_2=k_2f(x_1,x_2)$. Entonces, de $x_1$, $f(x_1,x_2)$ no debe tener coeficientes 0 de $x_1$ y de $x_2,k_2$ no tiene coeficientes 0 $(\to\gets)$ ya que entonces no se cumpliría $x_1=k_1f(x_1,x_2)$. Por lo tanto, $F\left[x_1, x_2\right]$ no es un ideal de anillos principales.
\end{dem}
\end{problema}

\begin{problema}[Problema 11]
    If $R$ is an integral domain, and if $F$ is its field of quotients, prove that any element $f(x)$ in $F[x]$ can be written as $f(x)=\left(f_0(x) / a\right)$, where $f_0(x) \in R[x]$ and where $a \in R$.
    \begin{dem}
        Sea $f(x)\in F[x]$ tal que 
        $$f(x)=\sum_{i=0}^k\frac{a_i}{b_i}x^i,$$
        en donde $a_i,b_i\neq 0 \in R$, entonces:
        \begin{align*}
            f(x) &=\sum_{i=0}^k\frac{a_i}{b_i}x^i\\
                &= \frac{a_0}{b_0}+ \frac{a_1}{b_1}x+\cdots + \frac{a_k}{b_k}x^k\\
                &= \frac{a_0b_1b_2\cdots b_k}{b_0b_1b_2\cdots b_k}+\frac{a_1b_0b_2\cdots b_k}{b_0b_1b_2\cdots b_k}x+\cdots+\frac{a_kb_0b_1b_2\cdots b_{k-1}}{b_0b_1b_2\cdots b_k}x^k\\
                &= \frac{a_0b_1b_2\cdots b_k+a_1b_0b_2\cdots b_kx+\cdots+a_kb_0b_1b_2\cdots b_{k-1}x^k}{b_0b_1b_2\cdots b_k}\\
                &:= \frac{f_0(x)}{a}
        \end{align*}
        En donde $f_0(x)\in R[x]$ y $a\in R$.
    \end{dem}
\end{problema}

%---------------------------
%\bibliographystyle{apa}
%\bibliography{referencias.bib}

\end{document}