\documentclass[a4paper,12pt]{article}
\usepackage[top = 2.5cm, bottom = 2.5cm, left = 2.5cm, right = 2.5cm]{geometry}
\usepackage[T1]{fontenc}
\usepackage[utf8]{inputenc}
\usepackage{multirow} 
\usepackage{booktabs} 
\usepackage{graphicx}
\usepackage[spanish]{babel}
\usepackage{setspace}
\setlength{\parindent}{0in}
\usepackage{float}
\usepackage{fancyhdr}
\usepackage{amsmath}
\usepackage{amssymb}
\usepackage{amsthm}
\usepackage[numbers]{natbib}
\newcommand\Mycite[1]{%
	\citeauthor{#1}~[\citeyear{#1}]}
\usepackage{graphicx}
\usepackage{subcaption}
\usepackage{booktabs}
\usepackage{etoolbox}
\usepackage{minibox}
\usepackage{hyperref}
\usepackage{xcolor}
\usepackage[skins]{tcolorbox}
%---------------------------

\newtcolorbox{cajita}[1][]{
	 #1
}

\newenvironment{sol}
{\renewcommand\qedsymbol{$\square$}\begin{proof}[\textbf{Solución.}]}
	{\end{proof}}

\newenvironment{dem}
{\renewcommand\qedsymbol{$\blacksquare$}\begin{proof}[\textbf{Demostración.}]}
	{\end{proof}}

\newtheorem{problema}{Problema}
\newtheorem{definicion}{Definición}
\newtheorem{ejemplo}{Ejemplo}
\newtheorem{teorema}{Teorema}
\newtheorem{corolario}{Corolario}[teorema]
\newtheorem{lema}[teorema]{Lema}
\newtheorem{prop}{Proposición}
\newtheorem*{nota}{\textbf{NOTA}}
\renewcommand\qedsymbol{$\blacksquare$}
\usepackage{svg}
\usepackage{tikz}
\usepackage[framemethod=default]{mdframed}
\global\mdfdefinestyle{exampledefault}{%
linecolor=lightgray,linewidth=1pt,%
leftmargin=1cm,rightmargin=1cm,
}




\newenvironment{noter}[1]{%
\mdfsetup{%
frametitle={\tikz\node[fill=white,rectangle,inner sep=0pt,outer sep=0pt]{#1};},
frametitleaboveskip=-0.5\ht\strutbox,
frametitlealignment=\raggedright
}%
\begin{mdframed}[style=exampledefault]
}{\end{mdframed}}
\newcommand{\linea}{\noindent\rule{\textwidth}{3pt}}
\newcommand{\linita}{\noindent\rule{\textwidth}{1pt}}

\AtBeginEnvironment{align}{\setcounter{equation}{0}}
\pagestyle{fancy}

\fancyhf{}









%----------------------------------------------------------
\lhead{\footnotesize Álgebra Moderna}
\rhead{\footnotesize  Rudik Roberto Rompich}
\cfoot{\footnotesize \thepage}


%--------------------------

\begin{document}
 \thispagestyle{empty} 
    \begin{tabular}{p{15.5cm}}
    \begin{tabbing}
    \textbf{Universidad del Valle de Guatemala} \\
    Departamento de Matemática\\
    Licenciatura en Matemática Aplicada\\\\
   \textbf{Estudiante:} Rudik Roberto Rompich\\
   \textbf{Correo:}  \href{mailto:rom19857@uvg.edu.gt}{rom19857@uvg.edu.gt}\\
   \textbf{Carné:} 19857
    \end{tabbing}
    \begin{center}
        MM2035 - Álgebra Moderna - Catedrático: Ricardo Barrientos\\
        \today
    \end{center}\\
    \hline
    \\
    \end{tabular} 
    \vspace*{0.3cm} 
    \begin{center} 
    {\Large \bf  Tarea 13
} 
        \vspace{2mm}
    \end{center}
    \vspace{0.4cm}
%--------------------------
Problemas 3, 6, 7, 8, 10 y 12, sección 3.2.

\section*{Sección 3.2}

$R$ is a ring in all the problems.


\begin{problema}[Problema 3]
    Find the form of the binomial theorem in a general ring; in other words, find an expression for $(a+b)^{n}$, where $n$ is a positive integer.
    \begin{sol}
        Tenemos 2 casos:
        \begin{enumerate}
            \item Si el anillo es conmutativo, se tiene la definición usual del teorema binomial: 
                $$(a+b)^{n} = \sum_{k=0}^n\begin{pmatrix}
                    n\\
                    k
                \end{pmatrix}x^{n-k}y^k$$
            \item Si el anillo no es conmutativo, tenemos que 
            $$(a+b)^{n} = \sum_{k=0}^n(x_1\cdots x_n),$$
            en donde la sumas van sobre todos los elementos de longitud $n$ con $x_i=a$ o $x_i=b$.
        \end{enumerate}

    \end{sol}
\end{problema}

\begin{problema}[Problema 6]
    If $D$ is an integral domain and $D$ is of finite characteristic, prove that the characteristic of $D$ is a prime number.
    \begin{dem}
        Debemos probar que el característico de $D$ es un número primo. Por hipotesis, $D$ es un dominio entero y $D$ es de característica finita $\implies$ por definición, $p$ es el entero más pequeño tal que $pa=0 \quad \forall a\in D$. Por reducción al absurdo, supóngase que $p$ no es un número primo, es decir es un número compuesto $\implies \exists z_1,z_2\in \mathbb{Z}^+\ni p=z_1z_2$. Nótese entonces que $(z_1z_2)a=0 \quad \forall a\in D\implies$ pero esto también nos permite asegurar $(z_1z_2)a^2=(z_1a)(z_2a)=0\implies (z_1a)=0$ o $(z_2a)=0. (\to \gets)$ Pero es una contradicción ya que $p$ es el entero más pequeño y no puede ser 0. Por lo tanto, el característico $p$ de $D$ es un número primo. 
    \end{dem}
\end{problema}

\begin{problema}[Problema 7]
    Give an example of an integral domain which has an infinite number of elements, yet is of finite characteristic.
    \begin{sol}
        Basándonos en los ejemplos usuales del libro, $J_p$ (el anillo de enteros $\mod p$) un ejemplo claro sería $J_p[X]$ el anillo de polinomios sobre el anillo $J_p$.
    \end{sol}
\end{problema}

\begin{problema}[Problema 8]
    If $D$ is an integral domain and if $n a=0$ for some $a \neq 0$ in $D$ and some integer $n \neq 0$, prove that $D$ is of finite characteristic.
    \begin{dem}
        Debemos probar que $D$ es de característica finita. Sea entonces $x\in D \ni$
        $$ (na)x = a(nx)=0,\quad a\neq 0.$$
        $\implies (nx)=0\quad \forall x\in D$, cumpliendo la definición de característico finito para $D$. 
    \end{dem}
\end{problema}

\begin{problema}[Problema 10]
    Show that the commutative ring $D$ is an integral domain if and only if for $a, b, c \in D$ with $a \neq 0$ the relation $a b=a c$ implies that $b=c$.
    \begin{dem}
        Tenemos dos implicaciones: 
        \begin{itemize}
            \item $(\implies)$ Por hipótesis, $D$ es un dominio entero, tenemos que $a\neq 0$, tal que: 
            \begin{align*}
                ab&=ac\\
                (ab-ac)&=0\\
                a(b-c)&=0\\
                b-c&=0\\
                b&=c
            \end{align*}
            \item $(\impliedby)$ Por hipótesis, $a,b,c\in D$ con $b=c$. Por reducción al absurdo, supóngase $D$ no es de dominio entero, es decir $\exists a,b\in D-\{0\}\ni ab=0$. Pero 
            $$0=ab=a\cdot 0\implies a(b-0)\implies b-0=0\implies b=0(\to\gets).$$
            Por lo tanto, $D$ es un dominio entero.  
        \end{itemize}
    \end{dem}
\end{problema}

\begin{problema}[Problema 12]
    Prove that any field is an integral domain.
    \begin{dem}
        Sea $D$ un campo $\implies$ es un anillo conmutativo divisible $\implies$ cada elemento no cero de $D$ es invertible. Por reducción al absurdo, supóngase que $D$ no es de dominio entero, entonces $m,a\in D-\{0\}\ni ma=0\implies m,a\neq 0, m$ es invertible, es decir que $m^{-1}$ existe, tal que: 
        \begin{align*}
            m^{-1}(ma) &= m^{-1}0\\
            (m^{-1}m)a &= 0\\
            (e)a &= 0\\
            a &= 0 (\to\gets)
        \end{align*}
        Por lo tanto, $m,a=0$ y $D$ es de dominio entero. 
    \end{dem}
\end{problema}





%---------------------------
%\bibliographystyle{apa}-
%\bibliography{referencias.bib}

\end{document}