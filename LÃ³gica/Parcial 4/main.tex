\documentclass[a4paper,12pt]{article}
\usepackage[top = 2.5cm, bottom = 2.5cm, left = 2.5cm, right = 2.5cm]{geometry}
\usepackage[T1]{fontenc}
\usepackage[utf8]{inputenc}
\usepackage{multirow} 
\usepackage{booktabs} 
\usepackage{graphicx}
\usepackage[spanish]{babel}
\usepackage{setspace}
\setlength{\parindent}{0in}
\usepackage{float}
\usepackage{fancyhdr}
\usepackage{amsmath}
\usepackage{amssymb}
\usepackage{amsthm}
\usepackage[numbers]{natbib}
\newcommand\Mycite[1]{%
	\citeauthor{#1}~[\citeyear{#1}]}
\usepackage{graphicx}
\usepackage{subcaption}
\usepackage{booktabs}
\usepackage{etoolbox}
\usepackage{minibox}
\usepackage{hyperref}
\usepackage{xcolor}
\usepackage[skins]{tcolorbox}
%---------------------------

\newtcolorbox{cajita}[1][]{
	 #1
}

\newenvironment{sol}
{\renewcommand\qedsymbol{$\square$}\begin{proof}[\textbf{Solución.}]}
	{\end{proof}}

\newenvironment{dem}
{\renewcommand\qedsymbol{$\blacksquare$}\begin{proof}[\textbf{Demostración.}]}
	{\end{proof}}

\newtheorem{problema}{Problema}
\newtheorem{definicion}{Definición}
\newtheorem{ejemplo}{Ejemplo}
\newtheorem{teorema}{Teorema}
\newtheorem{corolario}{Corolario}[teorema]
\newtheorem{lema}[teorema]{Lema}
\newtheorem{prop}{Proposición}
\newtheorem*{nota}{\textbf{NOTA}}
\renewcommand\qedsymbol{$\blacksquare$}
\usepackage{svg}
\usepackage{tikz}
\usepackage[framemethod=default]{mdframed}
\global\mdfdefinestyle{exampledefault}{%
linecolor=lightgray,linewidth=1pt,%
leftmargin=1cm,rightmargin=1cm,
}




\newenvironment{noter}[1]{%
\mdfsetup{%
frametitle={\tikz\node[fill=white,rectangle,inner sep=0pt,outer sep=0pt]{#1};},
frametitleaboveskip=-0.5\ht\strutbox,
frametitlealignment=\raggedright
}%
\begin{mdframed}[style=exampledefault]
}{\end{mdframed}}
\newcommand{\linea}{\noindent\rule{\textwidth}{3pt}}
\newcommand{\linita}{\noindent\rule{\textwidth}{1pt}}

\AtBeginEnvironment{align}{\setcounter{equation}{0}}
\pagestyle{fancy}

\fancyhf{}









%----------------------------------------------------------
\lhead{\footnotesize Álgebra Moderna}
\rhead{\footnotesize  Rudik Roberto Rompich}
\cfoot{\footnotesize \thepage}


%--------------------------

\begin{document}
 \thispagestyle{empty} 
    \begin{tabular}{p{15.5cm}}
    \begin{tabbing}
    \textbf{Universidad del Valle de Guatemala} \\
    Departamento de Matemática\\
    Licenciatura en Matemática Aplicada\\\\
   \textbf{Estudiante:} Rudik Roberto Rompich\\
   \textbf{Correo:}  \href{mailto:rom19857@uvg.edu.gt}{rom19857@uvg.edu.gt}\\
   \textbf{Carné:} 19857
    \end{tabbing}
    \begin{center}
        MM2035 - Álgebra Moderna - Catedrático: Ricardo Barrientos\\
        \today
    \end{center}\\
    \hline
    \\
    \end{tabular} 
    \vspace*{0.3cm} 
    \begin{center} 
    {\Large \bf  Tarea 13
} 
        \vspace{2mm}
    \end{center}
    \vspace{0.4cm}
%--------------------------

\begin{problema}
    Realice la negación del siguiente enunciado:
$(y)((A y \wedge B y) \Longrightarrow((\sim C y \wedge T y) \Longrightarrow((\exists x)(\sim D x \wedge \sim E y))))$
Nota: Se deben negar hasta que el conectivo de negación quede en los predicados correspondientes.
(Valor 30 puntos).
\begin{sol}
    Tenemos:
    \begin{align*}
        \sim \left[(y)((A y \wedge B y) \Longrightarrow((\sim C y \wedge T y) \Longrightarrow((\exists x)(\sim D x \wedge \sim E y)))) \right]\\
        (\exists y)\sim ((A y \wedge B y) \Longrightarrow((\sim C y \wedge T y) \Longrightarrow((\exists x)(\sim D x \wedge \sim E y))))\\
        (\exists y)((A y \wedge B y) \wedge \sim ((\sim C y \wedge T y) \Longrightarrow((\exists x)(\sim D x \wedge \sim E y))))\\
        (\exists y)((A y \wedge B y) \wedge ((\sim C y \wedge T y) \wedge \sim ((\exists x)(\sim D x \wedge \sim E y))))\\
        (\exists y)((A y \wedge B y) \wedge ((\sim C y \wedge T y) \wedge  ((x)\sim(\sim D x \wedge \sim E y))))\\
        (\exists y)((A y \wedge B y) \wedge ((\sim C y \wedge T y) \wedge  ((x) (\sim \sim Dx\vee \sim\sim Ey))))\\
        (\exists y)((A y \wedge B y) \wedge ((\sim C y \wedge T y) \wedge  ((x) (Dx\vee Ey))))\\
    \end{align*}
\end{sol}
\end{problema}

\begin{problema}
    Demostrar que $((\exists x)(F x \vee \sim G x)) \Longrightarrow \sim((x)(\sim F x \wedge G x))$
Nota: $F$ y $G$ son predicados de aridad 1 y hay que utilizar el sistema axiomático definido para los predicados.
\begin{dem}
    Sea 
    \begin{align*}
        ((\exists x)(F x \vee \sim G x)) \Longrightarrow \sim((x)(\sim F x \wedge G x))
    \end{align*}
    Por transpositiva, esto es equivalente a decir: 
    \begin{align*}
        \sim (\sim((x)(\sim F x \wedge G x))) &\implies \sim ((\exists x)(F x \vee \sim G x)) & \\
        \underbrace{(x)(\sim F x \wedge G x)}_{\text{hipótesis}}&\implies (x)\sim(Fx\vee \sim Gx) & \text{Negación}\\
        &\implies (x)(\sim Fx\wedge \sim\sim Gx) & \text{De Morgan}\\
        &\implies \underbrace{(x)(\sim Fx\wedge Gx}_{\text{tesis}}) & \text{Doble negación}
    \end{align*}
    Entonces, la demostración sería: 
    \begin{align*}
        (x)(\sim F x \wedge G x)&\implies&\\
        &\implies \sim Fy\wedge Gy& \text{IE 1}\\
        &\implies \sim Fy& \text{Simp 2}\\
        &\implies Gy& \text{Simp 2}\\
        &\therefore(x)(\sim F x \wedge G x)
    \end{align*}
\end{dem}
\end{problema}

\begin{problema}

En caso de ser estudiante de Matemática, responda haciendo una reflexión de las siguientes preguntas:
\begin{enumerate}
    \item ¿Por qué seleccionó la carrera de Matemática?
    \begin{sol}
        Supongo que por casualidades de la vida llegué a esta carrera. Estuve 2 años en física y luego me percaté que no era lo que yo esperaba, entonces decidí buscar otros rumbos. Mis otras opciones eran computación o ciencia de datos. Sin embargo, esas carreras no estaban centradas en la investigación (lo cual es un aspecto al que yo estoy interesado un montón) y por lo tanto, me di cuenta que esas carreras me decepcionarían eventualmente. Lo más cercano que encontré fue matemática y entonces decidí que ese sería mi camino. 
    \end{sol}
    \item ¿En que ha contribuido la Mátematica en su formación?
    \begin{sol}
        La verdad es que ha sido una experiencia bastante interesante. Actualmente trabajo haciendo algoritmos de Machine Learning y tener un conocimiento en matemática me ha ayudado a tener una mayor comprensión de las cuestiones que trato y hacer mi trabajo mucho más eficiente. 
    \end{sol}
    \item Finalizando la carrera de Matemática, ¿Cómo o dónde considera que los conocimientos adquiridos puedan ser aplicados?
    \begin{sol}
        Probablemente haga un PhD en algo relacionado a inteligencia artificial o computación, en donde tener un conocimiento profundo en matemática es un prerrequisito.  
    \end{sol}
\end{enumerate}


\end{problema}


%---------------------------
%\bibliographystyle{apa}
%\bibliography{referencias.bib}

\end{document}