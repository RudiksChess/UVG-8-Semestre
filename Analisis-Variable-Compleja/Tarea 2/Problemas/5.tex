\section{Problema 5}

\begin{tcolorbox}[colback=blue!15,colframe=blue!1!blue,title=Mapeo expansivo]
	Sea $X$ un subconjunto no vacío de $\mathbb{R}$. Un mapeo $f: X \rightarrow X$ es expansivo si
	$$
	|f(x)-f(y)| \geq|x-y|, \quad \forall \ x, y \in X.$$
\end{tcolorbox}

Demuestre que si  $f:\mathbb{R} \rightarrow \mathbb{R}$ es continua y expansiva, entonces es un homeomorfismo con inversa Lipschitz.
\begin{noter}{Literatura}
	Para la realización de esta prueba, se consultaron las siguientes fuentes: 
	\begin{enumerate}
		\item \cite{georgiev2020multiple}. El teorema 1.9.6 se ofrece una de las generalizaciones y trata un caso específico de un teorema similar. 
		\item \cite{carothers2000real}. Este libro profundiza en las definiciones de homeomorfismos desde su punto analítico y topológico.
		\item \cite{brouwer1911beweis}. Teorema de la invarianza del dominio (deducción avanzada con temas de topología). 
	\end{enumerate}
\end{noter}
\begin{proof}
	A probar: $f$ es un homeomorfismo ([1] biyectiva, [2] continua, [3] inversa continua) que su inversa es Lipschitz. Por hipótesis tenemos un subconjunto $\mathbb{R}$ y un mapeo $f: \mathbb{R}\to\mathbb{R}$ continuo tal que: 
	$$	|f(x)-f(y)| \geq|x-y|, \quad \forall \ x, y \in \mathbb{R}.$$
	$\implies $ Nótese que la inyectividad se cumple trivialmente y además por hipótesis sabemos que es continua. Hace falta comprobar que posee la sobreyectividad para ser biyectiva; y finalmente, comprobar que su inversa también es continua. Estos dos factores se demuestran por el teorema de la invarianza del dominio de Brouwer (\textbf{literatura - inciso 3}). Por lo tanto, tenemos que $f$ es un homeomorfismo. $\implies f$ tiene inversa. Por lo tanto, podemos asumir, tomando como referencia el caso general de \textbf{literatura - inciso 1}: 
	
	$$|x-y|\geq |f^{-1}(x)-f^{-1}(y)|.$$
	Que claramente es Lipschitz con $\alpha=1, A =1$. 
\end{proof}