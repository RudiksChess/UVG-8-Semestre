Clase: 15/11/2022


\begin{teorema}[5K]
    Si $F$ es un campo de característica 0 y $a,b\in F$ son algebraicos, entonces exiten $c\in F(a,b)$ tal que $F(a,b)=F(c)$
    \begin{dem}
        Sean $f(x),g(x)\in F[x]$ ambos irreducibles sobre $F$ y $f(a)=f(b)=0$. Sea $k$ una extensión de $F$ es la que $f(x)$ y $g(x)$ se factorizan como el producto de polinomios lineales en $K[x]$. 
        Por el corolario al lema 5.6, todas las raíces de $f(x)$ y de $g(x)$ son distintos (i.e. no tienen raíces de multiplicidad 2 o mayor). Sean $\{a=a_1,\cdots, a_{gr(f)}\}\subseteq K$ las raíces de $f(x)$ y $\{b=b_1,\cdots, b_{gr(g)}\}\subseteq K$ las raíces de $g(x)$.\bigbreak

        Si $j\neq 1\implies$ la ecuación $a_i+\lambda b_j=a_1+\lambda b_1=a+\lambda b$ tiene una única solución en $K$, $a_i-a=\lambda(b-b_j)\implies \lambda = \frac{a_i-a}{b-b_j}$. 
        Además, si $F$ es de característica 0 $\implies F$ es infinito $\implies \exists \gamma\in F\ni a_i+\gamma b_j\neq a+\gamma b,\forall i=1,\cdots,gr(f)$ y $j\neq 1$. Sea $c=a+\gamma b$, $c=a+\gamma b\in F(a,b)\implies F(c)\subset F(a,b)$. 
        
        Como $g(b)=0$, $b$ es raíz de $g(x)\in F[x]\implies b$ es raíz de $g(x)\in F(c)[x]$. Además, si $h(x)=f(c-\gamma x)\in F(c)[x]$ y $h(b)=f(c-\gamma b)=f(a)=0$. $\implies$ existe una extensión de $F(c)$ sobre la cual $g(x)$ y $h(x)$ tienen a $x-b$ como factor común. 
        Si $j\neq 1\implies b_j\neq b$ es otra raíz de $g(x)\implies h(b_j)=f(c-\gamma b_j)\neq 0$ ya que $c-\gamma b_j$ no coincide con ninguna raíz de $f(x)$. Además, $(x-b)^2\not| g(x)\implies (x-b)^2\not| (g(x),h(x))\implies x-b=(g(x),h(x))$ sobre alguna extensión de $F(c)$. Por lema previo al lema 5.6, $x-b=(g(x),h(x))$ sobre $F(c)$. Entonces, $x-b\in F(c)[x]\implies b\in F(b)\implies a=c-\gamma b\in F(x)\implies F(a,b)\subseteq F(c)$. En resumen, $F(c)=F(a,b)$.  
    \end{dem}
\end{teorema}

\begin{corolario}
    Toda extensión finita de un campo de característica 0 es simple. 
    \begin{dem}
        Un argumento inductivo (tarea) asegura que si $\alpha_1,\cdots, \alpha_n$ son algebraicos sobre $F$ entonces $\exists c\in F(\alpha_1,\cdots, \alpha_n)\ni F(c)=F(\alpha_1,\cdots, \alpha_n)$
    \end{dem}
\end{corolario}

\begin{cajita}
    Revisar los problemas, 6,7, 8 de la sección 3.2 asignados en la tarea 12 y resolver los problemas 1,2,3,4,5,14 de la sección 5.5
\end{cajita}

\subsection{Elementos de la teoría de Galois}

\begin{teorema}[5L]
    Si $K$ es un campo y $\sigma_1,\cdots, \sigma_n$ son automorfismos distintos de $K$, entonces es imposible encontrar $k_1,\cdots, k_n\in K$, no todos cero, tales que la suma $\sum_{i=1}^n k_i\sigma_i(u)=0$, para todo $u\in K$. 
    \begin{dem}
        Supóngase que existe $\{k_1,\cdots,k_n\}\in K$, por lo menos un $k_i\neq 0\ni \sum_{i=1}^n k_i\sigma_i(u)=0,\forall u\in K$. Elíjase el subconjunto de elementos no nulos de $\{k_1,\cdots, k_n\}$ el subconjunto retiquetado más pequeños de elementos no nulos de $\{k_1,\cdots,k_n\}=\sum_{j=1}^m k_j\sigma_j(u)=0,\forall u\in K$.

        Si $m=1\implies k_1\sigma_1(u)=0,\forall u\in K\implies k_1=0(\to\gets)$

        Si $m>1$, como $\sigma_1,\cdots, \sigma_n$ son distintos, $\exists c\in K\ni \sigma_1(c)\neq \sigma_m(c)$. Ahora bien, $\sum_{j}^m k_j\sigma_j(u)=0,\forall u\in K\implies 0\sum_{j=1}^m k_j\sigma_j(cu)=\sum_{j=1}^m k_j \sigma_j(c)\sigma_j(u),\forall u\in K$. Además, $0=\sigma_1(c)\left(\sum_{j=1}^m k_j \sigma_j(u)\right) = \sum_{j=1}^m k_j \sigma_j(u)=0=0\cdot 0=\sum_{j=1}^m k_j -\sum_{j=1}^mk_j\sigma_1(c)\sigma_j(u)=\sum_{j=1}^n k_j(\sigma_1(c)-\sigma_j(c))\sigma_j(u)=\sum_{j=2}^m k_j(\sigma_1(c)-\sigma_j(c))\sigma(u)$. 

        Como $k_j(\sigma_1(c)-\sigma_j(c))\in K,k_m\neq 0, \sigma_1(c)-\sigma_m(c)\neq 0\implies k_m(\sigma_1(c)-\sigma_m(c))\neq 0\implies \{k_1,\cdots,k_m\}$ no es el subconjunto de elementos no nulos de $\{k_1,\cdots,k_n\}$ tales que $\sum_{j=1}^m k_j \sigma_j(u)=0,\forall u\in K$
        \begin{definicion}
            Si $K$ es un campo y $G$ es un subgrupo de $\mathbb{A}(K)$ (i.e $G$ es un grupo de automorfismos de $K$), entonces el subcampo de $K$ fijado por $G$ es $\{k\in K:\sigma(k)=k,\forall \sigma\in G\}$.
            \begin{nota}
                La definición sigue teniendo sentido si, en vez de un subgrupo de $\mathbb{A}(k)$, se considera un subconjunto no vacío de $\mathbb{A}(K)$. El problema 1 de la tarea 24 se verifica que el subcampo fijado de un conjunto de automorfismos y del subgrupo de $\mathbb{A}(k)$ generado por ese subconjunto, en efecto son iguales. 
            \end{nota}
        \end{definicion}
    \end{dem}
\end{teorema}

\begin{lema}
    Si $K$ es un campo, $G$ es un subgrupo de $\mathbb{A}(K)$, entonces el subcampo de $K$ fijado por $G$, en efecto es un subcampo de $K$.
    \begin{dem}
        Sean $k_1,k_2$ elementos del subcampo de $k$ fijado por $G\implies \sigma(k_1)=k_1$ y $\sigma(k_2)=k_2,\forall \sigma\in G$. 
        
    \end{dem}
\end{lema}
