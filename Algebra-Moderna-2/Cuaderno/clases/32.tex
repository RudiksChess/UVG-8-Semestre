Clase: 17/11/2022

\begin{definicion}
    Si $F$ es un campo y $K$ es una extensión de $F$, entonces el Grupo de Automorfismos de $K$ relativo a $F$, denotado por $G(k,F)$ es 
    $$\{\sigma\in\mathbb{A}(k):\sigma(\alpha)=\alpha,\forall \alpha\in F\}$$
\end{definicion}

\begin{lema}[5.8]
    Si $F$ es un campo y $K$ es una extensión de $F$, entonces, $G(K,F)$ es un subgrupo de de $\mathbb{A}(K)$
    \begin{dem}
        
    \end{dem}
\end{lema}

\begin{cajita}
    Si $K$ es un campo de característica 0 $\implies \mathbb{Q}\subseteq K$. Entonces, el subcampo fijado por cualquier grupo de automorfismos de $K$, siendo campo, debe contener a $\mathbb{Q}$. Entonces todo racional queda invariante por cualquier automorfismo de $K$. 
\end{cajita}

\begin{ejemplo}
    Tenemos los siguientes ejemplos:
    \begin{enumerate}
        \item a
        \item b
        \item c
    \end{enumerate}
\end{ejemplo}

\begin{teorema}[5M]
    Si $K$ es una extensión finita del campo $F$, entonces el $o(G(K,F))\leq [K:F]$
    \begin{dem}
        
    \end{dem}
\end{teorema}

\begin{cajita}
    \begin{nota}
        Si $F$ es un campo, $\sigma\in S_n$ y $r(x_1,x_2,\cdots,x_n)\in F(x_1,\cdots,x_n)\implies\sigma $ puede ser extendido sobre $F(x_1,\cdots, x_n)$ de la función siguiente
        $$r(x_1,\cdots,x_n)\in F(x_1,\cdots,x_n)\to_{\sigma} r(x_{r(1)},\cdots, r_{r(m)})\in F(x_1,x_2,\cdots, x_m)$$
        Vista así,
        $$\sigma \in \mathbb{A}(F(x_1,\cdots,x_m))$$
    \end{nota}
\end{cajita}

\begin{definicion}
    SI $F$ es campo, el campo de las funciones  racionales simétricas, denotado por $S$, es el subcampo de $F(x_1,x_2,\cdots,x_n)$ fijado por $S_n$
\end{definicion}

\begin{definicion}
    Las funciones simétricas elementales de $x_1,\cdots,x_n$ son: 
    \begin{enumerate}
        \item 1, 
        $$\sum_{i=1}^n x_i$$
        \item 2, 
        $$\sum_{i<j}x_ix_j$$
        \item 3, 
        $$\sum_{i\leq j\leq k}x_ix_jx_k$$
        \item 4, 
        $$a_n=\prod_{i\leq k}^nx_i$$
    \end{enumerate}
\end{definicion}

\begin{ejemplo}
    Tenemos múltiples ejemplos. 
\end{ejemplo}