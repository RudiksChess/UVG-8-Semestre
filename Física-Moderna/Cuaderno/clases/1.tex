
\section{Teoría especial de la relatividad}

Clase: 11/07/2022

\begin{definicion} En el vacío la velocidad de la luz 

    $$v=c=\frac{1}{\sqrt{\mu_0 \epsilon_0}},$$
    donde: 
    \begin{enumerate}
        \item $\mu_0=$ Permeabilidad del vacío 
        \item $\epsilon_0=$ Permitividad del vacío.
    \end{enumerate}
    
\end{definicion}

\begin{definicion}[Rapidez de la tierra]
    R = $10^{-4}C$
\end{definicion}

\begin{definicion}[Interferencia constructiva]
    Se define como: 
    $$2(l_1-l_2)=n\lambda,$$
    en donde: 
    \begin{enumerate}
        \item $\lambda$ longitud de onda
        \item $n$ número entero
        \item $l_1=$ longitud entre $AB$
        \item $l_2=$ longitud entre $AD$
    \end{enumerate}
\end{definicion}


\begin{definicion}[Transformaciones galileanas - The Michelson Interferometer]
    Se tiene:
    \begin{enumerate}
        \item $c+v$ la luz acarrea el éter
        \item $c-v$ cuando la luz viene de regreso.
    \end{enumerate}
    $$t_1=\frac{l_1}{c+v}+\frac{l_1}{c-v}=\frac{2l_1c}{c^2-v^2}$$
    $\vdots$

    El éter estacionario no existe. 

\end{definicion}

\begin{definicion}[Postulados de Einstein]
    \begin{enumerate}
        \item El principio de la relatividad: las leyes de los fenómenos electromagnéticos y las leyes de la mecánica son las mismas en todos los marcos de referencia inerciales. Todos los marcos de referencia inerciales son equivalentes.
        \item La velocidad de las luz es independiente del movimiento de la fuente. 
    \end{enumerate}
\end{definicion}

\begin{definicion}[Marco de referencia inicial]
    Es aquel en donde todas las leyes de la física son válidas. 
\end{definicion}

\begin{definicion}[Transformaciones de Lorentz]
    Tenemos:
    \begin{align*}
        x' &=\frac{x-vt}{\sqrt{1-\frac{v^2}{c^2}}}\\
        t' &= \frac{t-(\frac{vx}{c^2})}{\sqrt{1-v^2/c^2}}\\
        y' &= y\\
        z' &= z
    \end{align*}
\end{definicion}