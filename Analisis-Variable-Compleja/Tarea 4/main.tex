\input{Configuraciones/paquetes}

%--------------------------

\begin{document}
\input{Configuraciones/nombres}
%--------------------------
Página 87: 5,6,7, 10 (**)

Página 95: 1, 2 

\section{Página 87}

\begin{problema}[Problema 5 - Conway]
    Let $\gamma$ be a closed rectifiable curve in $\mathbb{C}$ and $a \notin\{\gamma\}$. Show that for $n \geq 2$ $\int_\gamma(z-a)^{-n} d z=0$.
    \begin{dem}
    Debemos probar que para $n\geq 2$, 
    $$ 0=\int_\gamma(z-a)^{-n} d z=\int_\gamma \frac{1}{(z-a)^{n}}dz$$
    

    Sea $G$ una región y sea $f:G\to \mathbb{C}$ una función analítica tal que $f(z)=1$. Por hipótesis, $\gamma$ es un curva cerrada rectificable en $\mathbb{C}$ y $a\not\in \{\gamma\}$, entonces por \textbf{teorema 4.4}, $n(\gamma;w)$ es una constante, en donde $w\in \mathbb{C}-G$ y además $n(\gamma;w)=0,\forall w\in \mathbb{C}-G$. Entonces, se tiene la hipótesis del \textbf{corolario 5.9 al teorema de la fórmula integral de Cauchy}, tal que para $a\in G-\{y\}$ 
    \begin{align*}
        f^{(k)}(a)\cdot n(\gamma;a)&=\frac{k!}{2\pi i}\int_\gamma \frac{f(z)}{(z-a)^{k+1}}dz
        \intertext{Si $k=n-1$ y además se tenía que $f(z)=1$, tal que:}
        f^{(n-1)}(a)\cdot n(\gamma;a)&=\frac{(n-1)!}{2\pi i}\int_\gamma \frac{(1)}{(z-a)^{(n-1)+1}}dz
        \intertext{Ahora bien, nótese que para $n\geq 2$, $f^{(n-1)}(a)$ es 0, entonces:}
        0\cdot n(\gamma;a)&=\int_\gamma \frac{1}{(z-a)^{n}}dz\\
        0&=\int_\gamma (z-a)^{-n}dz.
    \end{align*}
    \end{dem}
\end{problema}

\begin{problema}[Problema 6 - Conway]
    Let $f$ be analytic on $D=B(0 ; 1)$ and suppose $|f(z)| \leq 1$ for $|z|<1$. Show $\left|f^{\prime}(0)\right| \leq 1$.
    \begin{dem}
        Usando el teorema de las desigualdades de Cauchy, se tiene $|f(z)|\leq 1$ para $|z|<1$: 
        \begin{align*}
            |f^{k}(z_0)| &\leq \frac{k!}{R^k}M,\quad k=1,2,3,\cdots
            \intertext{Tenemos que $z_0=0, R=1,M=1$ y $k=1$, tal que:}
            |f^{1}(0)| &\leq \frac{1!}{1^1}1\\
            |f^{1}(0)| &\leq 1.
        \end{align*}
    \end{dem}
\end{problema}

\begin{problema}[Problema 7 - Conway]
    Let $\gamma(t)=1+e^{i t}$ for $0 \leq t \leq 2 \pi$. Find $\int_\gamma\left(\frac{z}{z-1}\right)^n d z$ for all positive integers $n$.
    \begin{dem}
        Debemos probar que para $n\geq 1$, 
        $$ \int_\gamma\left(\frac{z}{z-1}\right)^n d z$$
        
    
        Sea $G$ una región y sea $f:G\to \mathbb{C}$ una función analítica tal que $f(z)=z^n$. Por hipótesis, $\gamma$ es un curva cerrada rectificable en $\mathbb{C}$ y $a\not\in \{\gamma\}$, entonces por \textbf{teorema 4.4}, $n(\gamma;w)$ es una constante, en donde $w\in \mathbb{C}-G$ y además $n(\gamma;w)=0,\forall w\in \mathbb{C}-G$. Entonces, se tiene la hipótesis del \textbf{corolario 5.9 al teorema de la fórmula integral de Cauchy}, tal que para $a=1\in G-\{y\}$ 
        \begin{align*}
            f^{(k)}(1)\cdot n(\gamma;1)&=\frac{k!}{2\pi i}\int_\gamma \frac{f(z)}{(z-1)^{k+1}}dz
            \intertext{Si $k=n-1$ y además se tenía que $f(z)=z^n$, tal que:}
            f^{(n-1)}(1)\cdot n(\gamma;1)&=\frac{(n-1)!}{2\pi i}\int_\gamma \frac{z^n}{(z-1)^{(n-1)+1}}dz\\
            f^{(n-1)}(1)\cdot n(\gamma;1)&=\frac{(n-1)!}{2\pi i}\int_\gamma \left(\frac{z}{z-1}\right)^ndz\\
            f^{(n-1)}(1)\cdot \left(\frac{1}{2\pi i}\int_\gamma \frac{1}{z-1}dz\right)&=\frac{(n-1)!}{2\pi i}\int_\gamma \left(\frac{z}{z-1}\right)^ndz\\
            f^{(n-1)}(1)\cdot \left(\frac{1}{2\pi i}\int_0^{2\pi} \frac{1}{(1+e^{it})-1}ie^{it}dz\right)&=\frac{(n-1)!}{2\pi i}\int_\gamma \left(\frac{z}{z-1}\right)^ndz\\
            f^{(n-1)}(1)\cdot \left(1\right)&=\frac{(n-1)!}{2\pi i}\int_\gamma \left(\frac{z}{z-1}\right)^ndz
        \end{align*}
        \begin{cajita}
            Nótese que  
            \begin{align*}
               n=1 &\quad f^{0}(1) = z^n = (1)^0=1\\
               n=2 &\quad f^{1}(1) = nz^{n-1} = 2\cdot (1)^{1}=2\\
               n=3 &\quad f^{2}(1) = (n-1)nz^{n-2} = (2)\cdot 3 \cdot (1)^{1}=6\\
               n=4 &\quad f^{3}(1) = (n-2)(n-1)nz^{n-3} = (2)(3)(4)(1)(1)^{1}=24\\
               \vdots\\
               n=k &\quad f^{k-1}(1) = n!\\
            \end{align*}
        \end{cajita}
        \begin{align*}
            n!&=\frac{(n-1)!}{2\pi i}\int_\gamma \left(\frac{z}{z-1}\right)^ndz\\
            \frac{2\pi i \cdot n!}{(n-1)!}&=\int_\gamma \left(\frac{z}{z-1}\right)^ndz\\
            \frac{2\pi i \cdot n(n-1)!}{(n-1)!}&=\int_\gamma \left(\frac{z}{z-1}\right)^ndz\\
            2n\pi i&=\int_\gamma \left(\frac{z}{z-1}\right)^ndz\\
        \end{align*}
        \end{dem}
\end{problema}

\begin{problema}[Problema 10 - Conway]
    Use Cauchy's Integral Formula to prove the Cayley-Hamilton Theorem: If $A$ is an $n \times n$ matrix over $\mathbb{C}$ and $f(z)=\operatorname{det}(z-A)$ is the characteristic polynomial of $A$ then $f(A)=0$. (This exercise was taken from a paper by C. A. McCarthy, Amer. Math. Monthly, 82 (1975), 390-391).
    \begin{dem}
        Considerando a \cite{mccarthy1975cayley}, por la fórmula integral de Cauchy aplicada a matrices, tenemos: 
        \begin{align*}
            f(A)&=\frac{1}{2\pi i}\int_{\gamma}\frac{\det(zI-A)}{zI-A}dz\\
                &= \frac{1}{2\pi i}\int_{\gamma}\det(zI-A)(zI-A)^{-1}dz
        \intertext{Además, consideramos que las entradas $(k,l)$ de $(zI-A)^{-1}$ son $((zI-A)^{-1})_{k,l}=[1/\det(zI-A)]\cdot c_{k,l}(z)$ en donde $c_{k,l}$ son los cofactores de $(zI-A)$ y además cada $c_{k,l}$ es un polinomio en $f(z)$ de lo grado a lo más $n-1$.}
                f_{k,l}(A)&= \frac{1}{2\pi i}\int_{\gamma}\det(zI-A)\cdot \frac{c_{k,l}}{\det(zI-A)}dz\\
                &= \frac{1}{2\pi i}\int_{\gamma}c_{k,l} \ dz\\
                &=0
        \end{align*}

    \end{dem}
\end{problema}

\section{Página 95}

\begin{problema}[Problema 1 - Conway]
    Let $G$ be a region and let $\sigma_1, \sigma_2:[0,1] \rightarrow G$ be the constant curves $\sigma_1(t) \equiv a, \sigma_2(t) \equiv b$. Show that if $\gamma$ is closed rectifiable curve in $G$ and $\gamma \sim \sigma_1$ then $\gamma \sim \sigma_2$. (Hint: connect $a$ and $b$ by a curve.)
    \begin{dem}
        Debemos probar que $\gamma \sim \sigma_2$. Ahora bien, nótese que $\sim$ es una relación de equivalencia, entonces si demostramos la transitividad, la prueba está resuelta. Es decir, es necesario probar $\sigma_1\sim \sigma_2$. Por la definición de homotopía, de la hipótesis $\sigma_1,\sigma_2:[0,1]\to G$ dos curvas rectificables y sea $\Gamma[0,1]\times[0,1]\to G$ definido como $\Gamma(s,t)= t(b-a)+a$ tal que: 
        $$\begin{cases}
            \Gamma(s,0)=a =\sigma_1(s) \quad \text{y}\quad \Gamma(s,1)=b=\sigma_2(s) \quad (0\leq s\leq 1)\\
            \Gamma(0,t) =\Gamma(1,t)\quad (0\leq t\leq 1)
        \end{cases}$$
        Por lo tanto, tenemos que $\sigma_1\sim \sigma_2$ y aplicando transitividad, tenemos que:
        $$(\gamma\sim\sigma_1)\wedge(\sigma_1\sim\sigma_2)\implies \gamma\sim \sigma_2.$$
    \end{dem}
\end{problema}

\begin{problema}[Problema 2 - Conway]
    Show that if we remove the requirement " $\Gamma(0, t)=\Gamma(1, t)$ for all $t$ " from Definition 6.1 then the curve $\gamma_0(t)=e^{2 \pi i t}, 0 \leq t \leq 1$, is homotopic to the constant curve $\gamma_1(t) \equiv 1$ in the region $G=\mathbb{C}-\{0\}$.
    \begin{dem}
        De la definición de homotopía, tenemos $\gamma_0, \gamma_1:[0,1] \rightarrow G-\{0\}$ son dos curvas rectificables cerradas en una región  $G$; y si existe una función continua  $\Gamma:[0,1] \times[0,1] \rightarrow G-\{0\}$ definido como $\Gamma(s,t)=e^{2\pi\cdot s\cdot i(1-t) }$ tal que 
$$\begin{cases}\Gamma(s, 0)=\gamma_0(s) & \text { y } \quad \Gamma(s, 1)=1=\gamma_1(s) \quad(0 \leq s \leq 1) \end{cases},$$
cumpliendo la definición de homotopía sin la segunda restricción. 
    \end{dem}
\end{problema}



%---------------------------
\bibliographystyle{plain}
\bibliography{referencias.bib}

\end{document}