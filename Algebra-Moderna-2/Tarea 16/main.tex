\documentclass[a4paper,12pt]{article}
\usepackage[top = 2.5cm, bottom = 2.5cm, left = 2.5cm, right = 2.5cm]{geometry}
\usepackage[T1]{fontenc}
\usepackage[utf8]{inputenc}
\usepackage{multirow} 
\usepackage{booktabs} 
\usepackage{graphicx}
\usepackage[spanish]{babel}
\usepackage{setspace}
\setlength{\parindent}{0in}
\usepackage{float}
\usepackage{fancyhdr}
\usepackage{amsmath}
\usepackage{amssymb}
\usepackage{amsthm}
\usepackage[numbers]{natbib}
\newcommand\Mycite[1]{%
	\citeauthor{#1}~[\citeyear{#1}]}
\usepackage{graphicx}
\usepackage{subcaption}
\usepackage{booktabs}
\usepackage{etoolbox}
\usepackage{minibox}
\usepackage{hyperref}
\usepackage{xcolor}
\usepackage[skins]{tcolorbox}
%---------------------------

\newtcolorbox{cajita}[1][]{
	 #1
}

\newenvironment{sol}
{\renewcommand\qedsymbol{$\square$}\begin{proof}[\textbf{Solución.}]}
	{\end{proof}}

\newenvironment{dem}
{\renewcommand\qedsymbol{$\blacksquare$}\begin{proof}[\textbf{Demostración.}]}
	{\end{proof}}

\newtheorem{problema}{Problema}
\newtheorem{definicion}{Definición}
\newtheorem{ejemplo}{Ejemplo}
\newtheorem{teorema}{Teorema}
\newtheorem{corolario}{Corolario}[teorema]
\newtheorem{lema}[teorema]{Lema}
\newtheorem{prop}{Proposición}
\newtheorem*{nota}{\textbf{NOTA}}
\renewcommand\qedsymbol{$\blacksquare$}
\usepackage{svg}
\usepackage{tikz}
\usepackage[framemethod=default]{mdframed}
\global\mdfdefinestyle{exampledefault}{%
linecolor=lightgray,linewidth=1pt,%
leftmargin=1cm,rightmargin=1cm,
}




\newenvironment{noter}[1]{%
\mdfsetup{%
frametitle={\tikz\node[fill=white,rectangle,inner sep=0pt,outer sep=0pt]{#1};},
frametitleaboveskip=-0.5\ht\strutbox,
frametitlealignment=\raggedright
}%
\begin{mdframed}[style=exampledefault]
}{\end{mdframed}}
\newcommand{\linea}{\noindent\rule{\textwidth}{3pt}}
\newcommand{\linita}{\noindent\rule{\textwidth}{1pt}}

\AtBeginEnvironment{align}{\setcounter{equation}{0}}
\pagestyle{fancy}

\fancyhf{}









%----------------------------------------------------------
\lhead{\footnotesize Álgebra Moderna}
\rhead{\footnotesize  Rudik Roberto Rompich}
\cfoot{\footnotesize \thepage}


%--------------------------

\begin{document}
 \thispagestyle{empty} 
    \begin{tabular}{p{15.5cm}}
    \begin{tabbing}
    \textbf{Universidad del Valle de Guatemala} \\
    Departamento de Matemática\\
    Licenciatura en Matemática Aplicada\\\\
   \textbf{Estudiante:} Rudik Roberto Rompich\\
   \textbf{Correo:}  \href{mailto:rom19857@uvg.edu.gt}{rom19857@uvg.edu.gt}\\
   \textbf{Carné:} 19857
    \end{tabbing}
    \begin{center}
        MM2035 - Álgebra Moderna - Catedrático: Ricardo Barrientos\\
        \today
    \end{center}\\
    \hline
    \\
    \end{tabular} 
    \vspace*{0.3cm} 
    \begin{center} 
    {\Large \bf  Tarea 13
} 
        \vspace{2mm}
    \end{center}
    \vspace{0.4cm}
%--------------------------
Problemas 6 y 7, sección 3.7.

\begin{problema}[Problema 6]
    Prove that the units in a commutative ring with a unit element form an abelian group.
    \begin{dem}
        Sean $e_1,e_2$ unidades en un anillo conmutativo $R$ y por la definición de unidades existen $e_1^{-1},e_2^{-1}\in R$ tal que $e_1e_1^{-1}=1$ y $e_2e_2^{-1}=1$. Ahora comprobaremos las propiedades de grupo: 
        \begin{itemize}
            \item Cerradura. Sea 
            $$(e_1 e_2)\cdot (e_1^{-1}e_2^{-1})=(e_1 e_2)\cdot (e_2^{-1}e_1^{-1})=e_1\cdot(e_2\cdot e_2^{-1})\cdot e_1^{-1}= e_1\cdot 1\cdot e_1^{-1}=1\in R$$
            \item Elemento neutro. En este caso, la hipótesis lo da, $1\in R$.
            \item Inversos. Por hipótesis, tenemos que si $e$ es una unidad, entonces existe $e^{-1}$ tal que $ee^{-1}\in R$.
            \item Asociatividad. Sean $e_1,e_2,e_3$ unidades, tal que $(e_1e_2)e_3= e_1(e_2e_3)$.
        \end{itemize}
        Como $R$ es conmutativo, $ee^{-1}=e^{-1}e$ también se cumple. Por lo tanto, las unidades en un anillo conmutativo con un elemento unitario 1 forma un grupo abeliano. 
    \end{dem}
\end{problema}

\begin{problema}[Problema 7]
    Given two elements $a, b$ in the Euclidean ring $R$ their least common multiple $c \in R$ is an element in $R$ such that $a \mid c$ and $b \mid c$ and such that whenever $a \mid x$ and $b \mid x$ for $x \in R$ then $c \mid x$. Prove that any two elements in the Euclidean ring $R$ have a least common multiple in $R$.
    \begin{dem}
        Por hipótesis, $a,b\in R-\{0\}$. Ahora bien, supóngase que $(c)=\{c\in R\ni a|c,b|c\}$ es un ideal en $R$, sea entonces para $x,y\in (c)$ tal que $a|(x+y)$ y $b|(x+y)$. De la misma manera, nótese que se cumple que para $r\in R, a|xr, a|rx$. Por lo tanto, $(c)$ es un ideal en $R$ para un $c\in R$.  Supóngase ahora que $c$ es el mínimo común múltiple en $R$, por definición sabemos también que $(c)$ es un anillo de ideales principales en donde se cumple $c|x$ en donde $x\in R$. Por lo tanto, $c$ es el mínimo común múltiplo en $R$.
    \end{dem}
\end{problema}

%---------------------------
%\bibliographystyle{apa}
%\bibliography{referencias.bib}

\end{document}