Clase: 04/10/2022

\begin{corolario}
    Si $F$ es un campo, $f(x)\in F[x]$, entonces existe una extensión $E$ de $F$, finita, tal que contiene por lo menos una raíz de $f(x)$ y $[E:F]\leq gr(f)$.
    \begin{dem}
        Si $F$ contiene a todas las raíces de $F$, entonces $E=F$ y $[E:F]=[F:F]\leq gr(f)$. Si $p(x)$ es un factor de $f(x)$, irreducible sobre $F$, entonces por el teorema 5.G existe $E$, extensión de $F$, tal que contiene una raíz de $p(x)$ y por lo tanto, también de $f(x)$ y $[E:F]=gr(p)\leq gr(f)$.
    \end{dem}
\end{corolario}

\begin{definicion}
    Si $F$ es un campo y $f(x)\in F[x]$, $E$ es un campo de descomposición de $f(x)$ sobre $F$, si $E$ es una extensión finita de $F$ en la que $f(x)$ puede factorizarse como producto de polinomios lineales sobre $E$, y esta factorización no es posible sobre ningún subcampo propio de $E$.\bigbreak Es decir, $E$ es un campo de descomposición de $f(x)$ sobre $F$, si $E$ es una extensión finita de $F$ que contiene a todas las raíces de $f(x)$ y $[E:F]$ es mínimo.  
\end{definicion}

\begin{cajita}
    \begin{ejemplo}
        Tenemos 
        $$x^3-2\in \mathbb{Q}(\sqrt[3]{2})[x]$$
        en donde:
        $$x^2-2=(x-\sqrt[3]{2})(x^2+\sqrt[2]{2}x+(\sqrt[3]{2})^2)$$
        en donde $a_2x^2+a_1x+a_0$ es irreducible sobre $\mathbb{Q}(\sqrt[3]{2})$, por el teorema 5.B.C.G $\implies\exists E$ extensión de $\mathbb{Q}(\sqrt[2]{2})\ni \alpha_1=x+(x^2+\sqrt[2]{2}x+(\sqrt[3]{2})^2)\in \mathbb{Q}[x]/(x^2+\sqrt[2]{2}x+(\sqrt[3]{2})^2)\sim \mathbb{Q}(\sqrt[3]{2})(\alpha_1)$. Tenemos: 

        $$[\mathbb{Q}(\sqrt[3]{2})(\sqrt[3]{2}w):\mathbb{Q}(\sqrt[3]{2})]=2$$

        $$[\mathbb{Q}(\sqrt[3]{2},\sqrt[3]{2}w ):\mathbb{Q}]=$$
    \end{ejemplo}
\end{cajita}

\begin{teorema}[5H-Existencia de los campos de descomposición]
    Si $F$ es un campo, $f(x)\in F[x]$ y $gr(f)\geq 1$, entonces existe una extensión de $F$, de grado a lo más $gr(f)!$ tal que contiene a las $gr(f)$ raíces de $f(x)$. 
    \begin{dem}
        Procediendo por inducción sobre $gr(f)$:
        \begin{enumerate}
            \item Si $gr(f)=1\implies f(x)=a_1x+a_0$, con $a_1,a_0\in F,a_1\neq 0\implies -a_0/a_1 \in F$ es raíz de $f(x)\implies F$ es la extensión requerida de $F$, con $[F:F]=1$.
            \item Supóngase el teorema válido para todos los polinomios en $F[x]$ de grado menor a $gr(f)$. 
            \item Por el corolario al teorema 5G, existe $E_0$ extensión de $F$, $[E_0:F]\leq gr(f)$ y $\exists \alpha \in E_0\ni f(\alpha)=0$.$\implies$ por el teorema del residuo (lema 3.1) y su corolario, $\exists q(x)\in E_0(x)\ni (x-\alpha)q(x)=f(x)\implies gr(f)=gr(x-\alpha)+gr(q)=1+gr(q)>gr(q)\implies$ por la hipótesis inductiva, $\exists E$, extensión de $E_0$, $[E:E_0]\leq gr(q)!$ y todas las raíces de $q(x)$ están contenidas en $E$. Ahora bien, $\alpha\in E_0\subseteq E\implies \alpha\in E\implies E$ contiene a todas las raíces de $f(x)$. Además, por el teorema 3A, $[E:F]=[E:E_0][E_0:F]\leq ((gr(f)-1)!)(gr(f))=gr(f)!$        
        \end{enumerate}
    \end{dem}
\end{teorema}

\begin{cajita}
    \begin{nota}
        Si $F$ es un campo y $f(x)\in F[x]$, el teorema 5H garantiza la existencia de $E$, extensión de $F$, que contiene a todas las raíces de $f(x)$ y $[E:F]\leq gr(f)!\implies \{E:[E:F]\in \mathbb{Z}^+ \text{ y todas las raíces de $f(x)$ están contenidas en $E$}\}\neq \varnothing\implies$ existe un elemento de este conjunto $\ni[E:F]$ es mínimo, y en ese caso, un campo de descomposición de $f(x)$ sobre $F$. 
    \end{nota}
    \begin{nota}
        Se verá más adelante que existen campos $F$ y polinomios $f(x)\in F[x]$, cuyos campos de descomposición $E$ sobre $F$ alcanzan la cota superior $[E:F]=gr(f)!$. Por ejemplo, $x^3-2\in \mathbb{Q}[x]$, se demostrará que si $E$ es el campo de descomposición de $x^3-2$ sobre $\mathbb{Q}$ entonces $[E:\mathbb{Q}]=6=3!=gr(x^3-2)!$
    \end{nota}
    \begin{nota}
        Si $F$ es un campo, $f(x)\in F[x]$ y $E_1,E_2$ son campos de descomposición de $f(x)$ sobre $F$. ¿Existe alguna relación entre $E_1$ y $E_2$?
    \end{nota}
\end{cajita}