\documentclass[a4paper,12pt]{article}
\usepackage[top = 2.5cm, bottom = 2.5cm, left = 2.5cm, right = 2.5cm]{geometry}
\usepackage[T1]{fontenc}
\usepackage[utf8]{inputenc}
\usepackage{multirow} 
\usepackage{booktabs} 
\usepackage{graphicx}
\usepackage[spanish]{babel}
\usepackage{setspace}
\setlength{\parindent}{0in}
\usepackage{float}
\usepackage{fancyhdr}
\usepackage{amsmath}
\usepackage{amssymb}
\usepackage{amsthm}
\usepackage[numbers]{natbib}
\newcommand\Mycite[1]{%
	\citeauthor{#1}~[\citeyear{#1}]}
\usepackage{graphicx}
\usepackage{subcaption}
\usepackage{booktabs}
\usepackage{etoolbox}
\usepackage{minibox}
\usepackage{hyperref}
\usepackage{xcolor}
\usepackage[skins]{tcolorbox}
%---------------------------

\newtcolorbox{cajita}[1][]{
	 #1
}

\newenvironment{sol}
{\renewcommand\qedsymbol{$\square$}\begin{proof}[\textbf{Solución.}]}
	{\end{proof}}

\newenvironment{dem}
{\renewcommand\qedsymbol{$\blacksquare$}\begin{proof}[\textbf{Demostración.}]}
	{\end{proof}}

\newtheorem{problema}{Problema}
\newtheorem{definicion}{Definición}
\newtheorem{ejemplo}{Ejemplo}
\newtheorem{teorema}{Teorema}
\newtheorem{corolario}{Corolario}[teorema]
\newtheorem{lema}[teorema]{Lema}
\newtheorem{prop}{Proposición}
\newtheorem*{nota}{\textbf{NOTA}}
\renewcommand\qedsymbol{$\blacksquare$}
\usepackage{svg}
\usepackage{tikz}
\usepackage[framemethod=default]{mdframed}
\global\mdfdefinestyle{exampledefault}{%
linecolor=lightgray,linewidth=1pt,%
leftmargin=1cm,rightmargin=1cm,
}




\newenvironment{noter}[1]{%
\mdfsetup{%
frametitle={\tikz\node[fill=white,rectangle,inner sep=0pt,outer sep=0pt]{#1};},
frametitleaboveskip=-0.5\ht\strutbox,
frametitlealignment=\raggedright
}%
\begin{mdframed}[style=exampledefault]
}{\end{mdframed}}
\newcommand{\linea}{\noindent\rule{\textwidth}{3pt}}
\newcommand{\linita}{\noindent\rule{\textwidth}{1pt}}

\AtBeginEnvironment{align}{\setcounter{equation}{0}}
\pagestyle{fancy}

\fancyhf{}









%----------------------------------------------------------
\lhead{\footnotesize Álgebra Moderna}
\rhead{\footnotesize  Rudik Roberto Rompich}
\cfoot{\footnotesize \thepage}


%--------------------------

\begin{document}
 \thispagestyle{empty} 
    \begin{tabular}{p{15.5cm}}
    \begin{tabbing}
    \textbf{Universidad del Valle de Guatemala} \\
    Departamento de Matemática\\
    Licenciatura en Matemática Aplicada\\\\
   \textbf{Estudiante:} Rudik Roberto Rompich\\
   \textbf{Correo:}  \href{mailto:rom19857@uvg.edu.gt}{rom19857@uvg.edu.gt}\\
   \textbf{Carné:} 19857
    \end{tabbing}
    \begin{center}
        MM2035 - Álgebra Moderna - Catedrático: Ricardo Barrientos\\
        \today
    \end{center}\\
    \hline
    \\
    \end{tabular} 
    \vspace*{0.3cm} 
    \begin{center} 
    {\Large \bf  Tarea 13
} 
        \vspace{2mm}
    \end{center}
    \vspace{0.4cm}
%--------------------------

\section{Contexto histórico}

\subsection{Primera fase }
\begin{itemize}
    \item En 1744, Euler demostró que $e$ es irracional. 
    \item En 1761, Lambert demostró que $\pi$ es irracionall.
    \item Louville es el precursor de la demostración de que $e$ es trascendental, él fue quien introdujo el concepto de números trascendentes en 1844.
    \begin{itemize}
        \item Para cualquier irracional $\alpha\exists $ una sucesión infinita de racionales $p/q (q>0)\ni$
            $$|\alpha-p/q|<1/q^2$$
        \item Propuso una generalización para la propiedad anterior:
        \begin{nota}[Teorema]
            Para cualquier número algebraico $\alpha$ con grado $n>1$, existe $c=c(\alpha)>0$ tal que $|\alpha-p/q|>c/q^n$ para todos los raciones $p/q (q>0)$.
        \end{nota}
        \item Si no es algebraico, es trascendente.
    \end{itemize}
    \item En 1874, Cantor introdujo el concepto de contable y esto implicó que \textbf{casi} todos los números son trascendentes.
\end{itemize}

\subsection{Segunda fase }

\begin{itemize}
    \item En 1873, Hermite establece la trascendencia de $e$.
    \item Años despupes, Lindemann hace una generalización de Hermite y demuestra la trascendencia de $\pi$ y resuelve el problema histórico de la cuadratura del círculo. 
    \item En 1885, Weierstrass simplifica la prueba. 
    \item Luego Hilbert presenta una prueba mucho más simplificada, la cual es modificada por Hurwitz y Gordan en 1893. 
    \item Nosotros presentamos la prueba de Alan Baker propuesta en su libro Transcendental Number Theory de 1975. \textbf{Esta prueba es similar a la de Hurwitz (la cual es la que está en el Herstein)}.
\end{itemize}

\section{Previos}
\begin{cajita}

\begin{nota}
    Sea $\sum_{j=0}^{r}f^{(j)}(x)$, considerando que $f^{(r+1)}(x)=0$ y que $(d/dx)e^x=e^x$, entonces
        \begin{align*}
            \frac{d}{du}\left[-e^{-u}\left(\sum_{j=0}^{r}f^{(j)}(u)\right)\right] &= -e^{-u}\left(\sum_{j=1}^{r+1}f^{(j)}(u)- \sum_{j=0}^{r}f^{(j)}(u)\right)\\
            &= e^{-u}f(u)
        \end{align*}

        Entonces 
        \begin{align*}
            e^t\int_0^t e^{-u} f(u) d u &= e^t\left[-e^{-u}\left(\sum_{j=0}^{r}f^{(j)}(u)\right)\right]_0^t\\
            &= e^t\left(\sum_{j=0}^{r}f^{(j)}(0)\right)-e^t\left(e^{-t}\sum_{j=0}^{r}f^{(j)}(u)\right)\\
            &= e^t \sum_{j=0}^m f^{(j)}(0)-\sum_{j=0}^m f^{(j)}(t)
        \end{align*}
\end{nota}
\end{cajita}

\begin{cajita}
    \begin{lema}
        Para cualquier polinomio de coeficientes complejos, la función $I: \mathbb{C} \rightarrow \mathbb{C}$ definido por
$$
I(t):=\sum_{j=0}^{m}\left(\mathrm{e}^t f^{(j)}(0)-f^{(j)}(t)\right)
$$
Cumple con:
$$
\forall t \in \mathbb{C} \quad|I(t)| \leq|t| \mathrm{e}^{|t|} \bar{f}(|t|)
$$
donde $\bar{f}$ es el polinomio cuyos coeficientes son los módulos (valor absoluto) de $f$.

\begin{dem}
    El caso de cualquier polinomio resulta del de un monomio $f(x)=x^n$.
En este caso, según el desarrollo de \textbf{serie de la compleja exponencial},

\begin{align*}
    I(t)&=\sum_{j=0}^n \left(e^t\frac{n!(0)^{n-j}}{(n-j)!}-\frac{n!t^{n-j}}{(n-j)!}\right)\\
    &=\mathrm{e}^t n !-\sum_{j=0}^n \frac{n !}{(n-j) !} t^{n-j}\\
    &=n ! \sum_{k=0}^{\infty} \frac{t^{n+1+k}}{(n+1+k) !}
\end{align*}

por lo tanto
$$
\begin{aligned}
|I(t)| & \leq \sum_{k=0}^{\infty} \frac{n !}{(n+1+k) !}|t|^{n+1+k} \\
&=|t|^{n+1} \sum_{k=0}^{\infty} \frac{1}{n+1+k} \frac{n ! k !}{(n+k) !} \frac{|t|^k}{k !} \\
& \leq \frac{|t|^{n+1}}{n+1} \sum_{k=0}^{\infty} \frac{|t|^k}{k !} \\
&=\frac{|t|^{n+1}}{n+1} \mathrm{e}^{|t|} \\
&\leq \mathrm{e}^{|t|} \int_0^{|t|} \bar{f}(s) ds\\
& \leq|t| \mathrm{e}^{|t|} \bar{f}(|t|) .
\end{aligned}
$$
   \end{dem}
    \end{lema}
    
\end{cajita}

\begin{cajita}
    \begin{lema}
Sean $f \in \mathbb{C}[X], I$ como previamente lo definimos y sea $ q_0, \ldots, q_n, \alpha_1, \ldots, \alpha_n \in \mathbb{C}, \alpha_0=0$ y 
$$
J:=\sum_{k=1}^n q_k I\left(\alpha_k\right)
$$
Si $\sum_{k=0}^n q_k \mathrm{e}^{\alpha_k}=0$ entonces $J=-\sum_{j=0}^{m} \sum_{k=0}^n q_k f^{(j)}\left(\alpha_k\right)$.
\begin{dem}
    Ya que $I(0)=0$
$$
\begin{aligned}
J &=\sum_{k=0}^n q_k I\left(\alpha_k\right) \\
&=\sum_{k=0}^n q_k \sum_{j=0}^{m}\left(\mathrm{e}^{\alpha_k} f^{(j)}(0)-f^{(j)}\left(\alpha_k\right)\right) \\
&=\sum_{j=0}^{m}\left(f^{(j)}(0) \sum_{k=0}^n q_k \mathrm{e}^{\alpha_k}-\sum_{k=0}^n q_k f^{(j)}\left(\alpha_k\right)\right) \\
&=-\sum_{j=0}^{m} \sum_{k=0}^n q_k f^{(j)}\left(\alpha_k\right)
\end{aligned}
$$
\end{dem}

    \end{lema}
\end{cajita}

\begin{cajita}
    \begin{lema}
Sea $f(x)=x^{p-1}(Q(x))^p$ con $Q(x) \in \mathbb{Z}[X]$, y sea $j \in \mathbb{N}$.
Para toda la raíz $\alpha$ de $f$, el entero $f^{(j)}(\alpha)$ es:
\begin{itemize}
    \item Igual a $(p-1)! Q(0)^p$ si $(j, \alpha)=(p-1,0)$;
    \item Divisible por $p!$ de lo contrario.
\end{itemize}

Si $l, q \in \mathbb{Z}$ y $Q(x)=R(l x)$ con $R(y)=\prod_{k=1}^n\left(y-l \alpha_k\right) \in \mathbb{Z}[y]$, el entero $q f^{(j)}(0)+\sum_{k=1}^n f^{(j)}\left(\alpha_k\right)$ es:
\begin{itemize}
    \item Congruente a $q(p-1) ! Q(0)^p$ modulo $p !$ si $j=p-1$;
    \item Divisible por $p!$ de lo contrario.
\end{itemize}


    \end{lema}
    \begin{dem}
        Según la regla de Leibniz para derivadas (generalización de regla del producto),

        \begin{align*}
            f^{(j)}(x)&=\sum_{k=0}^{\min\{p,j\}-1}\left(\begin{array}{l}
                j \\
                k
                \end{array}\right) \frac{(p-1) !}{(p-1-k) !} x^{p-1-k}\left(Q^p\right)^{(j-k)}\\
                &=\sum_{k=0}^{\min\{p,j\}-1}\frac{j!}{(j-k)!k!} \frac{(p-1) !}{(p-1-k) !} x^{p-1-k}\left(Q^p\right)^{(j-k)}\\
                &= \sum_{k=0}^{\min\{p,j\}-1} \frac{(p-1)!}{(p-1-k)!k!}\frac{x^{p-1-k}}{(j-k)!}j!(Q^p)^{(j-k)}\\
                &=j ! \sum_{k=0}^{\min\{p,j\}-1}\left(\begin{array}{c}
                p-1 \\
                k
                \end{array}\right) x^{p-1-k} \frac{\left(Q^p\right)^{(j-k)}}{(j-k) !}
        \end{align*}
        
        Por lo tanto:
        \begin{itemize}
            \item Si $j \geq p$, $p!|f^{(j)}(X)$ y $p!|\sum_{k=1}^n f^{(j)}\left(\alpha_k\right)$, entonces  $$r, \ell \in \mathbb{N}, \frac{\left(y^r\right)^{(\ell)}}{\ell !}=\left(\begin{array}{l}r \\ \ell\end{array}\right) y^{r-\ell} \in \mathbb{Z}[y]$$
        \end{itemize}

        \begin{itemize}
            \item Si ($\alpha=0$ y $j<p-1$) o si ($Q(\alpha)=0$ y $j<p$):
            $$f^{(j)}(\alpha)=0$$
            \item Si $j=p-1$: 
                $$f^{(p-1)}(0)=(p-1) ! Q(0)^p$$
        \end{itemize}

            \end{dem}
\end{cajita}

\begin{cajita}
    \begin{lema}
        Si $a$  es un número real, mostrar $\left(a^m / m !\right) \rightarrow 0$ as $m \rightarrow \infty$.
        \begin{dem}
            Sea $x_n=a^n / n!$. Es suficiente probar que  $\left|x_{n+1} / x_n\right| \rightarrow 0$. Nótese que
$$
\left|\frac{x_{n+1}}{x_n}\right|=\frac{|a|}{n+1} \rightarrow 0 \quad \text { as } n \rightarrow \infty .
$$
Por lo tanto, $x_n \rightarrow 0$ as $n \rightarrow \infty$.
        \end{dem}

    \end{lema}
\end{cajita}

\section{Demostración}
\begin{teorema}
    $e$ es trascendente.
    \begin{dem}
        Sea $f(x)$ cualquier polinomio con  $m=\operatorname{gr}(f(x))$ y sea $f^{(j)}(x)$ la $j-$ésima derivada de $f(x)$, si
$$
I(t)=e^t\int_0^t e^{-u} f(u) d u,
$$
donde $t$  es un número complejo arbitrario y la integral es de 0 a $t$, entonces por la \textbf{Nota 1}:
$$
I(t)=e^t \sum_{j=0}^m f^{(j)}(0)-\sum_{j=0}^m f^{(j)}(t) .
$$
Además, si $\bar{f}(x)$ denota el polinomio obtenido de  $f$  al reemplzar cada coeficiente por su valor absoluto (módulo) entonces por \textbf{Lema 1} entonces
$$
|I(t)| \leqslant \int_0^t\left|e^{t-u} f(u)\right| d u \leqslant|t| e^{|t|} \vec{f}(|t|) .
$$
\textbf{Por reducción al absurdo}, supóngase que $e$ no es trascendente, entonces $e$ es algebraico, tal que 
$$
q_0+q_1 e+\ldots+q_n e^n=0
$$
para algunos enteros $n>0, q_0 \neq 0, q_1, \ldots, q_n$. Comparamos estimados para 
$$
J=q_0 I(0)+q_1 I(1)+\ldots+q_n I(n),
$$
donde $I(t)$ es definido como arriba con
$$
f(x)=x^{p-1}(x-1)^p \ldots(x-n)^p,
$$
donde $p$ denota un primo muy largo. Del \textbf{Lema 2}, tenemos:
$$
J=-\sum_{j=0}^m \sum_{k=0}^n q_k f^{(j)}(k),
$$

en donde por \textbf{Lema 3}, $m=(n+1) p-1$. Ahora claramente$f^{(j)}(k)=0$ if $j<p, k>0$ y si $j<p-1, k=0$, y entonces para todo $j, k$ otro que $j=p-1, k=0, f^{(j)}(k)$ es un entero divisible por $p!$.Además, tenemos que 
$$
f^{(p-1)}(0)=(p-1) !(-1)^{n p}(n !)^p,
$$
donde, si $p>n, f^{(p-1)}(0)$ es un entero divisible por $(p-1)!$ pero no por $p!$. 


Ahora bien, tenemos que si además $p>\left|q_0\right|$, entonces $J$ es un entero no cero divisible por $(p-1)!$ y entonces $|J| \geqslant(p-1) !$. 

Por el \textbf{Lema 1}, 
$$
|I(t)|  \leqslant|t| e^{|t|} \vec{f}(|t|) .
$$

\begin{align*}
    |J| = \sum_{k=0}^{m}|q_k||I(\alpha_k)|<\sum_{k=0}^m |q_k|e^k
    \intertext{Si $M=\sum_{k=0}^n|q_k|e^k$ tenemos $|J|\leq Mn\overline{f}(n)$, por lo tanto.}
    |J|\leq M(n\overline{Q}(n))^p = Mc^p
\end{align*}

Finalmente, nótese que si para un $i\to \infty$  
\begin{align*}
    \frac{Mc^i}{2^{i-i_0}}\to 0
\end{align*}






Por \textbf{Lema 4}. Por lo tanto, hay contradicción. $\therefore e$ es trascendente.
    \end{dem}
\end{teorema}

%---------------------------
%\bibliographystyle{apa}
%\bibliography{referencias.bib}

\end{document}