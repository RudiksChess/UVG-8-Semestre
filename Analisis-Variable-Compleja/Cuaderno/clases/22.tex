Clase: 18/10/2022


\begin{lema}
    Suponga que $r_0\geq 0$ y es tal que $|a_n|r_0^n\leq M$,$\forall n\in \mathbb{N}$, con $M\geq 0$. Entonces, para $r<r_0$, la serie $\sum_{n=0}^\infty a_n(z-z_0)^n$, converge absoluta y uniformemente sobre los conjuntos:
    $$A_r=\{z:|z-z_0|\leq r\}$$
\end{lema}

\begin{teorema}
    Suponga que $\sum_{n=0}^\infty a_nz^n$ converge para $z=z_0\neq 0$. Entonces, la serie converge
    \begin{enumerate}
        \item absolutamente para $|z|<|z_0|$
        \item uniformemente para $|z|\leq |z_1|$ donde $|z_1|<|z_0|$
    \end{enumerate}
    \begin{dem}
        Como la serie $\sum_{n=0}^\infty a_nz_0^n$ converge $\implies \lim_{n\to\infty}a_nz_0^n=0$
        \begin{dem}
            Dado $\varepsilon=1\exists N\in \mathbb{Z}^+\ni$ si $n\geq N\implies |a_nz_0^n -0|<1\implies |a_n|<|/|z_0|^n,\forall n\geq N$
        \end{dem}
    \end{dem}
\end{teorema}

\begin{prop}
    La serie de potencias $\sum_{n=0}^\infty a_nz^n$ y la serie $\sum_{n=1}^{\infty}na_nz^{n-1}$ tienen el mismo radio de convergencia y sea $0<|z_0|<R$. Como la serie converge en $z=z_0\neq 0$, entonces, $\exists N\in \mathbb{Z}^+\ni |a_n|<1/|z_0|^n,\forall n\geq N$. Por otro lado, 
    \begin{align*}
        |na_nz^{n-1}|\leq n\cdot 1/|z_0|^n \cdot |z|^{n-1},\forall n\geq N
    \end{align*}
    Nótese que 
    \begin{align*}
        \lim_{n\to\infty}\frac{(n+1)\frac{1}{|z_0|^{n+1}|z|^n}}{n\cdot \frac{1}{|z_0|^n}|z|^{n-1}} = \lim_{n\to\infty}\left(\frac{n+1}{n}\right)\left|\frac{z}{z_0}\right| <1\implies |z|<|z_0|
    \end{align*}
\end{prop}

\begin{teorema}
    Para cualquier región (disco cerrado) que está contenido en su círculo de convergencia, una serie de potencias: 
    \begin{enumerate}
        \item Representa una función continua $f(z)$
        \item Puede integrarse término a término y se obtiene la integral de $f(z)$
        \item Puede derivarse término a término y se obtiene $f'(z)$
    \end{enumerate}
    \begin{cajita}
        Si $f(z)=\sum_{n=0}^\infty a_n(z-z_0)^n\implies f$ es analítica en $z_0$
    \end{cajita}
\end{teorema}


\subsection{Serie de Laurent}

