\input{Configuraciones/paquetes}

%--------------------------

\begin{document}
\input{Configuraciones/nombres}
%--------------------------

\begin{problema}
    Con base a la información provista en el caso, calcule la inversión inicial requerida para establecer el huerto de uvas en un área de cinco acres.
    \begin{sol}
        Se tiene: 
        \begin{figure}[H]
            \centering
            \includegraphics[scale=0.5]{imagenes/1.png}
        \end{figure}
        La solución final es de 2,686,000 rupias. 
    \end{sol}
\end{problema}

\begin{problema}
    Estime los flujos de efectivo anuales del proyecto para los años 1–6 y determine el valor terminal con base en los flujos de efectivo de los años 6–60. Suponga que todos los flujos de efectivo ocurren al final de cada año y que no se aplica el impuesto sobre la renta.
    \begin{sol}
        Se tiene:
        \begin{figure}[H]
            \centering
            \includegraphics[scale=0.5]{imagenes/2.png}
        \end{figure}
    \end{sol}
\end{problema}

\begin{problema}
    ¿Cuál es el valor presente neto [=VNA() o =NPV()] y la tasa interna de retorno. [=TIR() o =IRR()] del proyecto en función de los flujos de efectivo proyectados?
    \begin{sol}
        Se tiene:
        \begin{figure}[H]
            \centering
            \includegraphics[scale=0.5]{imagenes/3.png}
        \end{figure}
    \end{sol}
\end{problema}

\begin{problema}
    ¿Qué tan sensible es el valor presente neto del proyecto a los cambios en el rendimiento de producción y el precio de venta por kg de rendimiento? ¿Por debajo de qué rendimiento y precio el VAN del proyecto se vuelve negativo?
    \begin{sol}
        Se tiene:
        \begin{figure}[H]
            \centering
            \includegraphics[scale=0.2]{imagenes/4.png}
        \end{figure}
    \end{sol}
\end{problema}






%---------------------------
%\bibliographystyle{apa}
%\bibliography{referencias.bib}

\end{document}