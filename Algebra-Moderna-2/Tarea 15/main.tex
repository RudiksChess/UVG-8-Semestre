\documentclass[a4paper,12pt]{article}
\usepackage[top = 2.5cm, bottom = 2.5cm, left = 2.5cm, right = 2.5cm]{geometry}
\usepackage[T1]{fontenc}
\usepackage[utf8]{inputenc}
\usepackage{multirow} 
\usepackage{booktabs} 
\usepackage{graphicx}
\usepackage[spanish]{babel}
\usepackage{setspace}
\setlength{\parindent}{0in}
\usepackage{float}
\usepackage{fancyhdr}
\usepackage{amsmath}
\usepackage{amssymb}
\usepackage{amsthm}
\usepackage[numbers]{natbib}
\newcommand\Mycite[1]{%
	\citeauthor{#1}~[\citeyear{#1}]}
\usepackage{graphicx}
\usepackage{subcaption}
\usepackage{booktabs}
\usepackage{etoolbox}
\usepackage{minibox}
\usepackage{hyperref}
\usepackage{xcolor}
\usepackage[skins]{tcolorbox}
%---------------------------

\newtcolorbox{cajita}[1][]{
	 #1
}

\newenvironment{sol}
{\renewcommand\qedsymbol{$\square$}\begin{proof}[\textbf{Solución.}]}
	{\end{proof}}

\newenvironment{dem}
{\renewcommand\qedsymbol{$\blacksquare$}\begin{proof}[\textbf{Demostración.}]}
	{\end{proof}}

\newtheorem{problema}{Problema}
\newtheorem{definicion}{Definición}
\newtheorem{ejemplo}{Ejemplo}
\newtheorem{teorema}{Teorema}
\newtheorem{corolario}{Corolario}[teorema]
\newtheorem{lema}[teorema]{Lema}
\newtheorem{prop}{Proposición}
\newtheorem*{nota}{\textbf{NOTA}}
\renewcommand\qedsymbol{$\blacksquare$}
\usepackage{svg}
\usepackage{tikz}
\usepackage[framemethod=default]{mdframed}
\global\mdfdefinestyle{exampledefault}{%
linecolor=lightgray,linewidth=1pt,%
leftmargin=1cm,rightmargin=1cm,
}




\newenvironment{noter}[1]{%
\mdfsetup{%
frametitle={\tikz\node[fill=white,rectangle,inner sep=0pt,outer sep=0pt]{#1};},
frametitleaboveskip=-0.5\ht\strutbox,
frametitlealignment=\raggedright
}%
\begin{mdframed}[style=exampledefault]
}{\end{mdframed}}
\newcommand{\linea}{\noindent\rule{\textwidth}{3pt}}
\newcommand{\linita}{\noindent\rule{\textwidth}{1pt}}

\AtBeginEnvironment{align}{\setcounter{equation}{0}}
\pagestyle{fancy}

\fancyhf{}









%----------------------------------------------------------
\lhead{\footnotesize Álgebra Moderna}
\rhead{\footnotesize  Rudik Roberto Rompich}
\cfoot{\footnotesize \thepage}


%--------------------------

\begin{document}
 \thispagestyle{empty} 
    \begin{tabular}{p{15.5cm}}
    \begin{tabbing}
    \textbf{Universidad del Valle de Guatemala} \\
    Departamento de Matemática\\
    Licenciatura en Matemática Aplicada\\\\
   \textbf{Estudiante:} Rudik Roberto Rompich\\
   \textbf{Correo:}  \href{mailto:rom19857@uvg.edu.gt}{rom19857@uvg.edu.gt}\\
   \textbf{Carné:} 19857
    \end{tabbing}
    \begin{center}
        MM2035 - Álgebra Moderna - Catedrático: Ricardo Barrientos\\
        \today
    \end{center}\\
    \hline
    \\
    \end{tabular} 
    \vspace*{0.3cm} 
    \begin{center} 
    {\Large \bf  Tarea 13
} 
        \vspace{2mm}
    \end{center}
    \vspace{0.4cm}
%--------------------------

Problema 4, sección 3.6.

\section*{Sección 3.6}
\begin{problema}[Problema 4]
    Prove that if $K$ is any field which contains $D$ then $K$ contains a subfield isomorphic to $F$. (In this sense $F$ is the smallest field containing $D$.)
    \begin{dem}
        Debemos probar que $K$ contiene un subcampo isomorfo a $F$. Proponemos una función $\phi: F\to K$ de la forma 
        $$\phi([x,y])=xy^{-1}$$
        Entonces, comprobaremos las siguientes propiedades: 
        \begin{itemize}
            \item Función bien definida. 
                \begin{itemize}
                    \item Sea $[x,y]\in F$, en donde $x,y\in D$ y $y\neq0\implies x,y\in K\implies x,y^{-1}\in K\implies xy^{-1}\in K$.
                    \item Supóngase que $[x,y]=[x_0,y_0]$, en donde $y,y_0\neq 0$. Ahora bien, tenemos:
                    
                    \begin{align*}
                               xy_0 &= x_0y\\
                               xy_0(y^{-1}y_0^{-1}) &= x_0y(y^{-1}y_0^{-1})\\
                               xy^{-1} &= x_0y_0^{-1}\\
                               \phi[x,y] &= \phi[x_0,y_0]\\
                    \end{align*}
                    Por lo tanto, hemos probado las dos propiedades para ser una función bien definida. 
                    
                \end{itemize}
                \item Ahora bien, intentaremos probar que $\phi$ es isomorfismo, primero demostrando que es homomorfismo y luego que es inyectivo. 
                    \begin{itemize}
                        \item Sea $\phi$ un homomorfismo, tal que: 
                        \begin{enumerate}
                            \item Para la suma. Sea 
                            \begin{align*}
                                \phi([x_1,y_1]+[x_2,y_2]) &= \phi([x_1y_2+x_2y_1,y_1y_2])\\
                                &= (x_1y_2+x_2y_1)(y_1y_2)^{-1}\\
                                &= (x_1y_2+x_2y_1)(y_1^{-1}y_2^{-1})\\
                                &= x_1y_1^{-1} +x_2y_2^{-1}\\
                                &= \phi([x_1,y_1])+\phi([x_2,y_2])
                            \end{align*}
                            \item Para la multiplicación. Sea 
                            \begin{align*}
                                \phi([x_1,y_1]\cdot[x_2,y_2]) &= \phi([x_1x_2,y_1y_2])\\
                                &= x_1x_2(y_1y_2)^{-1}\\
                                &= x_1(y_1)^{-1}x_2(y_2)^{-1}\\
                                &= \phi([x_1,y_1])\cdot\phi([x_2,y_2])
                            \end{align*} 
                            
                        \end{enumerate}
                        Por lo tanto, $\phi$ es un homomorfismo para la suma y la multiplicación.
                    \end{itemize}
                    \item Ahora bien, demostraremos que también se cumple la inyectividad. Sea
                    \begin{align*}
                        \phi([x_1,y_1]) &= \phi([x_2,y_2])\\
                        x_1y_1^{-1} &= x_2y_2^{-1}\\
                        x_1y_1^{-1}(y_1y_2) &= x_2y_2^{-1}(y_1y_2)\\
                        x_1y_2 &= x_2y_2\\
                        [x_1,y_1] &= [x_2,y_2]
                    \end{align*}
        \end{itemize}
        Por lo tanto, se cumple una de las definición de isomorfismo para $\phi([x,y])$ y $F$. Ahora, nos hace falta verificar que $\phi(F)$ es un subcampo de $K$, pero esto se cumple por la definición de  homomorfismo inyectivo el cual se sumerge entre dos anillos. 
    \end{dem}
\end{problema}

%---------------------------
%\bibliographystyle{apa}
%\bibliography{referencias.bib}

\end{document}