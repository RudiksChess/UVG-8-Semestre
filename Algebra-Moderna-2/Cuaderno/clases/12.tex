Clase: 16/08/2022

\begin{definicion}
    Si $F$ es un campo, 

    $$p(x)=\sum_{i=0}^m a_ix_i,\quad q(x)=\sum_{j=0}^n b_jx^j\in F[x],$$
    entonces:
    \begin{enumerate}
        \item $p(x)=q(x)$, si y solo si, $m=n$ y $a_i=b_i,1\leq i\leq m$.
        \item $p(x)+q(x)=\sum_{k=0}^{\max\{m,n\}}(a_k+b_k)x^k,a_k=0$ para $k>m$ y $b_k=0$ para $k>n$
        \item $p(x)q(x)=\sum_{k=0}^{m+n}\left(\sum_{l=0}^k a_kb_{k-l}\right)x^k, a_l=0$ para $l>m$ y  $b_{k-l}=0$ para $k-l>n$. 
    \end{enumerate}
\end{definicion}

\begin{prop}
    Si $F$ es un campo, entonces $(F[x],+,\cdot)$ es un anillo conmutativo con elemento neutro multiplicativo. 
\end{prop}

\begin{definicion}
    Si $F$ es un campo y $f(x)=\sum_{i=0}^na_ix^i\in F[x], a_n\neq 0$, entonces el \textbf{grado} de $f(x)$ es $gr(f)=n$. No está definido el grado del polinomio cero y si $gr(f)=0$, entonces $f(x)$ se dice constante. 
\end{definicion}

\begin{lema}[3.17]
    Si $F$ es un campo y $f(x),g(x)\in F[X]-\{0\}$, entonces $gr(fg)=gr(f)+gr(g)$.
    \begin{dem}
        Se deduce directamente de la definición de producto en $F[x]$.
    \end{dem}
\end{lema}

\begin{corolario}
    Si $F$ es un campo y $f(x),g(x)\in F[x]-\{0\}$, entonces $gr(f)\leq gr(fg)$
    \begin{dem}
        Sea $0\leq gr(g)\implies gr(f)=gr(f)+0\leq gr(f)+gr(g)=gr(fg)$.
    \end{dem}
\end{corolario}

\begin{corolario}
    Si $F$ es un campo, entonces $(F[x],+,\cdot)$ es un dominio entero. 
\end{corolario}

\begin{definicion}
    Si $F$ es un campo, entonces el campo de cocientes del dominio entero $(F[X],+,\cdot)$ es $(F(x),+,\cdot)$, el campo de las funciones racionales en $x$ sobre $F$. 
\end{definicion}

\begin{prop}
    Si $F$ es un campo, entonces la función $\operatorname{gr}:F[X]-\{0\}\to \mathbb{Z}^+\cup\{0\}$ cumple: 
    \begin{enumerate}
        \item $\operatorname{gr}(f)\in \mathbb{Z}^+\cup\{0\},\forall f\in F[X]-\{0\}$
        \item $\operatorname{gr}(f)\leq \operatorname{gr}(f,g),\forall f,g\in F[X]-\{0\}$
    \end{enumerate}
\end{prop}

\begin{lema}[3.18 - Algoritmo de la división]
    Si $F$ sea un campo, $f(x),g(x)\in F[X]$ y $g(x)\neq 0$, entonces existen $q(r),r(x)\in F[X]$ tales que $f(x)=q(x)g(x)+r(x)$, con $r(x)<0$ o $gr(r)<gr(g)$.
    \begin{dem}
        Si $gr(f)<gr(g)\implies q(x)=0$ y $r(x)=f(x)$
    \end{dem}
\end{lema}


\begin{teorema}[3H]
    Si $F$ es un campo, entonces $(F[X],+,\cdot)$ es un anillo euclideano.
    \begin{dem}
        Se deduce directamente de las definiciones y propiedades de $F[X]$ y $\operatorname{gr}$ y del lema 3.18. 
    \end{dem}
\end{teorema}

\begin{lema}[3.19]
    Si $F$ es un campo, entonces $(F[x],+,\cdot)$ es un anillo de ideales principales. 
    \begin{dem}
        Se deduce directamente de los teoremas 3D y 3H. 
    \end{dem}
\end{lema}

\begin{lema}[3.20]
    Si $F$ es un campo, entonces $f(x),g(x)\in F[X]-\{0\}$ siempre tiene un máximo común divisor $d(x)\in F[x]$ y es tal que existen $\lambda(x),\delta(x)\in F[x]$ tales que $d(x)=\lambda(x)f(x)+\delta(x)g(x)$.
    \begin{dem}
        Se deduce directamente del teorema 3H y el lema 3.8
    \end{dem} 
\end{lema}

\begin{definicion}
    Si $F$ es un campo, $p(x)\in F[X]$ es irreducible sobre $F$ si $p(x)=g(x)h(x)$, con $g(x)h(x)\in F[X]-\{0\}$, entonces $\operatorname{gr}(g)=0$ o $\operatorname{gr}(h)=0$
\end{definicion}

\begin{ejemplo}
    $x^2+1$ es irreducible sobre $\mathbb{Q}$ pero no es irreducible sobre $\mathbb{C}$.
\end{ejemplo}
\begin{lema}[3.21]
    Si $F$ es un campo, entonces todo polinomio en $F[X]$ puede factorizarse de manera única, salvo asociación, como producto de un número finito de polinomios de $F[X]$ irreducibles sobre $F$. 
    \begin{dem}
        Se deduce directamente de los teoremas $3E$ y $3H$.
    \end{dem}
\end{lema}

\begin{lema}[3.22]
    Si $F$ es un campo, el ideal generado por el polinomio $p(x)$ de $F[X]$ es un ideal maximal del anillo de polinomios si y solo si, $p(x)\in F[x]$ es irreducible sobre $F$. 
    \begin{dem}
        Se deduce de los lemas 3.14, 3.21 y teorema 3H. 
    \end{dem}
\end{lema}
